\chapter{多复变数全纯函数\label{chap1}}
\section{全纯函数\label{sec1.1}}
\subsection{全纯函数的定义}
我们用$\mathbb{C}$\index[symbolindex]{\textbf{点集}!$\MC$}表示复数域,$\mathbb{C}^n$\index[symbolindex]{\textbf{点集}!$\MC^n$}表示复数域上的线性空间:
\[\mathbb{C}^n=\{(z_1,\cdots,z_n)\colon z_j\in\mathbb{C},\quad j=1,\cdots,n\}.\]
设$z=(z_1,\cdots,z_n),w=(w_1,\cdots,w_n)$是$\mathbb{C}^n$中两个点,定义它们的内积为\index[symbolindex]{\textbf{其它符号}!$\langle z$,$w \rangle$}
\[\langle z,w \rangle =\sum_{j=1}^n z_j\bar{w}_j ,\]
由此产生$z$的模为\index[symbolindex]{\textbf{其它符号}! $\vert z\vert$}$|z|=\langle z,z \rangle^{\frac12}=(\sum\limits_{j=1}^n |z_j|^2)^{\frac12}$.这样$\mathbb{C}^n$就是一个$n$维Hilbert空间\index{H!Hilbert空间}.


$\mathbb{C}^n$中连通开集$\Omega$称为\textbf{域}\index{Y!域}.当$\Omega$有界时,就称$\Omega$为\textbf{有界域}\index{Y!域!有界域}.下面两类简单的有界域值得我们特别注意.


设$a=(a_1,\cdots,a_n)\in\mathbb{C}^n,r=(r_1,\cdots,r_n),r_j>0,j=1,\cdots,n,$称\index[symbolindex]{\textbf{点集}!$P(a$,$r)$}
\[P(a,r)=\{(z_1,\cdots,z_n)\colon |z_j-a_j|<r_j,\quad j=1,\cdots,n\}\]
为以$a$为中心、$r$为半径的\textbf{多圆柱}\index{D!多圆柱}.特别,当$a=0,r_j=1,j=1,\cdots,n$时,称它为单位多圆柱,记为$U^n$\index[symbolindex]{\textbf{点集}!$U^n$},即
\[U^n=\{(z_1,\cdots,z_n)\colon |z_j|<1,j=1,\cdots,n\}.\]
显然,当$n=1$时,它就是单位圆盘.


以$a=(a_1,\cdots,a_n)$为中心,$\rho>0$为半径的\textbf{球}\index{Q!球}是指\index[symbolindex]{\textbf{点集}!$B(a$,$\rho)$}
\[B(a,\rho)=\left\{(z_1,\cdots,z_n)\colon\sum_{j=1}^{n}|z_j-a_j|^2 <\rho^2\right\}.\]
特别,当$a=0,\rho=1$时,称它为单位球,记为$B_n$\index[symbolindex]{\textbf{点集}!$B_n$},即
\[B_n=\left\{(z_1,\cdots,z_n)\colon \sum_{j=1}^n |z_j|^2 <1\right\}.\]
当$n=1$时,它也是单位圆盘.


$U^n$和$B_n$都是单位圆盘在$\mathbb{C}^n$中的推广. 下面我们将看到(见定理\ref{thm2.3.15}和定理\\
\ref{thm2.5.15}),它们不是全纯等价\index{Q!全纯等价}的.因此,$U^n$和$B_n$上的函数论有很大的区别.


为引进全纯函数的概念,我们要讨论多重幂级数的性质.先从多重级数讲起.


给定依赖两个指标的数列$\{a_{jk}\}$,称$\sum\limits_{j,k=1}^\infty a_{jk}$为\textbf{二重级数}\index{D!多重级数!二重级数},数$S_{m,n}=\sum\limits_{j=1}^m \sum\limits_{k=1}^n a_{jk}$称为它的部分和.如果$\lim\limits_{m\to\infty\atop n\to\infty} S_{m,n}=S$存在,就称上述二重级数收敛,$S$是它的和.


用同样的方法可以定义一般\textbf{多重级数}\index{D!多重级数}收敛的概念.


级数
\[\sum_{\alpha_1,\cdots,\alpha_n=0}^\infty a_{\alpha_1\cdots\alpha_n}(z_1-a_1)^{\alpha_1}\cdots(z_n-a_n)^{\alpha_n}\]
称为\textbf{$n$重幂级数}\index{D!多重级数!$n$重幂级数},它在点$b=(b_1,\cdots,b_n)$收敛是指$n$重级数
\[\sum_{\alpha_1,\cdots,\alpha_n=0}^\infty a_{\alpha_1\cdots\alpha_n}(b_1-a_1)^{\alpha_1}\cdots(b_n-a_n)^{\alpha_n}\]
收敛.


关于多重幂级数,也有类似于单变数中的Abel定理\index{D!定理!Abel定理}.为简单起见,我们讨论$a=0$的情形.
\begin{prop}\label{prop1.1.1}
	如果$n$重幂级数
	\begin{equation}\label{eq1.1.1}
		\sum_{\alpha_1,\cdots,\alpha_n=0}^\infty a_{\alpha_1\cdots\alpha_n}z_1^{\alpha_1}\cdots z_n^{\alpha_n}
	\end{equation}
	在点$b=(b_1,\cdots,b_n)$收敛,这里$b_j\neq0,j=1,\cdots,n,$那么它在闭多圆柱
	\[\bar{P}(0,r)=\{(z_1,\cdots,z_n)\colon |z_j|\le r_j,\quad j=1,\cdots,n\}\]
	中绝对一致收敛,这里$r=(r_1,\cdots,r_n),r_j<|b_j|,j=1,\cdots,n$.
\end{prop}
\begin{proof}
	因为幂级数$\sum\limits_{\alpha_1,\cdots,\alpha_n=0}^\infty a_{\alpha_1\cdots\alpha_n} b_1^{\alpha_1}\cdots b_n^{\alpha_n}$收敛,
	所以存在常数$M$,使得对任意$\alpha_1,\cdots,\alpha_n\ge0$,有
	\[|a_{\alpha_1\cdots\alpha_n}|\le\frac{M}{|b_1|^{\alpha_1}\cdots|b_n|^{\alpha_n}}.\]
	故当$|z_j|\le r_j<|b_j|(j=1,2,\cdots,n)$时,
	\[|a_{\alpha_1\cdots\alpha_n}z_1^{\alpha_1}\cdots z_n^{\alpha_n}|\le M\bigg| \frac{z_1}{b_1}\bigg|^{\alpha_1}\cdots\bigg| \frac{z_n}{b_n}\bigg|^{\alpha_n}\le M\bigg(\frac{r_1}{|b_1|}\bigg)^{\alpha_1}\cdots\bigg(\frac{r_n}{|b_n|}\bigg)^{\alpha_n},\]
	于是
	\begin{align*}
		\sum_{\alpha_1,\cdots,\alpha_n=0}^\infty |a_{\alpha_1\cdots\alpha_n} z_1^{\alpha_1}\cdots z_n^{\alpha_n}|
		&\le M\sum_{\alpha_1=0}^\infty \bigg(\frac{r_1}{|b_1|}\bigg)^{\alpha_1}\cdots\sum_{\alpha_n=0}^\infty\bigg(\frac{r_n}{|b_n|}\bigg)^{\alpha_n}\\
		&=M\bigg(1-\frac{r_1}{|b_1|}\bigg)^{-1}\cdots\bigg(1-\frac{r_n}{|b_n|}\bigg)^{-1}.\qedhere
	\end{align*}
\end{proof}
为简化记号,我们采用如下的记号.对于有序数组$\alpha=(\alpha_1,\cdots,\alpha_n),$其中$\alpha_j$都是非负整数,记\index[symbolindex]{\textbf{其它符号}! $\vert \alpha\vert$,$\alpha$!}\index[symbolindex]{\textbf{其它符号}! $z^\alpha$}
\[|\alpha|=\alpha_1+\cdots+\alpha_n,\quad \alpha!=\alpha_1!\cdots\alpha_n!,\quad z^\alpha=z_1^{\alpha_1}\cdots z_n^{\alpha_n},\]
其中$z=(z_1,\cdots,z_n)$.这样,幂级数\eqref{eq1.1.1}就可以简记为$\sum\limits_\alpha a_\alpha z^\alpha$或者$\sum\limits_{\alpha\ge0}a_\alpha z^\alpha$.
\begin{definition}\label{def1.1.2}
	设$\Omega$是$\mathbb{C}^n$中的域,$f\colon\Omega\to\mathbb{C}$是定义在$\Omega$上的一个复值函数.如果对每一点$a\in\Omega,$存在多圆盘$P(a,\rho)\subset\Omega$和幂级数$\sum\limits_\alpha c_\alpha(z-a)^\alpha,$使得
	\begin{equation}\label{eq1.1.2}
		f(z)=\sum_\alpha c_\alpha (z-a)^\alpha
	\end{equation}
	在$P(a,\rho)$中成立,则称$f$为$\Omega$中的\textbf{全纯函数}\index{Q!全纯函数}.
	
	我们用$H(\Omega)$\index[symbolindex]{\textbf{函数和映射}!$H(\Omega)$}记$\Omega$上全纯函数所成的集.
\end{definition}
设$f\in H(\Omega),$则对每个$a\in\Omega,$\eqref{eq1.1.2}在$a$附近成立.容易看出
\[\frac{\partial^{\alpha_1+\cdots+\alpha_n}f(z)}{\partial z_1^{\alpha_1}\cdots\partial z_n^{\alpha_n}}\bigg|_{z=a}=\alpha !c_{\alpha} .\]
如果记\index[symbolindex]{\textbf{导数}!$\DD^\alpha f$}
\[(\DD^\alpha f)(a)=\frac{\partial^{\alpha_1+\cdots+\alpha_n}f(z)}{\partial z_1^{\alpha_1}\cdots\partial z_n^{\alpha_n}}\bigg|_{z=a},\]
那么$f$在$a$点的展开式\eqref{eq1.1.2}可写为
\[f(z)=\sum_\alpha\frac{(\DD^\alpha f)(a)}{\alpha!}(z-a)^\alpha .\]
\subsection{Cauchy-Riemann方程组\index{C!Cauchy-Riemann方程组}}
设$f=u+\ii v\in H(\Omega).$当固定$z_1,\cdots,z_{j-1},z_{j+1},\cdots,z_n$时,记
\[D_j=\{z_j\in\mathbb{C}\colon (z_1,\cdots,z_{j-1},z_j,z_{j+1},\cdots,z_n)\in\Omega\},\]
那么$z_j\mapsto f(z_1,\cdots,z_{j-1},z_j,z_{j+1},\cdots,z_n)$作为单变量函数在$D_j$中全纯,因而有Cauchy-Riemann方程
\[\frac{\partial u}{\partial x_j}=\frac{\partial v}{\partial y_j},\quad \frac{\partial u}{\partial y_j}=-\frac{\partial v}{\partial x_j},\]
其中$z_j=x_j+iy_j$.


引入记号\index[symbolindex]{\textbf{导数}!$\frac{\partial}{\partial z_j}$,$\frac{\partial}{\partial \bar{z}_j}$}
\begin{equation}\label{eq1.1.3}
	\begin{aligned}
		\frac{\partial}{\partial z_j}
		&=\frac12\left(\frac{\partial}{\partial x_j}-\ii\frac{\partial}{\partial y_j}\right),\\
		\frac{\partial}{\partial \bar{z}_j}
		&=\frac12\left(\frac{\partial}{\partial x_j}+\ii\frac{\partial}{\partial y_j}\right),
	\end{aligned}
\end{equation}
那么上面的Cauchy-Riemann方程可写为
\begin{align*}
	\frac{\partial f}{\partial\bar{z}_j}
	&=\frac12\left(\frac{\partial }{\partial x_j}+\ii\frac{\partial}{\partial y_j}\right)(u+iv)\\
	&=\frac12\left(\frac{\partial u}{\partial x_j}-\frac{\partial v}{\partial y_j}\right)+\frac{\ii}{2}\left(\frac{\partial u}{\partial y_j}+\frac{\partial v}{\partial x_j}\right)=0,\quad j=1,\cdots,n,
\end{align*}
称它为Cauchy-Riemann方程组.


后面我们将证明,如果$\frac{\partial f}{\partial\bar{z}_j}=0$在$D_j(j=1,\cdots,n)$上成立,那么$f\in H(\Omega).$这就是著名的Hartogs定理(定理\ref{thm1.5.14}).
\subsection{唯一性定理}
在单复变中有如下的唯一性定理:“设$G$是复平面$\mathbb{C}$上的域,$f\in H(G).$如果点列$\{z_k\}$在$G$中有聚点,且$f(z_k)=0,k=1,2,\cdots,$那么$f$在$G$上恒等于$0$.”这样的唯一性定理在多复变中不再成立.例如,$f(z_1,z_2)=z_1z_2$在双圆盘$\{(z_1,z_2)\colon|z_1|<1,|z_2|<1\}$中全纯,点列$\left\{\left(0,\frac1k\right)\right\},k=2,3,\cdots,$以$(0,0)$为聚点,且$f\left(0,\frac1k\right)=0,$但$f$在双圆盘中不恒等于$0$.多复变中有下面的
\begin{theorem}[(\textbf{唯一性定理})]\label{thm1.1.3}\index{D!定理!唯一性定理}
	设$\Omega$是$\mathbb{C}^n$中的域,$f\in H(\Omega).$如果$f$在非空开集$E\subset\Omega$上恒等于$0$,那么$f$在$\Omega$上恒等于$0$.
\end{theorem}
\begin{proof}
	命$K=\{z\in \Omega\colon(\DD^\alpha f)(z)=0,\text{对所有}\alpha=(\alpha_1,\cdots,\alpha_n)\},$
	\[K_\alpha=\{z\in \Omega\colon(\DD^\alpha f)(z)=0,\text{对某个}\alpha=(\alpha_1,\cdots,\alpha_n)\}.\]
	由假定$E\subset K$,所以$K$不是空集.显然$K=\bigcap\limits_\alpha K_\alpha .$因为$D^\alpha f$是连续函数,所以$K_\alpha$是闭集,因而$K$也是闭集.任取$a\in K,$因为$f$在$\Omega$中全纯,故存在多圆柱$P(a,r)\subset\Omega,$使得
	\[f(z)=\sum_\alpha \frac{(\DD^\alpha f)(a)}{\alpha!}(z-a)^\alpha =0\]
	在$P(a,r)$中成立,因而$P(a,r)\subset K,$这说明$K$是一个开集.由于$\Omega\setminus K$也是开集,下面的等式
	\[\Omega=K\cup(\Omega\setminus K)\]
	和$\Omega$的连通性矛盾,因为$K$不是空集,故只能$\Omega=K$,即$f$在$\Omega$上恒等于$0$.
\end{proof}
作为唯一性定理的应用,我们可以证明下面的
\begin{theorem}[(\textbf{开映射定理})]\label{thm1.1.4}\index{D!定理!开映射定理}
	设$\Omega$是$\mathbb{C}^n$中的域,$f$是$\Omega$上的非常数的全纯函数,那么$f$把$\Omega$中的开集映成$\mathbb{C}$中的开集.
\end{theorem}
\begin{proof}
	$n=1$时,定理是成立的.现设$n>1.$设$a\in\Omega,$命$Q$为$a$的一个凸邻域(例如可以取$Q$为包含$a$的多圆盘),$Q\subset\Omega.$由唯一性定理知,$f$在$Q$上不能恒等于$f(a),$故能找到$b\in Q,$使得$f(a)\neq f(b).$命$D=\{\lambda\in\mathbb{C}\colon a+\lambda(b-a)\in Q\},$显然$D$非空,且由于$Q$是开集,$D$也是$\mathbb{C}$中的开集.命
	\[g(\lambda)=f(a+\lambda(b-a)),\quad \lambda\in D,\]
	则$g$是$D$上的全纯函数,且$g(0)=f(a)\neq f(b)=g(1),$即$g$不是常数.利用单复变的结果,$g(D)$是包含$g(0)$的一个开集,$g(D)\subset f(Q)$,所以$f(Q)$是一个开集.
\end{proof}
利用开映射定理又可以得到下面的最大模原理.
\begin{theorem}[(\textbf{最大模原理})]\label{thm1.1.5}\index{D!定理!最大模原理}
	设$\Omega$是$\mathbb{C}^n$中的域,$f$是$\Omega$上的非常数的全纯函数,那么$f$的模不可能在$\Omega$的内点达到最大值.
\end{theorem}
\begin{proof}
	如果存在$a\in\Omega$,使得$|f(z)|\le|f(a)|$对所有$z\in\Omega$成立.根据定理\ref{thm1.1.4},$f(\Omega)$是开集,它含在圆盘$\{|w|\le|f(a)|\}$之中,但因是开集,必含在$\{|w|<|f(a)|\}$之中,这就导致$|f(a)|<|f(a)|$的矛盾.
\end{proof}
\section{多圆柱的Cauchy积分公式及其应用\label{sec1.2}}
\subsection{多圆柱的Cauchy积分公式}
在单复变中,Cauchy积分公式的重要性是众所周知的,对于不同的域,Cauchy积分公式有相同的形式.但在多复变中,情况则复杂得多,对于不同的域,有不同的Cauchy积分公式.下面先给出多圆柱上的Cauchy积分公式.
\begin{theorem}\label{thm1.2.1}\index{D!多圆柱的Cauchy积分公式}
	设$\Omega$是$\mathbb{C}^n$中的域,$f\in H(\Omega)$.如果$\bar{P}(a,r)\subset\Omega,$则对$z\in P(a,r)$,我们有
	\begin{equation}\label{eq1.2.1}
		f(z)=\frac1{(2\pi \ii)^n}\int\limits_{|\zeta_1-a_1|=r_1}\cdots\int\limits_{|\zeta_n-a_n|=r_n}\frac{f(\zeta_1,\cdots,\zeta_n)}{(\zeta_1-z_1)\cdots(\zeta_n-z_n)}\mathrm{d}\zeta_1\cdots\mathrm{d}\zeta_n.
	\end{equation}
\end{theorem}

\begin{proof}
	当$n=1$时,这是熟知的圆盘上的Cauchy积分公式.今设定理对$n-1$个变数的全纯函数成立.分别在圆周$|\zeta_2-a_2|=r_2,\cdots,|\zeta_n-a_n|=r_n$上固定$\zeta_2,\cdots,\zeta_n,$则$f(z_1,\zeta_2,\cdots,\zeta_n)$是圆盘$|\zeta_1-a_1|\le r_1$上的全纯函数,由单复变的Cauchy积分公式得
	\begin{equation}\label{eq1.2.2}
		f(z_1,\zeta_2,\cdots,\zeta_n)=\frac1{2\pi\ii}\int\limits_{|\zeta_1-a_1|=r_1}\frac{f(\zeta_1,\zeta_2,\cdots,\zeta_n)}{\zeta_1-z_1}\mathrm{d}\zeta_1
	\end{equation}
对函数$(z_2,\cdots,z_n)\mapsto f(z_1,z_2,\cdots,z_n)$用归纳法的假定,并用\eqref{eq1.2.2}即得
\begin{align*}
	f(z_1,\cdots,z_n)
	&=\frac1{(2\pi\ii)^{n-1}}\int\limits_{|\zeta_2-a_2|=r_2}\cdots\int\limits_{|\zeta_n-a_n|=r_n}\frac{f(z_1,\zeta_2,\cdots,\zeta_n)}{(\zeta_2-z_2)\cdots(\zeta_n-z_n)}\mathrm{d}\zeta_2\cdots\mathrm{d}\zeta_n\\
	&=\frac1{(2\pi\ii)^n}\int\limits_{|\zeta_1-a_1|=r_1}\cdots\int\limits_{|\zeta_n-a_n|=r_n}\frac{f(\zeta_1,\cdots,\zeta_n)}{(\zeta_1-z_1)\cdots(\zeta_n-z_n)}\mathrm{d}\zeta_1\cdots\mathrm{d}\zeta_n.\qedhere
\end{align*}
\end{proof}
如果记$D_j=\{z_j\in\mathbb{C}\colon|z_j-a_j|<r_j\}$,那么多圆柱$P(a,r)$是这$n$个圆盘的拓扑积
\[P(a,r)=D_1\times\cdots\times D_n.\]
它的边界$\partial P$由若干部分组成.例如,$\partial D_1\times D_2\times\cdots\times D_n,\partial D_1\times\partial D_2\times D_3\times\cdots\times D_n,\cdots,\partial D_1\times\cdots\times\partial D_n$都是它的边界的组成部分,其中维数最低的那一部分
\[\partial D_1\times\cdots\times\partial D_n=\{(z_1,\cdots,z_n)\colon |z_j-a_j|=r_j,j=1,\cdots,n\}\]
称为$P(a,r)$的\textbf{特征边界}\index{T!特征边界},记为$\partial_0 P$\index[symbolindex]{\textbf{点集}!$\partial_0 P$}.

多圆柱的Cauchy积分公式\eqref{eq1.2.1}的积分区域不是$P(a,r)$的全部边界,而只是它的边界的一部分——特征边界.这是多复变与单复变的一个重要区别.在单复变中,Cauchy积分公式的积分是在全部边界上进行的.

从多圆柱的Cauchy积分公式可以得到下面一些重要结论.
\begin{theorem}\label{thm1.2.2}
	设$f$在多圆柱$P(a,r)$中全纯,则$f$可以在$P(a,r)$中展开为
	\[f(z)=\sum_\alpha\frac{(\DD^\alpha f)(a)}{\alpha!}(z-a)^\alpha,\quad z\in P(a,r).\]
\end{theorem}
\begin{proof}
	取多圆柱$P(a,\rho),$使得$\bar{P}(a,\rho)\subset P(a,r),$则由定理\ref{thm1.2.1},对$z\in P(a,\rho)$有
	\begin{equation}\label{eq1.2.3}
		f(z)=\frac1{(2\pi\ii)^n}\int\limits_{|\zeta_1-a_1|=\rho_1}\cdots\int\limits_{|\zeta-a_n|=\rho_n}\frac{f(\zeta)}{\prod\limits_{j=1}^{n}(\zeta_j-z_j)}\mathrm{d}\zeta_1\cdots\mathrm{d}\zeta_n,
	\end{equation}
由此得
\begin{equation}\label{eq1.2.4}
	(\DD^\alpha f)(z)
		=\frac{\alpha!}{(2\pi\ii)^n}\int\limits_{|\zeta_1-a_1|=\rho_1}\cdots\int\limits_{|\zeta_n-a_n|=\rho_n}\frac{f(\zeta)}{\prod\limits_{j=1}^{n}(\zeta_j-z_j)^{\alpha_j +1}}\mathrm{d}\zeta_1\cdots\mathrm{d}\zeta_n.
\end{equation}
取$\rho'=(\rho_1',\cdots,\rho_n'),$使$\rho_j'<\rho_j(j=1,\cdots,n)$.于是,当$z\in\bar{P}(a,\rho'),|\zeta_j-a_j|=\rho_j(j=1,\cdots,n)$时,有
\begin{equation}\label{eq1.2.5}
	\frac1{\prod\limits_{j=1}^{n}(\zeta_j-z_j)}=\sum_{\alpha_1,\cdots,\alpha_n=0}^\infty \frac{(z_1-a_1)^{\alpha_1}\cdots(z_n-a_n)^{\alpha_n}}{(\zeta_1-a_1)^{\alpha_1 +1}\cdots(\zeta_n-a_n)^{\alpha_n +1}}
\end{equation}
且级数对$\zeta$是一致收敛的.在\eqref{eq1.2.5}的两端乘$f(\zeta),$并对$\zeta$在$\{|\zeta_j-a_j|=\rho_j,j=1,\cdots,n\}$上积分,再利用\eqref{eq1.2.3}和\eqref{eq1.2.4},即得
\begin{align*}
	f(z)
	&=\sum_\alpha (z_1-a_1)^{\alpha_1}\cdots (z_n-a_n)^{\alpha_n} \frac1{(2\pi\ii)^n}\cdot\\
	&\int\limits_{|\zeta_1-a_1|=\rho_1}\cdots\int\limits_{|\zeta_n-a_n|=\rho_n}\frac{f(\zeta)}{\prod\limits_{j=1}^{n}(\zeta_j-a_j)^{\alpha_j+1}}\mathrm{d}\zeta_1\cdots\mathrm{d}\zeta_n\\
	&=\sum_\alpha\frac{(\DD^\alpha f)(a)}{\alpha!}(z-a)^\alpha .
\end{align*}
由于$\rho_j'$可任意接近$\rho_j,\rho_j$可任意接近$r_j$,故上式在$P(a,r)$中成立.
\end{proof}
这个定理断言,多圆柱上的全纯函数在整个多圆柱上有一个统一的幂级数表达式.以后我们将看到,这样的表达式可以在更一般的域上成立(见定理\ref{thm1.3.3}).

如果在Cauchy积分公式\eqref{eq1.2.1}中对$z$求导数,可得
\[(\DD^\alpha f)(a)=\frac{\alpha!}{(2\pi\ii)^n}\int\limits_{\partial_0 P}\frac{f(\zeta)}{\prod\limits_{j=1}^{n}(\zeta_j-a_j)^{\alpha_j+1}}\mathrm{d}\zeta_1\cdots\mathrm{d}\zeta_n.\]
由此可以得到下面有用的Cauchy不等式\index{C!Cauchy不等式}.
\begin{theorem}\label{thm1.2.3}
	设$\Omega$是$\mathbb{C}^n$中的域,$\bar{P}(a,r)\subset\Omega.$如果$f\in H(\Omega)$,记$M=\sup\{|f(\zeta)|:\zeta\in\partial_0 P\}$,那么
	\begin{equation}\label{eq1.2.6}
		|(\DD^\alpha f)(a)|\le M\frac{\alpha!}{r^\alpha}.
	\end{equation}
\end{theorem}
\begin{proof}
	从上面的等式即得
	\begin{align*}
		|(\DD^\alpha f)(a)|
		&\le \frac{\alpha!}{(2\pi)^n}\int\limits_{\partial_0 P}\frac{f(\zeta)}{\prod\limits_{j=1}^{n}|\zeta_j-a_j|^{\alpha_j+1}}|\mathrm{d}\zeta_1|\cdots|\mathrm{d}\zeta_n|\\
		&\le\frac{\alpha!}{(2\pi)^n}\frac{M}{\prod\limits_{j=1}^{n}r_j^{\alpha_j+1}}(2\pi)^n r_1\cdots r_n=M\frac{\alpha!}{r^\alpha}.\qedhere
	\end{align*}
\end{proof}
注意,因为$f$在多圆柱$P(a,r)$中全纯,由定理\ref{thm1.2.2}知道,$f$可以在$P(a,r)$中展开成幂级数
\[f(z)=\sum_\alpha\frac{(\DD^\alpha f)(a)}{\alpha!}(z-a)^\alpha=\sum_\alpha a_\alpha (z-a)^\alpha .\]
实际上,不等式\eqref{eq1.2.6}给出了展开式系数的估计
\[|a_\alpha|=\frac{|(\DD^\alpha f)(a)|}{\alpha!}\le \frac{M}{r^\alpha}.\]
\subsection{Weierstrass定理}
利用Cauchy不等式可以得到一个用$f$的模来控制它的导数$D^\alpha f$的模的不等式.先引进
\begin{definition}\label{def1.2.4}
	如果$\Omega$的子集$G$满足:
	
	(1)\hypertarget{1.2.4}{}
	$\bar{G}\subset\Omega$;
	
	(2)\hypertarget{1.2.4}{}
	$\bar{G}$是紧的,

就称\textbf{$G$相对于$\Omega$是紧的}\index{X!相对紧集},记为$G\subset\subset\Omega$\index[symbolindex]{\textbf{点集}!$G\subset\subset\Omega$}.
\end{definition}
我们有下面的
\begin{theorem}\label{thm1.2.5}
	设$\Omega$是$\mathbb{C}^n$中的域,$f\in H(\Omega)$.如果紧集$K$及其邻域$G$满足条件$K\subset G\subset\subset\Omega$,那么有不等式
	\begin{equation}\label{eq1.2.7}
		\sup_{z\in K}|(\DD^\alpha f)(z)|\le C_\alpha \sup_{z\in G}|f(z)|,
	\end{equation}
	这里$C_\alpha$是与$K,G$及$\alpha$有关的常数.
\end{theorem}
\begin{proof}
	因为$\rho=d(K,\partial G)>0$,所以以$K$中任何点$a$为中心,$\rho$为半径的多圆柱
	\[P=\{(z_1,\cdots,z_n)\colon|z_j-a_j|<\rho,\quad j=1,\cdots,n\}\]
	都含在$G$中.于是,由Cauchy不等式
	\[|(\DD^\alpha f)(a)|\le \sup_{\zeta\in\partial_0 P}|f(\zeta)|\frac{\alpha!}{\rho^{|\alpha|}}\le C_\alpha \sup_{z\in G}|f(z)|.\qedhere\]
\end{proof}
不等式\eqref{eq1.2.7}很简单,而且还有很多应用.下面关于函数列的Weierstrass定理和正规族理论都要用到它.
\begin{theorem}[(\textbf{Weierstrass})]\label{thm1.2.6}\index{D!定理!Weierstrass定理}
	设$\Omega$是$\mathbb{C}^n$中的域,$\{f_k\}$是$\Omega$上一列全纯函数.如果它在$\Omega$上内闭一致收敛于$f$,那么$f\in H(\Omega)$,而且$\{D^\alpha f_k\}$在$\Omega$上内闭一致收敛于$D^\alpha f$.
\end{theorem}
\begin{proof}
	对于任意的$a\in\Omega$,我们证明$f$在$a$点附近能展开成幂级数.适当选取$\rho=(\rho_1,\cdots,\rho_n)$,使得$\bar{P}(a,\rho)\subset\Omega$.对$f_k$用Cauchy积分公式
	\[f_k(z)=\frac1{(2\pi\ii)^n}\int\limits_{\partial_0 P}\frac{f_k(\zeta)}{\prod\limits_{j=1}^n(\zeta_j-z_j)}\mathrm{d}\zeta_1\cdots\mathrm{d}\zeta_n,\quad z\in P(a,\rho).\]
	由于$\lim\limits_{k\to\infty}f_k(\zeta)=f(
	\zeta)$在$\partial_0 P$上一致成立,在上式中命$k\to\infty$,即得
	\[f(z)=\frac1{(2\pi\ii)^n}\int\limits_{\partial_0 P}\frac{f(\zeta)}{\prod\limits_{j=1}^n(\zeta_j-z_j)}\mathrm{d}\zeta_1\cdots\mathrm{d}\zeta_n,\quad z\in P(a,\rho).\]
	由于
	\[\frac1{\prod\limits_{j=1}^{n}(\zeta_j-z_j)}=\sum_\alpha \prod_{j=1}^{n}\frac{(z_j-a_j)^{\alpha_j}}{(\zeta_j-a_j)^{\alpha_j+1}}\]
	对$\zeta\in\partial_0 P$一致地成立,因而有
	\begin{align*}
		f(z)
		&=\sum_\alpha\left\{\frac1{(2\pi\ii)^n}\int\limits_{\partial_0 P}\frac{f(\zeta)\mathrm{d}\zeta_1\cdots\mathrm{d}\zeta_n}{\prod\limits_{j=1}^{n}(\zeta_j-a_j)^{\alpha_j+1}}\right\}(z-a)^\alpha \\
		&=\sum_\alpha c_\alpha (z-a)^\alpha ,
	\end{align*}
所以$f\in H(\Omega).$

对于任意紧集$K\subset\Omega$,取其邻域$G$,使得$K\subset G\subset\subset\Omega.$因为$\bar{G}$是紧的,故对任意$\varepsilon>0,$存在$k_0,$当$k>k_0$时,有$\sup_{z\in \bar{G}}|f_k(z)-f(z)|<\varepsilon.$于是,由定理\ref{thm1.2.5}得
\[\sup_{z\in K}|\DD^\alpha(f_k-f)(z)|\le C_\alpha \sup_{z\in \bar{G}}|f_k(z)-f(z)|<C_\alpha\varepsilon.\]
这正好说明$\DD^\alpha f_k$在$\Omega$上内闭一致收敛于$\DD^\alpha f$.
\end{proof}
\subsection{Montel定理}
现在把单复变中的正规族理论推广到多复变.
\begin{definition}
	设$F=\{f\}$是域$\Omega\subset\mathbb{C}^n$上的一个全纯函数族.如果它的任意序列$\{f_k\}\subset F$中一定包含一个在$\Omega$上内闭一致收敛的子列,就称$F$是$\Omega$上的一个\textbf{正规族}\index{Z!正规族}.
\end{definition}
\begin{definition}
	设$F=\{f\}$是域$\Omega\subset\mathbb{C}^n$上的一个全纯函数族.如果对任意紧集$K\subset\Omega$,存在常数$M(K)$,使得$|f(z)|\le M(K)$对任意的$z\in K$及$f\in F$成立,就称$F$是\textbf{局部一致有界的}\index{J!局部一致有界}.
\end{definition}
类似于单复变情形,我们有下面的
\begin{theorem}[(\textbf{Montel})]\label{thm1.2.9}\index{D!定理!Montel定理}
	设$\Omega$是$\mathbb{C}^n$中的域,$\Omega$上的全纯函数族$F$是$\Omega$上的正规族的充分必要条件是$F$在$\Omega$上局部一致有界.
\end{theorem}
\begin{proof}
		必要性\quad 如果$F$是$\Omega$上的正规族,但它并不局部一致有界,则必存在紧集$K\subset\Omega$,使得
		\[\sup\{|f(z)|\colon z\in K,f\in F\}=\infty.\]
		因而存在$f_k\in F$,使得$\sup\{|f_k(z)|\colon z\in K\}\ge k$.由于$F$是正规族,故必存在$\{f_k\}$的子列$\{f_{k_\nu}\}$,它在$\Omega$上内闭一致收敛于$f$,因而存在$\nu_0$,当$\nu>\nu_0$时,在$K$上存在不等式
		\[|f_{k_\nu}-f(z)|<1.\]
		$f$在$K$上当然是有界的,不妨设$|f(z)|\le M(z\in K)$.于是,当$z\in K$及$\nu>\nu_0$时,便有
		\[|f_{k_\nu}(z)|\le|f_{k_\nu}(z)-f(z)|+|f(z)|\le M+1,\]
		这就导致$k_\nu \le M+1$的矛盾.
		
		充分性\quad 如果$F$在$\Omega$上局部一致有界,任取$\{f_k\}\subset F$,我们证明必能从$\{f_k\}$中取出一个子列,它在$\Omega$上内闭一致收敛.为此取紧集$K\subset G\subset\subset\Omega$,由定理\ref{eq1.2.5}得
		\[\sup_K \bigg|\pp{f_k}{z_j}\bigg|\le C_j \sup_G |f_k|\le M(K),\]
		即$\pp{f_k}{z_j}$局部一致有界,因而$\pp{f_k}{x_j},\pp{f_k}{y_j}$也局部一致有界.把$f_k$的实部、虚部分开,分别应用中值公式,便知$\{f_k\}$在$K$上是等度连续的.又因为它在$K$上是一致有界的,故由Arzel\`a-Ascoli定理\index{D!定理!Arzel\`a-Ascoli定理},能在$\{f_k\}$中取出在$\Omega$上内闭一致收敛的子列,取一列紧集$K_j$,使得
		\[K_1\subset K_2\subset\cdots\subset K_j\subset\cdots\to\Omega.\]
		根据上面的讨论,对每个$K_j$,均能从$\{f_k\}$中取出在其上一致收敛的子列,再用Cantor的对角线法,即可取出一个在每个$K_j$上都一致收敛的子列,这个子列在$\Omega$上是内闭一致收敛的.
\end{proof}
\subsection{Hurwitz定理}
作为这一节的结尾,我们把单复变中的Hurwitz定理推广到多复变.
\begin{theorem}[(\textbf{Hurwitz})]\label{thm1.2.10}\index{D!定理!Hurwitz定理}
设$\Omega$是$\mathbb{C}^n$中的域,$f_k$是$\Omega$上一列处处不为$0$的全纯函数.如果$f_k$在$\Omega$上内闭一致收敛于$f$,那么$f$在$\Omega$中或者恒等于$0$,或者处处不等于$0$.
\end{theorem}
\begin{proof}
	设$f\not\equiv0$.任取$a\in\Omega$,我们证明$f(a)\neq0$.取多圆柱$P(a,r)\subset\Omega$,再取数$\alpha_1,\cdots,\alpha_n$和$\lambda$,使得$|\alpha_j|<r_j,|\lambda|<1$.于是$(a_1+\alpha_1\lambda,\cdots,a_n+\alpha_n\lambda)\in P(a,r)\subset\Omega$.在满足上述条件的$\alpha_j$中选取一组,使得
	\[\psi(\lambda)=f(a_1+\alpha_1\lambda,\cdots,a_n+\alpha_n\lambda)\]
	在$|\lambda|<1$中不恒等于$0$,这一定能办到,否则$f$在$P(a,r)$中恒等于$0$.命
	\[\psi_k(\lambda)=f_k(a_1+\alpha_1\lambda,\cdots,a_n+\alpha_n\lambda).\]
	于是,$\psi_k$在$|\lambda|<1$中内闭一致收敛于$\psi$,且$\psi$不恒等于$0$.由单复变的Hurwitz定理,$\psi$在$|\lambda|<1$中处处不等于$0$.特别$\psi(0)\neq0$,即$f(a)\neq0$.
\end{proof}

\section{Hartogs现象\label{sec1.3}}
\subsection{Hartogs现象}
设$D$是复平面$\mathbb{C}$上的域,$f$是$D$上的全纯函数.如果存在域$G\supset D$和$G$上的全纯函数$F$,使得当$z\in D$时,$F(z)=f(z)$,我们就说$F$是$f$在域$G$上的\textbf{全纯开拓}\index{Q!全纯开拓},或者说$f$能全纯开拓到更大的域$G$上去.

我们问,是否存在这样的域$D\subset\mathbb{C}$,在其上的每一个全纯函数都能全纯开拓到比$D$更大的域上去?答案是否定的,因为对$\partial D$上任意一点$a$,$f(z)=\frac1{z-a}$是$D$上的全纯函数,但它不能越过$a$开拓出去.

但是在多复变中出现了一种奇怪的现象,确实存在这样的域$\Omega\subset\mathbb{C}^n,n\ge1,H(\Omega)$中所有的函数都能全纯开拓到比$\Omega$更大的域上去.Hartogs首先发现了这种现象,故名之为Hartogs现象\index{H!Hartogs现象}.下面就是这种现象的一个简单的例子.

设$0<\alpha,\beta<1,$记
\[G_1=\{(z,w)\in\mathbb{C}^2\colon |z|<1,\beta<|w|<1\},\]
\[G_2=\{(z,w)\in\mathbb{C}^2\colon |z|<\alpha,|w|<1\},\]
\[G=G_1\cup G_2.\]
我们证明$H(G)$中每一个函数都能全纯开拓到比$G$更大的单位双圆柱域:
\[U^2=\{(z,w)\colon |z|<1,|w|<1\}.\]
任取$f\in H(G)$,它当然也是$G_1$上的全纯函数,固定$z(|z|<1)$,则$w\mapsto f(z,w)$是圆环$\beta<|w|<1$中的全纯函数,因而有Laurent展开式
\begin{equation}\label{eq1.3.1}
	f(z,w)=\sum_{k=-\infty}^\infty a_k(z)w^k,\quad \beta<|w|<1,
\end{equation}
这里
\[a_k(z)=\frac1{2\pi\ii}\int\limits_{|w|=\rho}f(z,w)w^{-k-1}\mathrm{d}w,\quad \beta<\rho<1,|z|<1.\]
\begin{minipage}{0.6\textwidth}
所以$a_k(z)$在$|z|<1$中全纯.另一方面,$f$是$G_2$中的全纯函数,当固定$z(|z|<\alpha)$时,$w\mapsto f(z,w)$是单位圆盘$|w|<1$中的全纯函数,因而有Taylor展开式
\begin{equation}\label{eq1.3.2}
	f(z,w)=\sum_{k=0}^\infty b_k(z)w^k,\quad |z|<\alpha,|w|<1,
\end{equation}
由于\eqref{eq1.3.1},\eqref{eq1.3.2}两级数在$G_1\cap G_2$中相等,所以
\end{minipage}
\noindent\begin{minipage}{0.4\textwidth}
	\centering
	\tikzset{global scale/.style={
			scale=#1,
			every node/.append style={scale=#1}
		}
	}
	\begin{tikzpicture}[>=Stealth,thick,global scale=0.7]
		\draw[->](0,0)node[below]{$O$}--(2,0)node[below]{$\alpha$}--(4,0)node[below]{$1$}--(5,0)node[below]{$|z|$};
		\draw[->](0,0)--(0,2.5)node[left]{$\beta$}--(0,4)node[left]{$1$}--(0,5)node[left]{$|w|$};
		\draw[-](2,0)--(2,2.5)--(4,2.5)--(4,4);
		\draw[-](0,4)--(4,4);
		\draw[densely dashed](2,4)--(2,2.5);
		\draw[densely dashed](0,2.5)--(2,2.5);
		\draw[densely dashed](4,2.5)--(4,0);
		\draw[domain=0:0.4]plot(\x,-\x+0.4);
		\draw[domain=0:0.8]plot(\x,-\x+0.8);
		\draw[domain=0:1.2]plot(\x,-\x+1.2);
		\draw[domain=0:1.6]plot(\x,-\x+1.6);
		\draw[domain=0:2.0]plot(\x,-\x+2.0);
		\draw[domain=0:2.0]plot(\x,-\x+2.4);
		\draw[domain=0:2.0]plot(\x,-\x+2.8);
		\draw[domain=0:2.0]plot(\x,-\x+3.2);
		\draw[domain=0:2.0]plot(\x,-\x+3.6);
		\draw[domain=0:2.0]plot(\x,-\x+4.0);
		\draw[domain=0.4:2.0]plot(\x,-\x+4.4);
		\draw[domain=0.8:2.0]plot(\x,-\x+4.8);
		\draw[domain=1.2:2.0]plot(\x,-\x+5.2);
		\draw[domain=1.6:2.0]plot(\x,-\x+5.6);
		\draw[domain=0:0.4]plot(\x,\x+3.6);
		\draw[domain=0:0.8]plot(\x,\x+3.2);
		\draw[domain=0:1.2]plot(\x,\x+2.8);
		\draw[domain=0.1:1.6]plot(\x,\x+2.4);
		\draw[domain=0.5:2]plot(\x,\x+2);
		\draw[domain=0.9:2.4]plot(\x,\x+1.6);
		\draw[domain=1.3:2.8]plot(\x,\x+1.2);
		\draw[domain=1.7:3.2]plot(\x,\x+0.8);
		\draw[domain=2.1:3.6]plot(\x,\x+0.4);
		\draw[domain=2.5:4]plot(\x,\x);
		\draw[domain=2.9:4]plot(\x,\x-0.4);
		\draw[domain=3.3:4]plot(\x,\x-0.8);
		\draw[domain=3.7:4]plot(\x,\x-1.2);
		\node[fill=white!,minimum height=3mm,minimum width=3mm] at (3,3.2) {$G_1$};
		\node[fill=white!,minimum height=3mm,minimum width=3mm]  at(1,1.5) {$G_2$};
		\node[fill=white!,minimum height=3mm,minimum width=3mm] at (1,3) {$G_1\cap G_2$};
	\end{tikzpicture}
	\captionof{figure}{\label{fig1.1}}
\end{minipage}
\[a_k(z)=b_k(z),k=0,\pm1,\pm2,\cdots\]
因而当$|z|<\alpha$时,$a_k(z)=0,k=-1,-2,\cdots,$但因$a_k(z)$在$|z|<1$中全纯,故由唯一性定理得
\[a_k(z)=0,k=-1,-2,\cdots,|z|<1.\]
这样$f(z,w)=\sum\limits_{k=0}^\infty a_k(z)w^k$在$G$中成立.设$a_k(z)$在$|z|<1$中的Taylor展开式为$a_k(z)=\sum\limits_{j=0}^\infty c_{jk}z^j$,故有
\begin{equation}\label{eq1.3.3}
	f(z,w)=\sum_{j,k=0}^{\infty}c_{jk}z^j w^k
\end{equation}
在$G$中成立.因为\eqref{eq1.3.3}右端的二重级数当$|z|<1,\beta<|w|<1$时收敛,故由命题\ref{prop1.1.1}知道,它在$|z|<1,|w|<1$中收敛.今定义
\[F(z,w)=\sum_{j,k=0}^\infty c_{jk}z^jw^k,\]
它是$U^2$中的全纯函数,且在$G$中$F=f$,因而它是$f$在$U^2$中的全纯开拓.
\subsection{全纯函数在Reinhardt域上的展开式}
为了进一步说明Hartogs现象,我们给出全纯函数在Reinhardt域上的展开式.
\begin{definition}
	设$\Omega$是$\mathbb{C}^n$中的域.如果对任意$(z_1,\cdots,z_n)\in\Omega$及$\theta_1,\cdots,\theta_n\in\mathbb{R}$($\mathbb{R}$表示实数域)必有$(\ee^{\ii\theta_1}z_1,\cdots,\ee^{\ii\theta_n}z_n)\in\Omega$,就称$\Omega$为\textbf{Reinhardt域}\index{Y!域!Reinhardt域}.
\end{definition}
例如多圆盘、球以及上例提到的域$G$都是Reinhardt域.
\begin{theorem}\label{thm1.3.2}
	设$\Omega$是$\mathbb{C}^n$中的Reinhardt域,则$\Omega$上任意全纯函数$f$必有如下的Laurent展开式\index{L!Laurent展开式}
	\begin{equation}\label{eq1.3.4}
		f(z)=\sum_{\alpha\in\mathbb{Z}^n} a_\alpha z^\alpha ,
	\end{equation}
这里$\mathbb{Z}^n=\{(\alpha_1,\cdots,\alpha_n)\colon\alpha_j\in\mathbb{Z},j=1,\cdots,n\}$($\mathbb{Z}$表示全体整数所成的集合\index[symbolindex]{\textbf{点集}!$\mathbb{Z}^n$}),该级数在$\Omega$中内闭一致收敛,且$a_\alpha$由$f$所唯一确定.
\end{theorem}
\begin{proof}
	我们先证,如果\eqref{eq1.3.4}在$\Omega$上内闭一致收敛,那么$a_\alpha$由$f$所唯一确定.事实上,取$w\in\Omega$,要求$w$的每个坐标$w_j\neq0$.对于这个固定的$w$,取$z=(\ee^{\ii\theta_1}w_1,\cdots,\ee^{\ii\theta_n}w_n)$,则$z\in\Omega$.于是
	\[f(\ee^{\ii\theta_1}w_1,\cdots,\ee^{\ii\theta_n}w_n)=\sum_{\alpha\in\mathbb{Z}^n}a_\alpha w_1^{\alpha_1}\cdots w_n^{\alpha_n}\ee^{\ii(\theta_1\alpha_1+\cdots\theta_n\alpha_n)},\]
	两端乘$\ee^{-\ii(\theta_1\alpha_1+\cdots\theta_n\alpha_n)}$,并分别对$\theta_1,\cdots,\theta_n$在$(0,2\pi)$上积分得
	\[a_\alpha=\frac1{(2\pi)^n}\frac1{w^\alpha}\int_{0}^{2\pi}\cdots\int_{0}^{2\pi} f(w_1\ee^{\ii\theta_1},\cdots,w_n\ee^{\ii\theta_n})\ee^{-\ii(\theta_1\alpha_1+\cdots+\theta_n\alpha_n)}\mathrm{d}\theta_1\cdots\mathrm{d}\theta_n ,\]
	故$a_\alpha$由$f$所唯一确定.
	
	现在证明\eqref{eq1.3.4}成立.取定$w\in\Omega$,由于$\Omega$是Reinhardt域,故可取$\varepsilon$充分小,使得
	\[G(w,\varepsilon)=\{z\in\mathbb{C}^n\colon |w_j|-\varepsilon<|z_j|<|w_j|+\varepsilon,j=1,\cdots,n\}\]
	含在$\Omega$中.这是因为对于任意的$z\in G(w,\varepsilon)$,有$||z_j|-|w_j||<\varepsilon,j=1,\cdots,n,$命$z_j'=\ee^{\ii\theta_j}z_j$,当然有
	\begin{equation}\label{eq1.3.5}
		||z_j'|-|w_j||<\varepsilon,\quad j=1,\cdots,n.
	\end{equation}
适当选择$\theta_j$,可使$\arg z_j'=\arg w_j$,于是\eqref{eq1.3.5}变成$|z_j'-w_j|<\varepsilon,j=1,\cdots,n$.因为$\Omega$是域,取$\varepsilon$充分小,可使$(z_1',\cdots,z_n')\in\Omega$,因为$\Omega$是Reinhardt域,故$z\in\Omega$.注意$G(w,\varepsilon)$是$n$个圆环的拓扑积,对$f$的每个变量分别用单复变中的Laurent定理,可得
\[f(z)=\sum_{\alpha\in\mathbb{Z}^n}a_\alpha(w)z^\alpha,\quad z\in G(w,\varepsilon),\]
它在$w$的邻域中一致收敛.不难证明,$a_\alpha(w)$实际上和$w$无关.为此,取$w'\in G(w,\varepsilon)$,同样有
\[f(z)=\sum_{\alpha\in\mathbb{Z}^n} a_\alpha(w')z^\alpha,\quad z\in G(w',\varepsilon'),\]
由上面证明的$a_\alpha$的唯一性知道$a_\alpha(w)=a_\alpha(w'),\alpha\in\mathbb{Z}^n$.这就证明了$a_\alpha(w)$在一个局部范围内是一个常数,利用$\Omega$的连通性便知$a_\alpha(w)=a_\alpha$在$\Omega$上成立.因而
\[f(z)=\sum_{\alpha\in\mathbb{Z}^n} a_\alpha z^\alpha\]
在$\Omega$中每一点的邻域中一致地成立,因而在$\Omega$中内闭一致地成立.
\end{proof}
从这个定理可得下面很有用的定理.
\begin{theorem}\label{thm1.3.3}
	设$\Omega$是$\mathbb{C}^n$中的Reinhardt域.如果对每个$j(j=1,\cdots,n)$,$\Omega$中都有第$j$个坐标为$0$的点,那么每个$f\in H(\Omega)$都有幂级数展开式:
	\begin{equation}\label{eq1.3.6}
		f(z)=\sum_\alpha a_\alpha z^\alpha,
	\end{equation}
它在$\Omega$上内闭一致地成立.
\end{theorem}
\begin{proof}
	根据定理\ref{thm1.3.2},$f$有Laurent展开式
	\begin{equation}\label{eq1.3.7}
		f(z)=\sum_{\alpha\in\mathbb{Z}^n} a_\alpha z^\alpha .
	\end{equation}
如果存在这样的$\alpha=(\alpha_1,\cdots,\alpha_n)$,其中某个$\alpha_k$是负整数,而对应的$a_\alpha\neq0$,在$\Omega$中取第$k$个坐标为$0$的点$z^{(k)}$,则\eqref{eq1.3.7}对$z^{(k)}$不能成立,这不可能.因而只有$a_\alpha=0$.于是\eqref{eq1.3.7}就是幂级数\eqref{eq1.3.6}.
\end{proof}
因为单位球$B_n$满足定理\ref{thm1.3.3}的条件,故有如下的
\begin{corollary}\label{cor1.3.4}
	每个$f\in H(B_n)$都有幂级数展开式
	\[f(z)=\sum_\alpha a_\alpha z^\alpha,z\in B_n.\]
	上面这个展开式也可写为
	\[f(z)=\sum_{k=0}^{\infty} \sum_{|\alpha|=k}a_\alpha z^\alpha =\sum_{k=0}^{\infty}P_k(z),\]
	这里$P_k(z)=\sum\limits_{|\alpha|=k}a_\alpha z^\alpha$是$z_1,\cdots,z_n$的$k$次齐次多项式,称它为$f$在$B_n$中的\textbf{齐次展开式}\index{Q!齐次展开式}.
\end{corollary}
从定理\ref{thm1.3.3}还可得到下面的全纯开拓定理.
\begin{theorem}\label{thm1.3.5}
	设$\Omega$是$\mathbb{C}^n$中的Reinhardt域.如果对每个$j(j=1,\cdots,n),\Omega$中都有第$j$个坐标为$0$的点,那么每个$f\in H(\Omega)$都能全纯开拓到域
	\[\Omega'=\{w=(\rho_1 z_1,\cdots,\rho_n z_n)\colon (z_1,\cdots,z_n)\in\Omega,\quad 0\le \rho_j\le 1,j=1,\cdots,n\}.\]
\end{theorem}
\begin{proof}
	根据定理\ref{thm1.3.3},$f$在$\Omega$中有幂级数展开式
	\[f(z)=\sum_\alpha a_\alpha z^\alpha,\quad z\in\Omega.\]
	任取$w\in\Omega'$,按定义,存在$z\in\Omega$及$0\le\rho_j\le1,j=1,\cdots,n,$使得$w_j=\rho_j z_j,j=1,\cdots,n,$因而$|w_j|\le|z_j|$.由于$\sum\limits_\alpha a_\alpha z^\alpha$收敛,由命题\ref{prop1.1.1},$\sum\limits_\alpha a_\alpha w^\alpha$收敛,且在$\Omega'$中内闭一致收敛.现在定义
	\[F(w)=\sum_\alpha a_\alpha w^\alpha ,\quad w\in\Omega' ,\]
	则$F\in H(\Omega')$,且$F|_\Omega =f$.所以$F$是$f$在$\Omega'$上的全纯开拓.
\end{proof}
\begin{corollary}\label{cor1.3.6}
	设$0<r<R$,
	\[\Omega=\{(z_1,\cdots,z_n)\colon r^2<|z_1|^2+\cdots|z_n|^2<R^2\},\]
	每个$f\in H(\Omega)$必能全纯开拓到球$B(0,R)$:
	\[|z_1|^2+\cdots|z_n|^2<R^2.\]
\end{corollary}
\begin{proof}
	利用定理\ref{thm1.3.5}即得.
\end{proof}
推论\ref{cor1.3.6}又给出了一类发生Hartogs现象的域.由它还可得到一个和单复变有本质不同的事实:在单复变中,全纯函数的零点一定是孤立的,可在多复变中恰好相反.
\begin{theorem}\label{thm1.3.7}
	设$\Omega$是$\mathbb{C}^n(n>1)$中的域,$f\in H(\Omega)$,那么$f$在$\Omega$上的零点一定不是孤立的.
\end{theorem}
\begin{proof}
	如果$a\in\Omega$是$f$的一个孤立零点,这意味着存在以$a$为中心,充分小的正数$\varepsilon$为半径的球$B(a,\varepsilon)\subset\Omega$,$f$在$B(a,\varepsilon)$中除了$a$以外不再有其它的零点.命
	\[g(z)=\frac1{f(z)},\]
	则$g$在$B(a,\varepsilon)\setminus\bar{B}\left(a,\frac{\varepsilon}{2}\right)$中全纯,由推论\ref{cor1.3.6},$g$必在$B(a,\varepsilon)$中全纯,因而$f(a)\neq 0$,这是一个矛盾.
\end{proof}
\subsection{全纯域}
推论\ref{cor1.3.6}只是一个更一般的定理的特殊情形,在第\ref{chap4}章中,我们将证明下面的结果(见定理\ref{thm4.7.5}):

设$n>1$,$\Omega$是$\mathbb{C}^n$中的域,$K$是$\Omega$中的紧集.如果$\Omega\setminus K$是连通的,那么每一个$\Omega\setminus K$上的全纯函数都可全纯开拓\index{Q!全纯开拓}到$\Omega$上去.

由此看来,$\mathbb{C}^n$中有相当一部分域都会发生Hartogs现象,那么是不是$\mathbb{C}^n(n>1)$中每个域都有Hartogs现象呢?当然不是.
\begin{definition}\label{def1.3.8}
	设$\Omega$是$\mathbb{C}^n$中的域.如果不存在比$\Omega$更大的域$\Omega'(\Omega'\supset\Omega,\Omega'\neq\Omega)$,使得$H(\Omega)$中每个函数都能全纯开拓到$\Omega'$上去,就称$\Omega$是\textbf{全纯域}\index{Y!域!全纯域}.
\end{definition}
$n=1$时,$\mathbb{C}$中所有的域都是全纯域.$n>1$时,什么样的域是全纯域?
\begin{definition}\label{def1.3.9}
	设$\Omega$是$\mathbb{C}^n$中的域.如果对任意点$\zeta\in\partial\Omega$,都存在一个通过$\zeta$的实超平面$Q$与$\overline{\Omega}$仅在$\zeta$处相交,即$Q\cap\overline{\Omega}=\{\zeta\}$,就称$\Omega$是\textbf{欧氏凸域}\index{Y!域!欧氏凸域}.
\end{definition}
我们有下面的
\begin{theorem}\label{thm1.3.10}
	$\mathbb{C}^n$中的欧氏凸域一定是全纯域.
\end{theorem}
\begin{proof}
	设$\Omega$是$\mathbb{C}^n$中的欧氏凸域,故对$\partial\Omega$上每一点$\zeta$,存在一个过$\zeta$的超平面$Q$,它与$\overline{\Omega}$只在$\zeta$处相交.不妨设$\zeta=0$,则过$\zeta$的超平面可写为
	\begin{equation}\label{eq1.3.8}
		a_1x_1+b_1y_1+\cdots a_nx_n+b_ny_n=0.
	\end{equation}
记$c_j=a_j-\ii b_j,z_j=x_j+\ii y_j,$则\eqref{eq1.3.8}可写为$\Re(\sum\limits_{j=1}^n c_jz_j)=0.$命$g(z)=\sum\limits_{j=1}^n c_j z_j,$易知$g$在$\Omega$中没有零点,因若有$w\in\Omega,$使得$g(w)=0$,则$\Re g(w)=0$,这等于说\eqref{eq1.3.8}通过$w$点,这与$Q$的取法矛盾.因而
\[f(z)=\frac1{g(z)}\]
是$\Omega$中的全纯函数,但它不可能通过边界点$\zeta$全纯开拓出去.
\end{proof}
显然,这个定理的逆是不成立的,即全纯域不一定是欧氏凸域,这在$n=1$时就有大量的反例.因此,全纯域是比欧氏凸域更为广泛的一类域.如何给出全纯域的特征是多复变理论中一个重要的基本问题,我们将在第\ref{chap5}、\ref{chap6}两章中讨论这个问题.
\section{球和球面上的积分\label{sec1.4}}
\subsection{积分的极坐标公式}
设$m_{2n}$\index[symbolindex]{\textbf{测度}!$m_{2n}$}是$\mathbb{R}^{2n}=\mathbb{C}^n$上的Lebesgue测度,已知
\[m_{2n}(B_n)=\frac{\pi^n}{n!}.\]
今在$\mathbb{C}^n$中引入测度$\nu_n$\index{C!测度$\nu_n$}\index[symbolindex]{\textbf{测度}!$\nu_{n}$},使得$\nu_n(B_n)=1$,它和$m_{2n}$的关系为
\begin{equation}\label{eq1.4.1}
	m_{2n}=\frac{\pi^n}{n!}\nu_n.
\end{equation}

$\mathbb{C}^n$中单位球面$\partial B_n$上的Lebesgue测度记为$\lambda_{2n-1}$\index[symbolindex]{\textbf{测度}!$\lambda_{2n-1}$},已知$\lambda_{2n-1}(\partial B_n)=\frac{2\pi^n}{(n-1)!}$.在$\partial B_n$上引进测度$\sigma_n$\index{C!测度$\sigma_n$}\index[symbolindex]{\textbf{测度}!$\sigma_{n}$},使得$\sigma_n(\partial B_n)=1$,它和$\lambda_{2n-1}$的关系为
\begin{equation}\label{eq1.4.2}
	\lambda_{2n-1}=\frac{2\pi^n}{(n-1)!}\sigma_n.
\end{equation}


在不致引起混乱的情况下,我们将把$\partial B_n$、$B_n$、$\sigma_n$和$\nu_n$简记为$\partial B$、$B$、$\sigma$和$\nu$.

下面的极坐标积分公式\index{J!极坐标积分公式}在今后的讨论中常要用到.
\begin{theorem}\label{thm1.4.1}
	如果$f$使下式中的积分有意义,那么
	\begin{equation}\label{eq1.4.3}
		\int_{\mathbb{C}^n}f\mathrm{d}\nu_n=2n\int_{0}^{\infty}r^{2n-1}\mathrm{d}r\int_{\partial B}f(r\zeta)\mathrm{d}\sigma_n(\zeta).
	\end{equation}
\end{theorem}
\begin{proof}
	记$z_j=x_j+\ii y_j,j=1,\cdots,n.$引进极坐标:
	\begin{align*}
		x_1
		&=r\cos\theta_1,y_1=r\sin\theta_1\cos\theta_2,x_2=r\sin\theta_1\sin\theta_2\cos\theta_3,\cdots,\\
		x_n
		&=r\sin\theta_1\sin\theta_2\cdots\sin\theta_{2n-2} \cos\theta_{2n-1},\\
		y_n
		&=r\sin\theta_1\cdots\sin\theta_{2n-2}\sin\theta_{2n-1},
	\end{align*}
其中$0\le\theta_1,\cdots,\theta_{2n-2}<\pi,0\le\theta_{2n-1}<2\pi$,则有
\[\mathrm{d} m_{2n}(z)=r^{2n-1}\sin^{2n-2}\theta_1\sin^{2n-3}\theta_2\cdots\sin\theta_{2n-2}\mathrm{d}r\mathrm{d}\theta_1\mathrm{d}\theta_2\cdots\mathrm{d}\theta_{2n-1},\]
\[\mathrm{d}\lambda_{2n-1}(\zeta)=\sin^{2n-2}\theta_1\sin^{2n-3}\theta_2\cdots\sin\theta_{2n-2}\mathrm{d}\theta_1\mathrm{d}\theta_2\cdots\mathrm{d}\theta_{2n-1}.\]
其中$z=r\zeta,\zeta\in\partial B_n$,因而得$\mathrm{d}m_{2n}(z)=r^{2n-1}\mathrm{d}r\mathrm{d}\lambda_{2n-1}(\zeta).$于是
\begin{align*}
	\int_{\mathbb{C}^n}f\mathrm{d}\nu_n
	&=\frac{n!}{\pi^n}\int_{\mathbb{C}^n}f\mathrm{d} m_{2n} \\
	&=\frac{n!}{\pi^n}\int_{0}^{\infty}r^{2n-1}\mathrm{d}r\int_{\partial B}f(r\zeta)\mathrm{d}\lambda_{2n-1}(\zeta)\\
	&=2n\int_{0}^{\infty}r^{2n-1}\mathrm{d}r\int_{\partial B} f(r\zeta)\mathrm{d}\sigma_n(\zeta).\qedhere
\end{align*}
\end{proof}
用$\rho B_n$记$\mathbb{C}^n$中以原点为中心,$\rho$为半径的球.如果积分区域是$\rho B_n$,那么用公式\eqref{eq1.4.3}可得
\begin{equation}\label{eq1.4.4}
	\int_{\rho B_n}f\mathrm{d}\nu_n=2n\int_{0}^{\rho} r^{2n-1}\mathrm{d}r\int_{\partial B_n}f(r\zeta)\mathrm{d}\sigma_n(\zeta).
\end{equation}
\subsection{球面上积分的几种表示}
设$f$是定义在$\bar{B}_n$上的函数,如果它只依赖于$z_1,\cdots,z_k$\\$k$个变数,$1\le k< n$,那么利用定理\ref{thm1.4.1}可把$\int_{\partial B_n}f\mathrm{d}\sigma_n$化为低维球$B_k$上的积分.
\begin{theorem}\label{thm1.4.2}
	设$f$只依赖于$z_1,\cdots,z_k,1\le k<n$,那么
	\begin{equation}\label{eq1.4.5}
		\int_{\partial B_n}f\mathrm{d}\sigma_n=\binom{n-1}{k}\int_{B_k}f(w)(1-|w|^2)^{n-k-1}\mathrm{d}\nu_k(w).
	\end{equation}
\end{theorem}
\begin{proof}
	记$I(r)=\int_{rB_n}f\mathrm{d}\nu_n,0<r<\infty,$由\eqref{eq1.4.4}得
	\[I(r)=2n\int_{0}^{r} t^{2n-1}\mathrm{d}t\int_{\partial B_n} f(t\zeta)\mathrm{d}\sigma_n(\zeta).\]
	对$r$求导数,并命$r=1$得
	\begin{equation}\label{eq1.4.6}
		I'(1)=2n\int_{\partial B_n} f(\zeta)\mathrm{d}\sigma_n(\zeta).
	\end{equation}
另一方面,由于$\rho B_{n-k}$的体积为$\frac{\pi^{n-k}}{(n-k)!}\rho^{2(n-k)}$,所以
\begin{align*}
	I(r)
	&=\frac{n!}{\pi^n}\int_{rB_n} f\mathrm{d} m_{2n}\\
	&=\frac{n!}{\pi^n}\int_{rB_k} f(w)\frac{\pi^{n-k}}{(n-k)!}(r^2-|w|^2)^{n-k}\mathrm{d}m_{2n}(w)\\
	&=\binom{n}{k}\int_{rB_k} f(w)(r^2-|w|^2)^{n-k}\mathrm{d}\nu_k(w)\\
	&=2k\binom{n}{k}\int_{0}^{r}t^{2k-1}\mathrm{d}t\int_{\partial B_k}f(t\zeta)(r^2-t^2)^{n-k}\mathrm{d}\sigma_k(\zeta).
\end{align*}
由此得
\begin{align}\label{eq1.4.7}
		I'(1)
	&=4k(n-k)\binom{n}{k}\int_{0}^{1}t^{2k-1}\mathrm{d}t\int_{\partial B_k} (1-t^2)^{n-k-1}f(t\zeta)\mathrm{d}\sigma_k(\zeta)\notag\\
	&=2(n-k)\binom{n}{k}\int_{B_k} f(w)(1-|w|^2)^{n-k-1}\mathrm{d}\nu_k(w).
\end{align}
比较\eqref{eq1.4.6}和\eqref{eq1.4.7}即得\eqref{eq1.4.5}.
\end{proof}
在定理\ref{thm1.4.2}中取$k=n-1$,可得
\begin{corollary}\label{cor1.4.3}
	如果$f$中不含$z_n$,那么
	\[\int_{\partial B_n}f\mathrm{d}\sigma=\int_{B_{n-1}}f\mathrm{d}\nu_{n-1}.\]
\end{corollary}
\begin{theorem}\label{thm1.4.4}
	下面两个恒等式成立:
	
		(1) \hypertarget{1.4.4}{}
		$\int_{\partial B_n}f\mathrm{d}\sigma_n=\int_{\partial B_n}\mathrm{d}\sigma_n(\zeta)\frac1{2\pi}\int_{-\pi}^{\pi}f(\ee^{\ii\theta}\zeta)\mathrm{d}\theta$;
		
		(2) \hypertarget{1.4.4}{}
		$\int_{\partial B_n}f\mathrm{d}\sigma_n=\int_{B_{n-1}}\mathrm{d}\nu_{n-1}(\zeta')\frac1{2\pi}\int_{-\pi}^{\pi} f(\zeta',\ee^{\ii\theta}\zeta_n)\mathrm{d}\theta$,

        这里$\zeta'=(\zeta_1,\cdots,\zeta_{n-1})$.
\end{theorem}
\begin{proof}
	由Fubini定理,
	\[\int_{\partial B_n}\mathrm{d}\sigma_n(\zeta)\frac1{2\pi}\int_{-\pi}^{\pi} f(\ee^{\ii\theta}\zeta)\mathrm{d}\theta=\frac1{2\pi}\int_{-\pi}^{\pi}\mathrm{d}\theta\int_{\partial B_n}f(\ee^{\ii\theta}\zeta)\mathrm{d}\sigma_n(\zeta),\]
	命$\eta=\ee^{\ii\theta}\zeta$,由$\sigma_n$的旋转不变性得$\mathrm{d}\sigma_n(\eta)=\mathrm{d}\sigma_n(\zeta).$所以上式右端为
	\[\frac1{2\pi}\int_{-\pi}^{\pi}\mathrm{d}\theta\int_{\partial B_n}f(\eta)\mathrm{d}\sigma_n(\eta)=\int_{\partial B_n}f(\eta)\mathrm{d}\sigma_n(\eta).\]
	由此即得\hyperlink{1.4.4}{(1)}.为了证明\hyperlink{1.4.4}{(2)},先用上面的证明方法可得
	\[\int_{\partial B_n}f\mathrm{d}\sigma_n=\int_{\partial B_n}\mathrm{d}\sigma_n(\zeta)\frac1{2\pi}\int_{-\pi}^{\pi} f(\zeta',\ee^{\ii\theta}\zeta_n)\mathrm{d}\theta.\]
	当$(\zeta',\zeta_n)\in\partial B_n$时,$|\zeta_n|^2=1-|\zeta'|^2$,所以上面的内积分实际上只依赖于$\zeta'$,利用推论\ref{cor1.4.3}即得\hyperlink{1.4.4}{(2)}.
\end{proof}
\begin{theorem}\label{thm1.4.5}
	设$\alpha=(\alpha_1,\cdots,\alpha_n),\beta=(\beta_1,\cdots,\beta_n)$都是多重指标,$\alpha_j\ge0,\beta_j\ge0,j=1,\cdots,n$.

		(1) \hypertarget{1.4.5}{} 如果$\alpha\neq\beta$,那么$\int_{\partial B}\zeta^\alpha\bar{\zeta}^\beta\mathrm{d}\sigma(\zeta)=0$;
		
		(2) \hypertarget{1.4.5}{}
		$\int_{\partial B}|\zeta^\alpha|^2\mathrm{d}\sigma(\zeta)=\frac{(n-1)!\alpha!}{(n-1+|\alpha|)!}.$

\end{theorem}
\begin{proof}
	\hyperlink{1.4.5}{(1)}
	不妨设$\alpha_n\neq\beta_n$.记$f(\zeta)=\zeta^\alpha \bar{\zeta}^\beta$,利用定理\ref{thm1.4.4}\hyperlink{1.4.4}{(2)},这时内积分
	\[\frac1{2\pi} \int_{-\pi}^{\pi} f(\zeta',\ee^{\ii\theta}\zeta_n)\mathrm{d}\theta=\zeta^\alpha\bar{\zeta}^\beta \frac1{2\pi}\int_{-\pi}^{\pi} \ee^{\ii(\alpha_n-\beta_n)\theta}\mathrm{d}\theta=0,\]
	因而\hyperlink{1.4.5}{(1)}成立.
	
	\hyperlink{1.4.5}{(2)}
	命$I=\int_{\mathbb{C}^n} |z^\alpha|^2 \ee^{-|z|^2}\mathrm{d}m_{2n}(z).$实际上,被积函数是$\prod\limits_{j=1}^{n} |z_j^{\alpha_j}|^2\cdot \ee^{-|z_j|^2}$,把上面的积分按变量分开可得
	\begin{equation}\label{eq1.4.8}
		I=\prod_{j=1}^{n} \int_{\mathbb{C}}|\lambda|^{2\alpha_j} \ee^{-|\lambda|^2} \mathrm{d}m_2(\lambda)=\pi^n \alpha!.
	\end{equation}
另一方面,应用定理\ref{thm1.4.1},
\begin{align}\label{eq1.4.9}
	I
	&=\frac{\pi^n}{n!}\int_{\mathbb{C}^n}|z^\alpha|^2\ee^{-|z|^2}\mathrm{d}\nu_n(z)\notag\\
	&=\frac{2\pi^n}{(n-1)!}\int_{0}^{\infty} r^{2n-1}\mathrm{d}r \int_{\partial B_n} |(r\zeta)^\alpha|^2 \ee^{-r^2} \mathrm{d}\sigma_n(\zeta)\notag\\
	&=\frac{2\pi^n}{(n-1)!}\int_{0}^{\infty} r^{2|\alpha|+2n-1}\ee^{-r^2}\mathrm{d}r\int_{\partial B_n} |\zeta^\alpha|^2\mathrm{d}\sigma_n(\zeta)\notag\\
	&=\frac{\pi^n}{(n-1)!}(n-1+|\alpha|)!\int_{\partial B_n} |\zeta^\alpha|^2 \mathrm{d}\sigma_n(\zeta).
\end{align}
比较\eqref{eq1.4.8},\eqref{eq1.4.9},即得\hyperlink{1.4.5}{(2)}.
\end{proof}
\begin{theorem}\label{thm1.4.6}
	设$\alpha,\beta$都是多重指标.

		(1) \hypertarget{1.4.6}{}
		如果$\alpha\neq\beta$,那么$\int_B z^\alpha\bar{z}^\beta\mathrm{d}\nu(z)=0$;
		
		(2) \hypertarget{1.4.6}{}
		$\int_B|z^\alpha|^2 \mathrm{d}\nu(z)=\frac{n!\alpha!}{(n+|\alpha|)!}$.

\end{theorem}
\begin{proof}
	利用极坐标积分公式,把球内的积分化成球面的积分,然后利用定理\ref{thm1.4.5}即得所要证的公式.
\end{proof}
\section{次调和函数和Hartogs定理\label{sec1.5}}
在这一节里,我们首先介绍次调和函数的概念,然后利用次调和函数的理论来证明\ref{sec1.1}中提到的Hartogs定理.
\subsection{单复变量次调和函数}
我们先从单复变数的情形说起.
\begin{definition}\label{def1.5.1}
	设$\Omega$是$\mathbb{C}$中的域.如果$\Omega$上的实函数$u\colon\Omega\to\mathbb{R}\cup\{-\infty\}(u\not\equiv-\infty)$满足:
	
	(1)\hypertarget{1.5.1}{}
	$u$是上半连续的;
	
	(2)\hypertarget{1.5.1}{}
	对任意以$a$为中心,$r$为半径的闭圆盘$\bar{D}(a,r)\subset\Omega$,有不等式
	\[u(a)\le\frac1{2\pi} \int_{-\pi}^{\pi} u(a+r\ee^{\ii\theta})\mathrm{d}\theta,\]
	
	就称$u$是$\Omega$上的\textbf{次调和函数}\index{C!次调和函数}.
\end{definition}
所谓$u$在$\Omega$中上半连续\index{S!上半连续},是指对$\Omega$中每点$a$有
\[\varlimsup\limits_{z\to a\atop z\neq a}u(z)\le u(a).\]
从定义\ref{def1.5.1}马上可以得到,每个调和函数必定是次调和的.

次调和函数有下面一些基本性质.
\begin{prop}\label{prop1.5.2}
	设$u$是$\mathbb{C}$中的域$\Omega$上的次调和函数,$\varphi$是$\mathbb{R}$上递增的凸函数,那么$\varphi\circ u$也是$\Omega$上的次调和函数.
\end{prop}
\begin{proof}
	因为$u$是次调和函数,故对任意$\overline{D}(a,r)\subset\Omega$,
	\[u(a)\le\frac1{2\pi}\int_{-\pi}^{\pi} u(a+r\ee^{\ii\theta})\mathrm{d}\theta.\]
	又因为$\varphi$是递增的凸函数,所以
	\begin{align*}
		\varphi(u(a))
		&\le \varphi\left(\frac1{2\pi}\int_{-\pi}^{\pi}u(a+r\ee^{\ii\theta})\mathrm{d}\theta\right)\\
		&\le\frac1{2\pi} \int_{-\pi}^{\pi}\varphi(u(a+r\ee^{\ii\theta}))\mathrm{d}\theta.
	\end{align*}
从$\varphi$的递增性和连续性知道,$\varphi\circ u$是上半连续的,所以$\varphi\circ u$是次调和的.
\end{proof}
为了给出次调和函数的极值特征,我们引入
\begin{definition}\label{def1.5.3}
	设$\Omega$是$\mathbb{C}$中的域,$u$是$\Omega$上的实值函数,如果对任意域$G\subset\subset\Omega$,有
	\[u(z)\le\sup_{\zeta\in\partial G}u(\zeta),z\in G,\]
	就称\textbf{极大值原理}\index{J!极大值原理}对$u$成立.
\end{definition}
显然极大值原理对$\Omega$上任一实值调和函数成立.
\begin{prop}\label{prop1.5.4}
	$\Omega\subset\mathbb{C}$上任一二次连续可微的实值函数$u$,如果满足$\Delta u\ge0$,那么极大值原理成立.
\end{prop}
\begin{proof}
	先设对$\Omega$上每点都有$\Delta u>0$.如果极大值原理不成立,那么存在域$G\subset\subset\Omega$,$u$在$G$上的最大值必在$G$的内部某点$z_0$达到.记$z_0=x_0+\ii y_0$,$g(t)=u(x_0,t)$,则$g$在$t=y_0$处有极大值,所以$\ppp{u}{y} (z_0)=g''(y_0)\le 0$.同理$\ppp{u}{x} (z_0)\le 0$,因而$\Delta u\le 0$.这与假定$\Delta u>0$矛盾.
	
	现在设$\Delta u\ge 0$,命$u_\varepsilon(z)=u(z)+\varepsilon|z|^2$,这里$\varepsilon>0$是任意一个正数,于是$\Delta u_\varepsilon(z)=\Delta u(z)+4\varepsilon>0$.故极大值原理对$u_\varepsilon$成立.于是
	\[u(z)=\lim\limits_{\varepsilon\to0}u_\varepsilon(z)\le\lim\limits_{\varepsilon\to0}(\sup_{\zeta\in\partial G} u_\varepsilon(\zeta))=\sup_{\zeta\in\partial G}u(\zeta).\]
	即极大值原理对$u$成立.
\end{proof}
现在给出次调和函数的极值特征.
\begin{theorem}\label{thm1.5.5}
	设$\Omega$是$\mathbb{C}$中的域.$\Omega$上的连续实值函数$u$是$\Omega$上的次调和函数的充分必要条件是,对任意域$G\subset\subset\Omega$及任意在$\bar{G}$上连续、在$G$内调和的实值函数$h$,如果$u(z)\le h(z)$在$\partial G$上成立,那么在$G$内也有$u(z)\le h(z)$.
\end{theorem}
\begin{proof}
		必要性\quad 如果$u(z_0)>h(z_0)$对某个$z_0\in G$成立,命$u_1=u-h$,则$u_1(z_0)>0$.因为$u_1$在$\overline{G}$上连续,故在$\bar{G}$上达到它的最大值$M$.记$E=\{z\in\bar{G}\colon u_1(z)=M\}$.因为在$G$的边界上$u_1\le0$,所以$u_1$的最大值只能在$G$中取到.因此$E$是$G$中的紧子集.今设$a$是$E$的一个边界点,于是有$r>0$,使得$\bar{D}(a,r)\subset G$,但$\bar{D}(a,r)$的边界上必有某段弧不在$E$中,因而
		\begin{equation}\label{eq1.5.1}
			u_1(a)=M>\frac1{2\pi} \int_{0}^{2\pi} u_1(a+r\ee^{\ii\theta})\mathrm{d}\theta.
		\end{equation}
	另一方面,$u$和$h$分别在$G$中次调和和调和,因而有
	\[u(a)\le\frac1{2\pi}\int_{0}^{2\pi} u(a+r\ee^{\ii\theta})\mathrm{d}\theta,\]
	\[h(a)=\frac1{2\pi}\int_{0}^{2\pi}h(a+r\ee^{\ii\theta})\mathrm{d}\theta,\]
	由此即得
	\[u_1(a)\le\frac1{2\pi}\int_{0}^{2\pi}u_1(a+r\ee^{\ii\theta})\mathrm{d}\theta,\]
	这和\eqref{eq1.5.1}矛盾.
	
		充分性\quad 任取$\bar{D}(a,r)\subset\Omega$,那么存在圆盘$D(a,r)$中的调和函数$h$\index{T!调和函数},它在圆周上和$u$一致.于是假定$u(z)\le h(z)$在圆内成立.这样,
		\begin{align*}
			u(a)
			&\le h(a)=\frac1{2\pi}\int_{-\pi}^{\pi} h(a+r\ee^{\ii\theta})\mathrm{d}\theta\\
			&=\frac1{2\pi}\int_{-\pi}^{\pi}u(a+r\ee^{\ii\theta})\mathrm{d}\theta.
		\end{align*}
	所以$u$是次调和的.
\end{proof}
定理\ref{thm1.5.5}的一种等价说法是:要使$u$是$\Omega$上的次调和函数,其充分必要条件是,极大值原理对$u-h$成立.

从定理\ref{thm1.5.5}可以看出,次调和函数是凸函数概念在平面上的推广.事实上,如果把$\ddd{u}{x}=0$看做一维的Laplace方程,那么这个方程的解$u=ax+b$便是一维的调和函数.而凸函数是在任一区间的两个端点和一线性函数有相同的值,在区间内部,它不超过这个线性函数.把区间换成平面上的区域,线性函数换成二维调和函数,那么凸函数就是这儿定义的次调和函数.

作为这个定理的一个应用,我们有
\begin{theorem}\label{thm1.5.6}
	设$u$是单位圆盘$U$中的连续次调和函数,命
	\[m(r)=\frac1{2\pi}\int_{0}^{2\pi} u(r\ee^{\ii\theta})\mathrm{d}\theta,\quad 0<r<1,\]
	那么$m(r)$是$r$的非降函数.
\end{theorem}
\begin{proof}
	设$0<r_1<r_2<1$.存在圆盘$D(0,r_2)$上的调和函数$h$,它在圆周上和$u$一致.由定理\ref{thm1.5.5},在$D(0,r_2)$中有$u\le h$,因而
	\begin{align*}
		m(r_1)
		&=\frac1{2\pi}\int_{0}^{2\pi} u(r_1\ee^{\ii\theta})\mathrm{d}\theta\le\frac1{2\pi}\int_{0}^{2\pi} h(r_1\ee^{\ii\theta})\mathrm{d}\theta=h(0)\\
		&=\frac1{2\pi}\int_{0}^{2\pi}h(r_2\ee^{\ii\theta})\mathrm{d}\theta=\frac1{2\pi}\int_{0}^{2\pi} u(r_2\ee^{\ii\theta})\mathrm{d}\theta=m(r_2).\qedhere
	\end{align*}
\end{proof}
下面给出两个具体的次调和函数.
\begin{prop}\label{prop1.5.7}
	设$f$是域$\Omega$上的全纯函数,$f\not\equiv0$,那么$\log|f|$和$|f|^p$,$0<p<\infty$,都是$\Omega$上的次调和函数.
\end{prop}
	\begin{proof}
		容易知道,$\log|f|$是上半连续的.任取圆盘$\bar{D}(a,r)\subset\Omega$,我们要证明
		\begin{equation}\label{eq1.5.2}
			\log|f(a)|\le\frac1{2\pi} \int_{0}^{2\pi} \log|f(a+r\ee^{\ii\theta})|\mathrm{d}\theta.
		\end{equation}
	如果$f(a)=0$,那么$\log|f(a)|=-\infty$,\eqref{eq1.5.2}显然成立.今设$f(a)\neq0$,$f$在$\bar{D}(a,r)$中也没有其他零点,那么通过直接计算知道$\log|f|$是$\bar{D}(a,r)$中的调和函数,因而它是次调和的.现若$f$在$D(a,r)$中有零点$z_1,\cdots,z_m,$重零点按重数重复计数,根据Jensen公式\index{J!Jensen公式},
	\begin{align*}
		\log|f(a)|
		&=-\sum_{k=1}^{m}\log\frac{r}{|a-z_k|}+\frac1{2\pi}\int_{0}^{2\pi} \log|f(a+r\ee^{\ii \theta})|\mathrm{d}\theta\\
		&\le\frac1{2\pi}\int_{0}^{2\pi}\log|f(a+r\ee^{\ii\theta})|\mathrm{d}\theta.
		\end{align*}
	这就是\eqref{eq1.5.2}.这就证明了$\log|f|$的次调和性.
	
	因为$\varphi(t)=\ee^{pt}$是增加的凸函数,而
	\[|f|^p=\ee^{p\log|f|}=\varphi(\log|f|),\]
	故由命题\ref{prop1.5.2},$|f|^p$($0<p<\infty$)是次调和的.
	\end{proof}
对于二次连续可微的函数,次调和性有更简单的特征.
\begin{theorem}\label{thm1.5.8}
	设$u$是域$\Omega\subset\mathbb{C}$上二次连续可微函数,那么$u$是次调和函数的充分必要条件是$\Delta u\ge0$在$\Omega$内处处成立.
\end{theorem}
\begin{proof}
		充分性\quad 设$\Delta u\ge0$在$\Omega$内处处成立,$h$是域$G\subset\subset \Omega$上的调和函数,于是在$G$上有
		\[\Delta(u-h)=\Delta u\ge0.\]
		由命题\ref{prop1.5.4},极值原理对$u-h$成立,再由定理\ref{thm1.5.5}即知$u$是次调和函数.
		
		必要性\quad 如果存在$a\in\Omega$,使得$\Delta u(a)<0$,则有$a$的一个邻域$U(a),\Delta u(z)<0$对$z\in U(a)$成立,于是$-\Delta u(z)>0$在$U(a)$成立.由充分性证明的结果,$-u$是$U(a)$上的次调和函数,因而$u$的平均值公式在$U(a)$中的任意小圆盘成立,所以$u$是$U(a)$中的调和函数.于是$\Delta u(a)=0$这和假定$\Delta u(a)<0$矛盾.因而$\Delta u\ge0$在$\Omega$中处处成立.
\end{proof}
\subsection{多复变数次调和函数}
现在把次调和函数的概念推广到多个复变数的情形.
\begin{definition}\label{def1.5.9}
	设$\Omega$是$\mathbb{C}^n$中的域.如果$\Omega$上的实值函数$u\colon\Omega\to\mathbb{R}\cup\{-\infty\}(u\not\equiv-\infty)$满足:
	
	(1)\hypertarget{1.5.9}{}
	$u$是上半连续的;
	
	(2)\hypertarget{1.5.9}{}
	对任意以$a$为中心,$r$为半径的闭球$a+r\bar{B}\subset\Omega$,有不等式
	
	\begin{equation}\label{eq1.5.3}
		u(a)\le\frac1{2\pi} \int_{\partial B} u(a+r\zeta)\mathrm{d}\sigma(\zeta)
	\end{equation}
	
	就称$u$是$\Omega$上的\textbf{次调和函数}\index{C!次调和函数}.
\end{definition}
如果在\eqref{eq1.5.3}中以$tr(0\le t\le 1)$换$r$,两端乘$2nt^{2n-1}$,并对$t$在区间$(0,1)$中积分,那么由定理\ref{thm1.4.1},即得
\begin{equation}\label{eq1.5.4}
	u(a)\le\int_{B} u(a+rz)\mathrm{d}\nu(z).
\end{equation}

这样我们得到
\begin{prop}\label{prop1.5.10}
	如果$u$是域$\Omega\subset\mathbb{C}^n$上的次调和函数,那么对任意$\bar{B(a,r)}\subset\Omega$,\eqref{eq1.5.4}成立.
\end{prop}
从命题\ref{prop1.5.7}即可得到下面的.
\begin{prop}\label{prop1.5.11}
	设$\Omega$是$\mathbb{C}^n$中的域,$f\in H(\Omega)$,$f\not\equiv0$,那么$\log|f|$和$|f|^p(0<p<\infty)$都是$\Omega$上的次调和函数.
\end{prop}
\begin{proof}
	任取$B(a,r)\subset\Omega$,再取$R>r$,使得$B(a,R)\subset\Omega$.在单位球面上取定$\zeta$,那么单复变数$\lambda$的函数$f(a+\lambda\zeta)$在$|\lambda|<R$中全纯,由命题\ref{prop1.5.7},$\log|f(a+\lambda\zeta)|$是$|\lambda|<R$中的次调和函数,因而
	\[\log|f(a)|\le\frac1{2\pi}\int_{-\pi}^{\pi}\log|f(a+r\ee^{\ii\theta}\zeta)|\mathrm{d}\theta,\]
	两边对$\zeta$在$\partial B$上积分,并利用定理\ref{thm1.4.4}\hyperlink{1.4.4}{(1)}得
	\begin{align*}
		\log|f(a)|
		&\le \int_{\partial B}\mathrm{d}\sigma(\zeta)\frac1{2\pi}\int_{-\pi}^{\pi}\log|f(a+r\ee^{\ii\theta}\zeta)|\mathrm{d}\theta\\
		&=\int_{\partial B}\log|f(a+r\zeta)|\mathrm{d}\sigma(\zeta),
	\end{align*}
所以$\log|f|$是$\Omega$上的次调和函数.由于命题\ref{prop1.5.2}在$\mathbb{C}^n$中也成立,可以用证明命题\ref{prop1.5.7}的方法证明$|f|^p$的次调和性.
\end{proof}
\subsection{关于次调和函数的Hartogs定理}
下面的定理\ref{thm1.5.13}在证明Hartogs定理时将起关键的作用,在证明它以前先证明一个简单的事实.
\begin{prop}\label{prop1.5.12}
	如果$u_1$和$u_2$都是域$\Omega\subset\mathbb{C}^n$上的次调和函数,那么$u=\max\{u_1,u_2\}$也是$\Omega$上的次调和函数.
\end{prop}
\begin{proof}
	对任意$a\in\Omega$,$u(a)$等于$u_1(a)$或$u_2(a)$,所以对于任意的$\overline{B}(a,r)\subset\Omega$,有
	\[u(a)= u_j(a)\le\int_{\partial B}u_j(a+r\zeta)\mathrm{d}\sigma(\zeta)\le\int_{\partial B}u(a+r\zeta)\mathrm{d}\sigma(\zeta),\]
	因而$u$是次调和的.
\end{proof}
现在可以证明
\begin{theorem}[(\textbf{Hartogs})]\label{thm1.5.13}\index{D!定理!次调和函数的Hartogs定理}
	设$\Omega$是$\mathbb{C}^n$中的域,$\{u_k\}$是$\Omega$上一列次调和函数,$\alpha$,$\beta$是两个实数.如果$\{u_k\}$满足:
	
	(1)\hypertarget{1.5.13}{}
	$u_k(z)\le\beta,k=1,2,\cdots,z\in\Omega$;
	
	(2)\hypertarget{1.5.13}{}
	$\varlimsup\limits_{k\to\infty}u_k(z)\le\alpha,\quad z\in\Omega$,
	
	那么对$\Omega$中任意的紧集$K$及$\varepsilon>0$,存在仅与$K$及$\varepsilon$有关的$k_0$,使得当$k>k_0$时,
	\[u_k(z)<\alpha+\varepsilon\]
	对所有$z\in K$成立.
\end{theorem}
\begin{proof}
	如果$\alpha\ge\beta$,结论是显然的.今设$\alpha<\beta$,命
	\begin{equation}\label{eq1.5.5}
		v_k(z)=\max\{\alpha,u_k(z)\},
	\end{equation}
则有命题\ref{prop1.5.12},$v_k(z)$是$\Omega$上的次调和函数,且$\alpha\le v_k(z)\le\beta$,而由\hyperlink{1.5.13}{(2)}知,对每个$z\in\Omega$均有
\begin{equation}\label{eq1.5.6}
	\lim\limits_{k\to\infty}v_k(z)=\alpha.
\end{equation}
现在要证明,在任意紧集$K\subset\Omega$上,上面的收敛是一致的.为此,命$$g_k=\sup\{v_k,v_{k+1},\cdots\},$$则$g_k\ge g_{k+1}$,且$\alpha\le g_k\le\beta,\lim\limits_{k\to\infty}g_k(z)=\alpha$($z\in\Omega$),取定$\Omega$中的紧集$K$,一定存在$r>0$,使得对任意$z\in K$,均有$B(z,r)\subset\Omega$.命
\[h_k(z)=\int_{B}g_k(z+rw)\mathrm{d}\nu(w),\quad z\in K,\]
则由控制收敛定理,
\begin{align}\label{eq1.5.7}
	\lim_{k\to\infty} h_k(z)
	&=\int_{B}\lim_{k\to\infty}g_k(z+rw)\mathrm{d}\nu(w)\notag\\
	&=\alpha\int_{B}\mathrm{d}\nu(w)=\alpha.
\end{align}
另外从$g_k\ge g_{k+1}$易知$h_k\ge h_{k+1}$.从$\alpha\le g_k\le\beta$知道$h_k$是$K$上的连续函数,这是因为当$z$和$z'$越靠近时,$B(z,r)$和$B(z',r)$这两个球重合的部分也越大,因而$h_k(z)$和$h_k(z')$之差越小.现在根据Dini定理,\eqref{eq1.5.7}中的收敛在$K$上是一致的.由$v_k$的次调和性,用命题\ref{prop1.5.10}即得
\[v_k(z)\le\int_{B}v_k(z+rw)\mathrm{d}\nu(w)\le\int_{B}g_k(z+rw)\mathrm{d}\nu(w)=h_k(z),\quad z\in K.\]
所以当$k>k_0$时,在$K$上有
\[\alpha\le v_k(z)\le h_k(z)<\alpha+\varepsilon,\]
再由\eqref{eq1.5.5}即得
\[u_k(z)\le v_k(z)<\alpha+\varepsilon,\quad k>k_0,\quad z\in K.\qedhere\]
\end{proof}
这个定理除了在证明Hartogs定理\ref{thm1.5.14}时要用外,我们还将用它来证明关于由齐次多项式构成的级数的一个重要性质(定理\ref{thm1.5.19}).
\subsection{关于全纯函数的Hartogs定理}
现在来证明\ref{sec1.1}中提到的Hartogs定理.
\begin{theorem}[(\textbf{Hartogs})]\label{thm1.5.14}\index{D!定理!全纯函数的Hartogs定理}
	设$\Omega$是$\mathbb{C}^n$中的域,$f\colon\Omega\to\mathbb{C}$是定义在$\Omega$上的函数.对于$a\in\mathbb{C}^n$定义$\mathbb{C}$中的域
	\[\Omega_{j,a}=\{z\in\mathbb{C}\colon(a_1,\cdots,a_{j-1},z,a_{j+1},\cdots,a_n)\in\Omega\}\]
	及$\Omega_{j,a}$上的函数
	\[f_{j,a}(z)=f(a_1,\cdots,a_{j-1},z,a_{j+1},\cdots,a_n),\]
	如果对任意$a\in\mathbb{C}^n$及$j=1,\cdots,n,f_{j,a}\in H(\Omega_{j,a}),$那么$f\in H(\Omega).$
\end{theorem}
我们通过下面四个引理来完成定理的证明.
\begin{lemma}\label{lem1.5.15}
	设$\Omega=P=\prod\limits_{j=1}^{n}D(a_j,r_j)$是一个多圆柱.如果$f\colon P\to\mathbb{C}$满足定理\ref{thm1.5.14}的条件,而且$f$在$\bar{P}$上连续,那么$f$在$P$上全纯.
\end{lemma}
\begin{proof}
	记$\zeta'=(\zeta_1,\cdots,\zeta_{n-1})$,先证明当$\zeta'\in\prod\limits_{j=1}^{n-1}\partial D(a_j,r_j),z_n\in D(a_n,r_n)$,有公式
	\[f(\zeta',z_n)=\frac1{2\pi\ii}\int_{|\zeta_n-a_n|=r_n}\frac{f(\zeta',\zeta_n)}{\zeta_n-z_n}\mathrm{d}\zeta_n.\]
	事实上,当$z'=(z_1,\cdots,z_{n-1})\in\prod\limits_{j=1}^{n-1}D(a_j,r_j)$时,$f(z',z_n)$是$D(a_n,r_n)$中的全纯函数,由连续性$\lim\limits_{z'\to\zeta'}f(z',z_n)=f(\zeta',z_n)$对$z_n$在$D(a_n,r_n)$中内闭一致成立,因而由定理\ref{thm1.2.6},$f(\zeta',z_n)$在$D(a_n,r_n)$中全纯,且在$\bar{D}(a_n,r_n)$上连续,因而由单复变的Cauchy积分公式,上式成立.同样道理,
\[
		f(\zeta_1,\cdots,\zeta_{n-2},z_{n-1},z_n)
		=\frac1{2\pi\ii}\int\limits_{|\zeta_{n-1}-a_{n-1}|=r_{n-1}}\frac{f(\zeta_1,\cdots,\zeta_{n-1},z_n)}{\zeta_{n-1}-z_{n-1}}\mathrm{d}\zeta_{n-1},
\]
把上面的$f(\zeta',z_n)$代进来,即得
\[
	f(\zeta_1,\cdots,\zeta_{n-2},z_{n-1},z_n)
	=\frac1{(2\pi\ii)^2}\int\limits_{|\zeta_n-a_n|=r_n\atop |\zeta_{n-1}-a_{n-1}|=r_{n-1}}\frac{f(\zeta_1,\cdots,\zeta_{n-2},\zeta_{n-1},\zeta_n)}{(\zeta_{n-1}-z_{n-1})(\zeta_n-z_n)}\mathrm{d}\zeta_{n-1}\mathrm{d}\zeta_n.
\]
重复上面的过程,即得
\[f(z)=\frac1{(2\pi\ii)^n}\int\limits_{\partial_0 P}\frac{f(\zeta)}{\prod\limits_{j=1}^n(\zeta_j-z_j)}\mathrm{d}\zeta_1\cdots\mathrm{d}\zeta_n,\quad z\in P.\]
再利用证明定理\ref{thm1.2.2}的方法,即可得$f$的幂级数展开式
\[f(z)=\sum_\alpha a_\alpha(z-a)^\alpha,\]
这里
\[a_\alpha=\frac1{(2\pi\ii)^n}\int\limits_{\partial_0 P}\frac{f(\zeta)}{\prod\limits_{j=1}^n (\zeta_j-a_j)^{\alpha_j+1}}\mathrm{d}\zeta_1\cdots\mathrm{d}\zeta_n.\]
因而$f$在$P$上全纯.
\end{proof}
\begin{lemma}\label{lem1.5.16}
	设$P=\prod\limits_{j=1}^{n}D(a_j,r_j)$是一个多圆柱.如果$f\colon P\to\mathbb{C}$满足定理\ref{thm1.5.14}的条件,而且$f$在$P$中有界,那么$f$在$P$上全纯.
\end{lemma}
\begin{proof}
	根据引理\ref{lem1.5.15},只要证明$f$在$P$上连续就行了.为简单起见,不妨设$a=0$,设$f$在$P$上满足$|f(z)|\le M$.任取$z,w\in P$,可写
\[
		f(z)-f(w)
		=\sum_{j=1}^{n}[f(w_1,\cdots,w_{j-1},z_j,\cdots,z_n)-f(w_1,\cdots,w_{j-1},w_j,z_{j+1},\cdots,z_n)],
\]
任意固定$j$,当$|\lambda|<r_j$时,记
\[
	\varphi(\lambda)
	=f(w_1,\cdots,w_{j-1},z_j,\cdots,z_n)-f(w_1,\cdots,w_{j-1},w_j,z_{j+1},\cdots,z_n).
\]
于是$|\varphi(\lambda)|\le 2M$,且$\varphi(w_j)=0$,因而$\varphi$是把$D(0,r_j)$映入$D(0,2M)$且把$w_j$映为$0$的全纯映射.设$\psi$是把$D(0,2M)$映为$D(0,1)$,且把$0$映为$0$的全纯映射,$\eta$是把$D(0,r_j)$映为$D(0,1)$且把$w_j$映为$0$的全纯映射,那么$\psi\circ\varphi\circ\eta^{-1}$是把$D(0,1)$映入$D(0,1)$且把$0$映为$0$的全纯映射,根据Schwarz引理,$|\psi\circ\varphi\circ\eta^{-1}(z)|\le|z|$,或者
\[|\psi(\varphi(\lambda))|\le|\eta(\lambda)|,\quad\lambda\in D(0,r_j).\]
由此即得\quad $|\varphi(\lambda)|\le 2Mr_j\left|\frac{\lambda-w_j}{r_j^2-\bar{w}_j \lambda}\right|$,让$\lambda=z_j$,得
\[|\varphi(z_j)|\le 2Mr_j\left|\frac{z_j-w_j}{r_j^2-\bar{w}_jz_j}\right|.\]
于是
\[|f(z)-f(w)|\le\sum_{j=1}^{n}|\varphi(z_j)|\le 2M\sum_{j=1}^{n}\frac{r_j|z_j-w_j|}{|r_j^2-\bar{w}_jz_j|},\]
这就证明了$f$的连续性.
\end{proof}
\begin{lemma}\label{lem1.5.17}
	设$\bar{P}=\prod\limits_{j=1}^{n}\bar{D}(a_j,r_j)$是一个闭多圆柱.如果当$z_n\in\bar{D}(a_n,r_n)$固定时,$f\colon \bar{P}\to\mathbb{C}$是$z'\in\prod\limits_{j=1}^{n-1} \bar{D}(a_j,r_j)$的连续函数;当$z'\in\prod\limits_{j=1}^{n-1} \bar{D}(a_j,r_j)$固定时,$f$是$z_n\in\bar{D}(a_n,r_n)$的连续函数,那么一定存在$\bar{D}(a_j',r_j')\subset\bar{D}(a_j,r_j),j=1,\cdots,n-1,$使得$f$在
	\[\bar{P}'=\prod_{j=1}^{n-1}\bar{D}(a_j',r_j')\times\bar{D}(a_n,r_n)\]
	中有界.
\end{lemma}
\begin{proof}
	命$E_k=\left\{z'\in\prod\limits_{j=1}^{n-1}\bar{D}(a_j,r_j)\colon|f(z',z_n)|\le k\text{对任意}z_n\in\bar{D}(a_n,r_n)\text{成立}\right\}$.显然
	\[\prod_{j=1}^{n-1}\bar{D}(a_j,r_j)=\bigcup_{k=1}^\infty E_k.\]
	根据Baire定理,必有某个$E_{k_0}$不是疏集,这就是说存在$\prod\limits_{j=1}^{n-1}\bar{D}(a_j,r_j)$中的开球$B,E_{k_0}$在$B$中是稠的.又因为$f$对$z'$是连续的,所以$E_{k_0}$是闭的,因而$B\subset E_{k_0}$.这说明$E_{k_0}$必有内点.故存在$\bar{D}(a_j',r_j')\subset\bar{D}(a_j,r_j),j=1,\cdots,n-1,$使得$\prod\limits_{j=1}^{n-1}\bar{D}(a_j',r_j')\subset E_{k_0}$.于是当$z\in\prod\limits_{j=1}^{n-1}\bar{D}(a_j',r_j')\times\bar{D}(a_n,r_n)$时,有$|f(z)|\le k_0$.
\end{proof}
\begin{lemma}\label{lem1.5.18}
	设$R>r>0$,$f$是多圆柱$P=\prod\limits_{j=1}^{n}D(a_j,R)$上的复函数.如果它满足:
	
	(1)\hypertarget{1.5.18}{}
	对于每个$z_n\in D(a_n,R)$,函数$z'\to f(z',z_n)$在$\prod\limits_{j=1}^{n-1} D(a_j,R)$上全纯;
	
	(2)\hypertarget{1.5.18}{}
	$f$在多圆柱$\prod\limits_{j=1}^{n-1}D(a_j,r)\times D(a_n,R)$中全纯且有界.
	
	那么$f$在$P$中全纯.
\end{lemma}
\begin{proof}
	为简单起见,不妨设$a=0$,因为当固定$z_n$时,$f$是$z'$的全纯函数,故有幂级数展开式
	\begin{equation}\label{eq1.5.8}
		f(z)=\sum_\alpha a_\alpha(z_n)z'^{\alpha},\quad (z',z_n)\in P,
	\end{equation}
这里$\alpha=(\alpha_1,\cdots,\alpha_{n-1})$是$n-1$重指标.由于
\[a_\alpha(z_n)=\frac{(\DD^\alpha f)(0,z_n)}{\alpha!},\]
故由\hyperlink{1.5.18}{(2)}知,$a_\alpha$是$|z_n|<R$中的全纯函数.现取$0<R_2<R$,因为当$z_n$固定时,\eqref{eq1.5.8}式右端的级数对$z'$是内闭一致收敛的,因而
\[\left|a_\alpha(z_n)R_2^{|\alpha|}\right|\to 0,\text{当}|\alpha|\to\infty,|z_n|<R.\]
所以当$|\alpha|$充分大时有
\[\frac1{|\alpha|}\log|a_\alpha(z_n)|<\log\frac1{R_2},\]
或者
\begin{equation}\label{eq1.5.9}
	\varlimsup_{|\alpha|\to\infty} \frac1{|\alpha|}\log|a_\alpha(z_n)|\le\log\frac1{R_2}.
\end{equation}
再由假定\hyperlink{1.5.18}{(2)},$f$在$\prod\limits_{j=1}^{n-1}D(a_j,r)\times D(a_n,R)$中有界,设其模不超过$M$,则由定理\ref{thm1.2.3}的Cauchy不等式得
\[\left|a_\alpha(z_n)r^{|\alpha|}\right|\le M,\]
即
\begin{equation}\label{eq1.5.10}
	\frac1{|\alpha|}\log|a_\alpha(z_n)|\le\log M+\log\frac1r.
\end{equation}
由命题\ref{prop1.5.7}知道,函数$z_n\to\frac1{|\alpha|}\log|a_\alpha(z_n)|$是圆盘$|z_n|<R$中的次调和函数族,根据不等式\eqref{eq1.5.9}、\eqref{eq1.5.10},利用定理\ref{thm1.5.13},对任意$0<R_1<R_2$,当$|\alpha|$充分大时,在紧集$|z_n|\le R_1$上成立不等式
\[\frac1{|\alpha|}\log|a_\alpha(z_n)|\le\log\frac1{R_1},\]
即当$|z_n|\le R_1$时有$\left|a_\alpha(z_n)R_1^{|\alpha|}\right|\le1$.现在很容易证明级数\eqref{eq1.5.8}在多圆柱$\{z\colon|z_j|<R_1,j=1,\cdots,n\}$中内闭一致收敛.事实上,任取$\rho<R_1$,则当$|z_j|\le\rho$时,$j=1,\cdots,n$,有
\begin{align*}
	\sum_\alpha |a_\alpha(z_n)z'^{\alpha}|
	&=\sum_\alpha\left|a_\alpha(z_n)R_1^{|\alpha|}\right|\left(\frac{|z_1|}{R_1}\right)^{\alpha_1}\cdots\left(\frac{|z_{n-1}|}{R_1}\right)^{\alpha_{n-1}}\\
	&\le\sum_\alpha\left(\frac{\rho}{R_1}\right)^{\alpha_1}\cdots \left(\frac{\rho}{R_1}\right)^{\alpha_{n-1}}=\left(1-\frac{\rho}{R_1}\right)^{-(n-1)}.
\end{align*}
由于$R_1$可任意接近$R_2,R_2$可任意接近$R$,因此级数\eqref{eq1.5.8}在$P$中内闭一致收敛,再注意到$a_\alpha(z_n)z'^{\alpha}$是$P$中的全纯函数,由定理\ref{thm1.2.6},\eqref{eq1.5.8}收敛到$P$中的全纯函数,即$f\in H(P)$.
\end{proof}
\begin{proof}[\textbf{定理\ref{thm1.5.14}的证明}]
	对维数用归纳法证明.$n=1$时,定理当然成立.现设对$j=1,2,\cdots,n-1$各维数定理已成立.任取$a\in\Omega$,适当取$R>0$,使得多圆柱$\prod\limits_{j=1}^n \bar{D}(a_j,2R)\subset\Omega$,由归纳假定,当$z_n$固定时,$f(z',z_n)$是$z'$的全纯函数,因而是连续函数.当$z'$固定时,$f(z',z_n)$是$z_n$的全纯函数,因而也是连续的.于是由引理\ref{lem1.5.17},存在多圆柱$\bar{P}'=\prod\limits_{j=1}^{n-1}\bar{D}(a_j',r_j)\times\bar{D}(a_n,R)$,其中$\bar{D}(a_j',r_j)\subset\bar{D}(a_j,R),j=1,\cdots,n-1,$使得$f$在$\bar{P}'$上有界.由引理\ref{lem1.5.16}知道,$f$在$\bar{P}'$上全纯.现在命
	\[Q=\prod_{j=1}^{n-1}D(a_j',R)\times D(a_j,R),\]
	则$Q\subset\Omega$,记$r=\min\{r_j\colon j=1,\cdots,n-1\}$,则$r<R$,且$f$在多圆柱
	\[\prod_{j=1}^{n-1}D(a_j',r)\times D(a_n,R)\]
	上全纯且有界,故由引理\ref{lem1.5.18},$f\in H(Q)$,因为$|a_j-a_j'|<R,j=1,\cdots,n-1,$所以$a\in Q$,故$f$在$a$的邻域中全纯.
\end{proof}
\subsection{由齐次多项式构成的级数}
作为次调和函数理论的另一个应用,我们证明由齐次多项式构成的级数的一个性质.
\begin{theorem}\label{thm1.5.19}
	设$\Omega$是$\mathbb{C}^n$中的域,$F_s$是$s$次齐次多项式,如果对每个固定的$z\in\Omega$,$\{F_s(z)\}$是一有界数列,那么级数$\sum\limits_{s=0}^\infty F_s(z)$在$\Omega$上内闭一致收敛,因而其和函数在$\Omega$上全纯.
\end{theorem}
我们需要下面两个引理.
\begin{lemma}\label{lem1.5.20}
	设$P(\lambda)$是单复变数$\lambda$的$k$次多项式,在单位圆周上满足$|P(\lambda)|\le1$,那么对$1\le|\lambda|<\infty$,有$|P(\lambda)|\le|\lambda|^k$.
\end{lemma}
\begin{proof}
	设$P(\lambda)=a_0+a_1\lambda+\cdots a_k\lambda^k$.由假定$|P(\lambda)|\le|\lambda|^k$因而$\frac1{2\pi}\int_{0}^{2\pi}|P(\ee^{\ii\theta})|^2\mathrm{d}\theta\le1$,由此即得$\sum\limits_{l=0}^k |a_l|^2\le1$,当然$|a_k|\le1$.因为$\frac{P(\lambda)}{\lambda^k}$在$|\lambda|>1$中全纯,而且$\lim\limits_{\lambda\to\infty}\frac{P(\lambda)}{\lambda^k}=a_k$,而在单位圆周上,它的模最多是$1$,因此由最大模原理$\left|\frac{P(\lambda)}{\lambda^k}\right|\le1$($1\le|\lambda|<\infty$).
\end{proof}
\begin{lemma}\label{lem1.5.21}
	设$\Omega$是$\mathbb{C}^n$中的域,$F_s$是$s$次齐次多项式.如果对每个$z\in\Omega$,$\{F_s(z)\}$是一有界数列,那么$|F_s|^{\frac1s}$在$\mathbb{C}^n$的任一紧集上一致有界.
\end{lemma}
\begin{proof}
	命
	\[A_k=\{z\in\bar{\Omega}\colon|F_s(z)|\le k,s=0,1,2,\cdots\}.\]
	因为对每个$z\in\Omega$,$\{F_s(z)\}$是有界数列,且因$F_s(z)$是多项式,故$\overline{\Omega}$中每一点必属于某个$A_k$,即
	\[\bar{\Omega}=\bigcup_{k=1}^\infty A_k.\]
	由Baire定理\index{D!定理!Baire定理},$A_k$($k=1,2,\cdots$)中至少有一个不是$\Omega$中的疏集,不妨设$A_l$不是疏集,因而必在$\Omega$的某个球$B(a,r)$中稠密.于是对任意$b\in B(a,r)$,必有一列$z^{(m)}\in A_l$,使得$\lim\limits_{m\to\infty}z^{(m)}=b$.由于$|F_s(z^{(m)})|\le l,s=0,1,\cdots,$让$m\to\infty$,即得$|F_s(b)|\le l,s=0,1,\cdots$.这就证明了$F_s$在$\bar{B}(a,r)$中一致有界.现设$K$是$\mathbb{C}^n$中任意紧集,则存在充分大的$\rho$,使得$K\subset B(a,\rho)$.我们证明$|F_s|^{\frac1s}$在$r<|w-a|\le\rho$中有界.为此,设$\zeta$是单位球面上的一个固定点,命
	\[P(\lambda)=\frac1l F_s(a+\lambda r\zeta),\]
	因为$F_s$是$s$次多项式,所以$P(\lambda)$是单复变数$\lambda$的$s$次多项式.当$|\lambda|\le1$时,由上面所证$|P(\lambda)|\le1$.于是,由引理\ref{lem1.5.20}得到,当$|\lambda|>1$时,$|F_s(a+\lambda r\zeta)|\le l|\lambda|^s$.记$w=a+\lambda r\zeta$,则当$r<|w-a|\le\rho$时,$1<|\lambda|\le\frac{\rho}{r}$,因而
	\[|F_s(w)|^{\frac1s}\le l^{\frac1s}|\lambda|<\frac1r l\rho.\]
	这就证明了$|F_s|^{\frac1s}$在$K$中一致有界.
\end{proof}
\begin{proof}[\textbf{定理\ref{thm1.5.19}的证明}]
	命$u_s=|F_s|^{\frac1s}$,由命题\ref{prop1.5.11},$u_s$是次调和函数.由引理\ref{lem1.5.21},它在$\MC^n$的任意紧集上一致有界.今取$\Omega$的紧子集$K$,取$\varepsilon$充分小,使得$(1+\varepsilon)^2K\subset\Omega$.于是,$u_s$在$(1+\varepsilon)^2 K$上一致有界.由于对于给定的$z\in\Omega,F_s(z)$是有界数列,所以
	\[\varlimsup_{s\to\infty}u_s(z)\le1,\quad z\in\Omega.\]
	于是由定理\ref{thm1.5.13},对于给定的$\varepsilon>0$,存在$s_0$,当$s>s_0$时,$u_s(z)<1+\varepsilon$在$(1+\varepsilon)^2 K$中成立,即
	\[|F_s((1+\varepsilon)^2 z)|<(1+\varepsilon)^s,z\in K,s>s_0.\]
	因为$F_s$是$s$次多项式,所以
	\[|F_s(z)|<\frac1{(1+\varepsilon)^s},z\in K,s>s_0.\]
	这就证明了$\sum\limits_{s=0}^\infty F_s(z)$在$K$上一致收敛.
\end{proof}
\section{Riemann可去奇点定理和Rad\'o定理\label{sec1.6}}
\subsection{Riemann可去奇点定理}
在\ref{sec1.3}中,我们提到过全纯开拓的问题.那里讨论的是某一类域,在其上全部全纯函数都能全纯开拓到更大的域上去.现在我们要讨论,对于给定的域,在其某个子集上全纯的函数,在什么条件下能全纯开拓到整个域上去.

单复变中全纯函数的孤立奇点分为三类:可去奇点、极点和本性奇点.$a$是$f$的可去奇点的充分必要条件是$f$在$a$的邻域(不包括$a$)中全纯且有界.如果把$a$看成是全纯函数$g(z)=z-a$的零点,那么可去奇点的定理可以叙述为:若$f$在全纯函数$g$的零点的邻域内(不包括零点本身)全纯并且有界,则$f$可全纯开拓到整个邻域(包括零点).这种形式的可去奇点定理在多复变中有相应的推广.
\begin{theorem}[(\textbf{Riemann可去奇点定理})]\label{thm1.6.1}\index{D!定理!Riemann可去奇点定理}
	设$\Omega$是$\mathbb{C}^n$中的域,$g\in H(\Omega),g\not\equiv0,E=\{z\in\Omega\colon g(z)=0\}$.如果$f\in H(\Omega\setminus E)$,且在$\Omega\setminus E$上有界,那么存在唯一的全纯函数$F\in H(\Omega)$,使得$F|_{\Omega\setminus E}=f$.
\end{theorem}
在给出定理的证明之前,先证明一个引理.
\begin{lemma}\label{lem1.6.2}
	设$G$是$\mathbb{C}^n$中原点的邻域,$g\in H(G),g(0)=0,g\not\equiv0$.那么存在非奇异的线性变换$z=wT,T$是$n$阶可逆方阵,使得函数$\tilde{g}=g(wT)$具有下面的性质:
	
	(1)\hypertarget{1.6.2}{}
	$\tilde{g}(0,\cdots,0,w_n)$在$w_n=0$处有$k$阶零点,$k$是某个自然数;
	
	(2)\hypertarget{1.6.2}{}
	存在以原点为中心的多圆柱$\Delta=\Delta'\times\Delta_n\subset G$(这里$\Delta'$是$\mathbb{C}^{n-1}$中以原点为中心的多圆柱,$\Delta_n$是以原点为中心的圆盘),使得对每个$z'\in\Delta',\tilde{g}(z',\lambda)$在$\Delta_n$中恰有$k$个零点.
	\begin{proof}
		因为$g(0)=0$,$g$在原点附近的Taylor展开为
		\[g(z)=\sum_{j=k}^{\infty}P_j(z),\quad k\ge1,\quad P_k(z)\not\equiv0,\]
		这里$P_j(z)=\sum\limits_{|\alpha|=j}c_\alpha z^\alpha$是$j$次齐次多项式.任选$b=(b_1,\cdots,b_n)$,使得$P_k(b)\neq0$,适当选择$t_{ij}$,使下述线性变换是非奇异的,
		\[z_i=\sum_{j=1}^{n-1} t_{ij}w_j+b_i w_n,\quad i=1,\cdots,n.\]
		于是
		\begin{align*}
			\tilde{g}(0,\cdots,0,w_n)
			&=g(b_1w_n,\cdots,b_nw_n)=\sum_{j=k}^{\infty}P_j(b_1w_n,\cdots,b_nw_n)\\
			&=\sum_{j=k}^{\infty}w_n^j P_j(b)=w_n^k\varphi(w_n),
		\end{align*}
	这里$\varphi(0)=P_k(b)\neq0$.这就证明了\hyperlink{1.6.2}{(1)}.
	
	现在证明\hyperlink{1.6.2}{(2)}.因为$\tilde{g}(0',\lambda)$在$\lambda=0$处有$k$阶零点,由于单复变数全纯函数的零点是孤立的,所以存在$r>0$,使得$\tilde{g}(0',\lambda)$在圆盘$\Delta_n=\{\lambda\colon|\lambda|\le r\}$中除去$\lambda=0$外没有其他零点.由连续性,存在$\delta>0$和$\MC^{n-1}$中以原点为中心的多圆柱$\Delta'$,使得当$z'\in\Delta',|\lambda|=r$时,$|\tilde{g}(z',\lambda)|>\delta$,而且$\Delta=\Delta'\times\Delta_n\subset G$.命
	\[J(z')=\frac1{2\pi\ii}\int\limits_{|\lambda|=r}(\tilde{g}(z',\lambda))^{-1} \pp{\tilde{g}(z',\lambda)}{\lambda}\mathrm{d}\lambda,\quad z'\in\Delta' .\]
	因为$|\tilde{g}(z',\lambda)|>\delta$,所以$J$是$\Delta'$中的连续函数.由单复变的知识知道,$J(z')$是$\tilde{g}(z',\lambda)$在$\Delta_n$中的零点的个数,所以$J(z')$是一个取自然数值的函数,但它又是连续的,故只能是常数.现已知$J(0')=k$,因而$J(z')\equiv k$.这就证明了\hyperlink{1.6.2}{(2)}.
	\end{proof}
\begin{proof}[\textbf{定理\ref{thm1.6.1}的证明}]
	任取$a\in E$,只要证明$f$可以在$a$的邻域中全纯开拓就行了.为简单起见,无妨设$a=0$.因为$g(0)=0$,由引理\ref{lem1.6.2}\hyperlink{1.6.2}{(1)},经过可逆的线性变换后,存在以原点为中心的多圆柱$\Delta=\Delta'\times\Delta_n$,当$z'\in\Delta',|\lambda|=r$时,$|g(z',\lambda)|>\delta$,因而这样的$(z',\lambda)\in\Omega\setminus E$,所以$f(z',\lambda)$有意义.命
	\[F(z',z_n)=\frac1{2\pi\ii}\int\limits_{|\lambda|=r} \frac{f(z',\lambda)}{\lambda-z_n}\mathrm{d}\lambda,\quad |z_n|<r.\]
	由于$(z',\lambda)\in\Omega\setminus E$时,$f(z',\lambda)$是$z'$的全纯函数,所以$F(z',z_n)$是$z'$的全纯函数.$F(z',z_n)$作为Cauchy积分,当然也是$z_n$的全纯函数.由Hartogs定理,$F\in H(\Delta'\times\Delta_n)$.由引理\ref{lem1.6.2}\hyperlink{1.6.2}{(2)},当$z'\in\Delta'$固定时,$g(z',\lambda)$在$\Delta_n$中只有有限个零点,这些零点都是$f(z',\lambda)$的可去奇点,所以$f(z',\lambda)$可以全纯开拓到$\Delta_n$中,故由Cauchy积分公式,
	\[F(z',z_n)=f(z',z_n).\]
	因而$F$是$f$在$\Delta'\times\Delta_n$中的全纯开拓.
\end{proof}
\end{lemma}
\subsection{Rad\'o定理}
下面的Rad\'o定理则是另一种形式的延拓.
\begin{theorem}\label{thm1.6.3}\index{D!定理!Rad\'o定理}
	设$\Omega$是$\mathbb{C}^n$中的域,$f\colon\Omega\to\mathbb{C}$是连续的,记$E=\{z\in\Omega\colon f(z)=0\}$.如果$f\in H(\Omega\setminus E)$,那么$f\in H(\Omega)$.
\end{theorem}
\begin{proof}
	根据Hartogs定理,我们只要证明$n=1$的情形就够了.不妨设$\bar{U}\subset\Omega$,这里$U$是单位圆盘.因此$f\in C(\Omega)$,所以$f$在$\bar{U}$上有界,不妨设$|f(z)|<1,z\in\bar{U}$.记$g=P[f]$\index[symbolindex]{\textbf{函数和映射}!$P[f]$}是$f$的Poisson积分\index{P!Poisson积分},即
	\[g(r\ee^{\ii\theta})=\frac1{2\pi}\int_{0}^{2\pi}\frac{1-r^2}{1-2r\cos(\theta-t)+r^2}f(\ee^{\ii t})\mathrm{d}t,\]
	那么$g$是$U$中的调和函数.命
	\[\varphi=\Re(f-g)+\alpha\log|f|,\]
	这里$\alpha>0$是个正常数,那么$\varphi$在$U\setminus E$中调和.
	
	注意,当$z\to z_0\in E$($z\in U\setminus E$)时,$\varphi(z)\to-\infty$;当$z\to\ee^{\ii\theta}\in T$($T$是单位圆周)时,$\varphi(z)\to\alpha\log|f(\ee^{\ii\theta})|<0$.根据调和函数的极值原理,当$z\in U\setminus E$时,$\varphi(z)<0$.让$\alpha\to0$,我们得到
	\[\Re(f(z)-g(z))\le0,\quad z\in U\setminus E.\]
	对$\alpha<0$进行同样的讨论,就可得到
	\[\Re(f(z)-g(z))\ge0,\quad z\in U\setminus E.\]
	同样的讨论对$f-g$的虚部也成立.因而有
	\[f(z)=g(z),\quad z\in T\cup\bar{U\setminus E}.\]
	现设$z\in\partial E$,则$f(z)=0$,因而$g(z)=0$,但$g$是调和函数,所以$g(z)\equiv0(z\in E)$.这样一来就有
	\[f(z)\equiv g(z),\quad z\in\bar{U}.\]
	所以$f\in C^1(U)$,而且$\pp{f}{\bar{z}}=0,z\in U\setminus E$.如果$E$有内点的话,它在这些内点上也成立.由连续性,$\pp{f}{\bar{z}}=0$在$U$上成立,因而$f\in H(U)$.
\end{proof}
定理\ref{thm1.6.1}和定理\ref{thm1.6.3}在第\ref{chap2}章讨论全纯映射的逆映射时都要用到(见定理\ref{thm2.2.3}).
\section*{注记}\addcontentsline{toc}{section}{注记}
本章内容是全书的基础.\ref{sec1.1}和\ref{sec1.2}是单复变的一些主要结果在多复变中的推广.\ref{sec1.3}中揭示的Hartogs现象是F. Hartogs于1906年发现的,Hartogs的这个重要发现标志着多复变理论的真正开始.在这个基础上产生了全纯域、全纯凸域和拟凸域等一系列新概念,以及研究多复变的一系列新方法,长期以来,它们推动着多复变理论的发展.次调和函数的概念也是F. Hartogs首先引进的,定理\ref{thm1.5.13}首先由Hartogs证明,它是定理\ref{thm1.5.14}和定理\ref{thm1.5.19}证明的基础.由Hartogs首先证明的定理\ref{thm1.5.14}是一个重要的基本定理,即使到现在也没发现一个比较简单的证明.

本章内容可参阅\cite{narasimhan1971several},\cite{hormander1973introduction},\cite{krantz2001function}和\cite{rudin2008function}.
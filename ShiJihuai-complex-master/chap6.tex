\chapter{$\bar{\partial}$问题及其应用\label{chap6}}
第\ref{chap5}章中我们已经证明了全纯域一定是拟凸域.反过来的问题是:拟凸域是否一定是全纯域?这就是著名的 Levi 猜测或 Levi 问题.20世纪60年代中期,由 Andreotti,Vesentini,Morrey,Kohn 和 H\"ormander 等人发展起来的$L^2$估计方法,把 Levi 问题的研究归结为$\bar{\partial}$问题的可解性及对解的估计.本章将介绍$L^2$方法,并用它来解决 Levi 问题.
\section{两项准备知识\label{sec6.1}}
在下面的讨论中,我们将要用到函数的光滑化方法和Sobolev空间的概念,作为准备知识,我们将分别予以介绍.
\subsection{函数的光滑化\index{H!函数的光滑化}}
函数的光滑化是指,对任意$u\in L^2(\MR^N)$,要找一列函数$u_l\in(C^\infty\cap L^2)(\MR^N)$\index[symbolindex]{\textbf{函数和映射}!$(C^\infty\cap L^2)(\MR^N)$},使得当$l\to\infty$时,有$\Vert u-u_l\Vert_{L^2}\to0$.这个从$u$找$\{u_l\}$的方法称为函数的光滑化方法或Friedrichs正则化方法\index{F!Friedrichs正则化方法}.其实,类似的方法在证明定理\ref{thm5.4.8}和定理\ref{thm5.5.10}时都已用过.

如证明定理\ref{thm5.4.8}那样,取$\chi\colon \MR^N\to\MR$,使其满足下列条件:

(1) $\chi\in C^\infty(\MR^N)$;

(2) $\supp\chi$是紧集;

(3) $\chi\ge0$;

(4) $\int_{\MR^N}\chi(x)\dif m(x)=1$.

这样的函数不难找到.例如,取
\[\psi(x)=\begin{cases}
	\exp\left(\frac1{|x|^2-1}\right),&|x|<1,\\
	0,&|x|\ge1.
\end{cases}\]
如果记$c=\int_{\MR^N}\psi(x)\dif m(x)$,那么$\chi(x)=\frac1c \psi(x)$就是满足上述条件的函数.

现在可以证明
\begin{theorem}\label{thm6.1.1}
	设$\chi$是$C_0^\infty(\MR^N)$中的非负函数,且满足
	\[\int_{\MR^N}\chi(x)\dif m(x)=1,\quad \supp\chi\subset\left\{x\in\MR^N\colon |x|\le1\right\},\]
	记$\chi_\varepsilon(x)=\frac1{\varepsilon^N}\chi\left(\frac{x}{\varepsilon}\right)$.如果对$u\in L^2(\MR^N)$,命\index[symbolindex]{\textbf{其它符号}!$u_\varepsilon$}\index[symbolindex]{\textbf{其它符号}!$(u\ast\chi_\varepsilon)(x)$}
	\[u_\varepsilon(x)=(u\ast\chi_\varepsilon)(x)=\int_{\MR^N}u(x-y)\chi_\varepsilon(y)\dif m(y),\]
	那么
	
	(1)\hypertarget{6.1.1}{}
	$u_\varepsilon\in(C^\infty\cap L^2)(\MR^N)$;
	
	(2)\hypertarget{6.1.1}{}
	$\supp u_\varepsilon\subset\{x\colon d(x,\supp u)\le\varepsilon\}$;
	
	(3)\hypertarget{6.1.1}{}
	当$\varepsilon\to0$时,$\Vert u_\varepsilon -u\Vert_{L^2}\to0$.
	
	这里的\hyperlink{6.1.1}{(2)}说明,如果$\supp u$是紧的,那么$\supp u_\varepsilon$也是紧的;\hyperlink{6.1.1}{(3)}说明$\{u_\varepsilon\}$就是$u$的一个光滑化序列.
\end{theorem}
\begin{proof}
	\hyperlink{6.1.1}{(1)}
	把$u_\varepsilon(x)$改写成
	\[u_\varepsilon(x)=\int_{\MR^N}u(y)\chi_\varepsilon(x-y)\dif m(y),\]
	便知$u_\varepsilon(x)\in C^\infty(\MR^N)$.下面证明$\Vert u_\varepsilon\Vert_{L^2}\le\Vert u\Vert_{L^2}$.事实上
	\begin{align*}
		\Vert u_\varepsilon\Vert_{L^2}^2
		&=\int_{\MR^N}|u_\varepsilon(x)|^2 \dif m(x)\\
		&=\int_{\MR^N}\left|\int_{\MR^N} u(x-y)\chi_\varepsilon(y)\dif m(y)\right|^2\dif m(x)\\
		&\le\int_{\MR^N}\left\{\int_{\MR^N}|u(x-y)|\chi_\varepsilon(y)\dif m(y)\right\}^2\dif m(x)\\
		&\le\int_{\MR^N}\left\{\int_{\MR^N}|u(x-y)|^2\chi_\varepsilon(y)\dif m(y)\right\}\cdot\left\{\int_{\MR^N}\chi_\varepsilon(y)\dif m(y)\right\}\dif m(x)\\
		&=\int_{\MR^N} \chi_\varepsilon(y)\dif m(y)\int_{\MR^N}|u(x-y)|^2\dif m(x)=\Vert u\Vert_{L^2}^2,
	\end{align*}
这里我们用了等式$\int_{\MR^N}\chi_\varepsilon(y)\dif m(y)=1$.由于$u\in L^2(\MR^N)$,所以$u_\varepsilon(x)\in L^2(\MR^N)$.

\hyperlink{6.1.1}{(2)}
任取$x_0\in\supp u_\varepsilon$.我们先设$u(x_0)\neq0$.如果$d(x_0,\supp u)>\varepsilon$,那么$B(x_0,\varepsilon)\cap\supp u=\varnothing$.由于$\supp\chi_\varepsilon=\{\chi\colon|\chi|\le\varepsilon\}$,所以
\begin{align*}
	u_\varepsilon(x_0)
	&=\int_{|y|\le\varepsilon}u(x_0-y)\chi_\varepsilon(y)\dif m(y)\\
	&=\int_{|x_0-w|\le\varepsilon}u(w)\chi_\varepsilon(x_0-w)\dif m(w)=0,
\end{align*}
这和$u_\varepsilon(x_0)\neq0$矛盾.再设$u_\varepsilon(x_0)=0$,则必有$x_k\to x_0$,而$u_\varepsilon(x_k)\neq0$.由上面的讨论知道,$d(x_k,\supp u)\le\varepsilon$.让$k\to\infty$,即得$d(x_0,\supp u)\le\varepsilon$.

\hyperlink{6.1.1}{(3)}
由于$u\in L^2(\MR^N)$,对于任给的$\delta>0$,存在具有紧支集的连续函数$v$,使得
\begin{equation}\label{eq6.1.1}
	\Vert u-v\Vert_{L^2}<\frac{\delta}{3}.
\end{equation}
由\hyperlink{6.1.1}{(1)}所证,对于任意的$\varepsilon>0$有
\begin{equation}\label{eq6.1.2}
	\Vert u_\varepsilon-v_\varepsilon\Vert_{L^2}\le\Vert u-v\Vert_{L^2}<\frac{\delta}{3}.
\end{equation}
记$K=\supp v$,则$K$是紧集,取定$\eta>0$,记
\[K_\eta=\left\{x\in\MR^N\colon d(x,K)\le\eta\right\},\]
则$K_\eta$也是紧集.由于$v$在$K_\eta$上连续,因而一致连续,故存在$\eta'>0$,当$x,x'\in K_\eta$,且$|x-x'|<\eta'$时,$|v(x)-v(x')|<\frac{\delta}{3m(K_\eta)}$,这里$m(K_\eta)$记$K_\eta$在$\MR^N$中的体积.今取$\varepsilon<\min\left\{\eta',\frac12 \eta\right\}$,则当$x\in K_\varepsilon,|y|<\varepsilon$时,$x-y\in K_{2\varepsilon}\subset K_\eta$,因而
\[|v(x-y)-v(x)|<\frac{\delta}{3m(K_\eta)}.\]
于是
\begin{align*}
	\max_{x\in K_\varepsilon}|v_\varepsilon(x)-v(x)|
	&\le\max_{x\in K_\varepsilon}\int_{\MR^N}|v(x-y)-v(x)|\chi_\varepsilon(y)\dif m(y)\\
	&=\max_{x\in K_\varepsilon}\int_{|y|\le\varepsilon}|v(x-y)-v(x)|\chi_\varepsilon(y)\dif m(y)<\frac{\delta}{3m(K_\eta)}.
\end{align*}
因为$\supp v=K$,由\hyperlink{6.1.1}{(2)}知$\supp v_\varepsilon\subset K_\varepsilon$,所以$\supp(v_\varepsilon-v)\subset K_\varepsilon$.于是
\begin{align}\label{eq6.1.3}
	\Vert v_\varepsilon-v\Vert_{L^2}
	&=\left\{\int_{\MR^N}|v_\varepsilon(x)-v(x)|^2\dif m(x)\right\}^{\frac12}\notag\\
	&=\left\{\int_{K_\varepsilon}|v_\varepsilon(x)-v(x)|^2\dif m(x)\right\}^{\frac12}<\frac\delta 3.
\end{align}
综合\eqref{eq6.1.1},\eqref{eq6.1.2},\eqref{eq6.1.3}即得
\[\Vert u_\varepsilon-u\Vert_{L^2}\le\Vert u_\varepsilon-v_\varepsilon\Vert_{L^2}+\Vert v_\varepsilon-v\Vert_{L^2}+\Vert v-u\Vert_{L^2}<\delta.\]
即当$\varepsilon\to0$时,$\Vert u_\varepsilon-u\Vert_{L^2}\to0$.
\end{proof}
注意,虽然上面的讨论是对$u\in L^2(\MR^N)$进行的,其实所有结论对$u\in L^p(\MR^N),1\le p<\infty$也都成立.
\subsection{$\MR^N$上的Sobolev空间}
设$G$是$\MR^N$中的开集,对于$f\in C^\infty(G)$及非负整数$s$,定义\index[symbolindex]{\textbf{其它符号}!$\Vert f\Vert_s^2$}
\[\Vert f\Vert_s^2=\sum_{|\alpha|\le s}\Vert\DD^\alpha f\Vert_{L^2}^2=\sum_{|\alpha|\le s}\int_G|(\DD^\alpha f)(x)|^2\dif m(x),\]
这里$\alpha=(\alpha_1,\cdots,\alpha_N)$是多重指标,$(\DD^\alpha f)(x)=\frac{\partial^{|\alpha|}f(x)}{\partial x_1^{\alpha_1}\cdots\partial x_N^{\alpha_N}}$.记\index[symbolindex]{\textbf{函数和映射}!$A_s(G)$}
\[A_s(G)=\{f\in C^\infty(G)\colon\Vert f\Vert_s<\infty\}.\]
对于$f,g\in A_s(G)$,定义它们的内积\index[symbolindex]{\textbf{其它符号}!$\langle f$,$g\rangle_s$}
\[\langle f,g\rangle_s=\sum_{|\alpha|\le s}\int_G (\DD^\alpha f)(x)\bar{(\DD^\alpha g)(x)}\dif m(x),\]
特别
\[\langle f,g\rangle_0=\int_G f(x)\bar{g(x)}\dif m(x)\]
就是普通$L^2(G)$中的内积,简记为$\langle f,g\rangle$.由Schwarz不等式
\[|\langle f,g\rangle_s|\le\Vert f\Vert_s \Vert g\Vert_s.\]

现设$\{f_j\}$是$A_s(G)$中的Cauchy序列,即当$j,k\to\infty$时,$\Vert f_j-f_k\Vert_s\to0$.按照$\Vert f\Vert_s$的定义,对于任意$|\alpha|\le s$,均有
\[\Vert\DD^\alpha f_j-\DD^\alpha f_k\Vert_{L^2}\to0,\quad j,k\to\infty.\]
特别当$\alpha=(0,\cdots,0)$时,有$\Vert f_j-f_k\Vert_{L^2}\to0$,即$\{f_j\}$是$L^2(G)$中一个Cauchy序列.由于$L^2(G)$是完备的,所以存在$f\in L^2(G)$,使得$\Vert f_j-f\Vert_{L^2}\to0,j\to\infty$.这个$f$是由$A_s(G)$中的Cauchy序列$\{f_j\}$所确定的.把所有这种$f$添加到$A_s(G)$上去,所得的空间称为$A_s(G)$按范数$\Vert\cdot\Vert_s$的完备化空间.
\begin{definition}\label{def6.1.2}
	$A_s(G)$按范数$\Vert\cdot\Vert_s$的完备化空间称为\textbf{Sobolev空间},记为$W^s(G)$\index[symbolindex]{\textbf{函数和映射}!$W^s(G)$}.
\end{definition}
为了弄清Sobolev空间$W^s(G)$的构造,我们引进弱导数的概念.

设$f\in C^\infty((a,b)),\varphi\in C_0^\infty((a,b))$,则由分部积分法
\[\int_{a}^{b}f(x)\bar{\varphi^{(m)}(x)}\dx=(-1)^m\int_{a}^{b}f^{(m)}(x)\bar{\varphi(x)}\dx,\]
即$\langle f,\varphi^{(m)}\rangle=(-1)^m\langle f^{(m)},\varphi\rangle$.根据这个简单的事实,我们给出下面的
\begin{definition}\label{def6.1.3}
	设$f\in L^2(G)$.如果存在$h\in L^2(G)$,使得对任意$\varphi\in C_0^\infty(G)$,总有
	\[\langle f,\DD^\alpha \varphi\rangle=(-1)^{|\alpha|}\langle h,\varphi\rangle,\]
	就称$h$为$f$的$\alpha$阶\textbf{弱导数}\index{R!弱导数},记为$\DD^\alpha f=h(\text{弱})$\index[symbolindex]{\textbf{导数}!$\DD^\alpha f(\text{弱})$}.这时称$f$的$\alpha$阶弱导数存在.
\end{definition}
显然,如果$f$的$\alpha$阶弱导数$h$存在,则在$L^2(G)$上$h$是唯一的,即若$\DD^\alpha f=h,\DD^\alpha f=g$,则$h$和$g$在$G$上几乎处处相等.如果$f$是$C^\infty$函数,那么它的弱导数就是普通的导数.这样就把导数的概念推广到$L^2(G)$函数上去了.

容易看出,弱导数的运算和普通导数是一样的,即若$f,g\in L^2(G)$,且弱导数$\DD^\alpha f,\DD^\alpha g$存在,那么对任意复数$a,b$有
\[\DD^\alpha(af+bg)=a\DD^\alpha f+b\DD^\alpha g.\]
又若弱导数$\pp{f}{x_j}$存在,$g\in C^\infty(G)$且$\pp{g}{x_j}$在$G$上有界,则
\[\pp{}{x_j}(fg)=\pp{g}{x_j}f+g\pp{f}{x_j}.\]
这些都可按弱导数的定义直接验证.

对于Sobolev空间来说,有下面的
\begin{prop}\label{prop6.1.4}
	如果$f\in W^s(G)$,则对任意$\alpha,|\alpha|\le s$,弱导数$\DD^\alpha f$存在.
\end{prop}
\begin{proof}
	根据$W^s(G)$的定义,存在$A_s(G)$中的Cauchy序列$\{f_j\}$,当$j\to\infty$时,$\Vert f_j-f\Vert_{L^2}\to0$.因而,当$j,k\to\infty$时,有$\Vert f_j-f_k\Vert_s\to0$.由此推出,对任意$\alpha,|\alpha|\le s$有
	\[\Vert\DD^\alpha f_j-\DD^\alpha f_k\Vert_{L^2}\to0,\quad j,k\to\infty.\]
	因为$L^2(G)$是完备的,所以存在$h\in L^2(G)$,使得当$j\to\infty$时,有$\Vert\DD^\alpha f_j-h\Vert_{L^2}\to0$.任取$\varphi\in C_0^\infty(G)$,则有
	\[\langle f,\DD^\alpha\varphi\rangle=\lim_{j\to\infty}\langle f_j,\DD^\alpha \varphi\rangle=(-1)^{|\alpha|}\lim_{j\to\infty}\langle \DD^\alpha f_j,\varphi\rangle=(-1)^{|\alpha|}\langle h,\varphi\rangle.\]
	此即$\DD^\alpha f=h(\text{弱})$.
\end{proof}
现在可以在$W^s(G)$中定义内积.

设$f,g\in W^s(G)$,定义它们的内积
\[\langle f,g\rangle_s=\sum_{|\alpha|\le s}\int_G(\DD^\alpha f)(x)\bar{(\DD^\alpha g)(x)}\dif m(x),\]
这里$\DD^\alpha f,\DD^\alpha g$都是弱导数.

如果$f,g\in A_s(G)$,那么上面的定义和原来的相一致.特别
\[\Vert f\Vert_s^2=\langle f,f\rangle_s=\sum_{|\alpha|\le s}\Vert\DD^\alpha f\Vert_{L^2}^2.\]

在这个内积下,$W^s(G)$是一个Hilbert空间.

下面证明命题\ref{prop6.1.4}的逆也是成立的,为此先证
\begin{prop}\label{prop6.1.5}
	设$G$是$\MR^N$中的开集,如果$x\in G,d(x,\partial G)>\varepsilon$,弱导数$\DD^\alpha f$存在,那么
	\[\DD^\alpha(f\ast\chi_\varepsilon)(x)=((\DD^\alpha f)\ast\chi_\varepsilon)(x),\]
	这里$\chi_\varepsilon$是定理\ref{thm6.1.1}中定义的函数.
\end{prop}
\begin{proof}
	因为$d(x,\partial G)>\varepsilon$,所以$B(x,\varepsilon)\subset G$.于是
	\begin{align*}
		\DD^\alpha(f\ast\chi_\varepsilon)(x)
		&=\DD^\alpha\left\{\int_G f(y)\chi_\varepsilon(x-y)\dif m(y)\right\}\\
		&=\int_G f(y)\DD_x^\alpha \chi_\varepsilon(x-y)\dif m(y)\\
		&=(-1)^{|\alpha|}\int_G f(y)\DD_y^\alpha \chi_\varepsilon(x-y)\dif m(y)\\
		&=(-1)^{|\alpha|}(-1)^{|\alpha|}\int_G \DD^\alpha f(y)\chi_\varepsilon(x-y)\dif m(y)\\
		&=((\DD^\alpha f)\ast\chi_\varepsilon)(x).\qedhere
	\end{align*}
\end{proof}
现在证明
\begin{theorem}\label{thm6.1.6}
	Sobolev空间$W^s(G)$可表示为\index{S!Sobolev空间}
	\[W^s(G)=\{f\in L^2(G)\colon\text{当$|\alpha|\le s$时,弱导数$\DD^\alpha f$存在}\}.\]
\end{theorem}
\begin{proof}
	若把满足上式右端条件的函数的全体记为$P_s(G)$,则命题\ref{prop6.1.4}已证$W^s(G)\subset P_s(G)$.现在要证$P_s(G)\subset W^s(G)$.任取$f\in P_s(G)$,则$f\in L^2(G)$,且对任意$\alpha,|\alpha|\le s$,弱导数$\DD^\alpha f$存在.我们要证明,存在$f_j\in A_s(G)$,使得当$j\to\infty$时,$\Vert f_j-f\Vert_s\to0$.任取有理点$a_j\in G$,则必存在充分小的$r_j$,使得$\bar{B}(a_j,r_j)\subset G$,当$a_j$走遍$G$中的有理点时,所得的球列$\{B(a_j,r_j)\}$覆盖了$G$.取从属于$\{B(a_j,r_j)\}$的单位分解$\{\eta_j\}$,则对任意$j,\supp(\eta_j f)\subset\subset G$.对任意$\alpha,|\alpha|\le s$及固定的$j$,弱导数$\DD^\alpha(\eta_j f)\in L^2(G)$.于是,由定理\ref{thm6.1.1}得知,当$\varepsilon\to0$时,有
	\[\Vert \DD^\alpha(\eta_j f)\ast\chi_\varepsilon-\DD^\alpha (\eta_j f)\Vert_{L^2}\to0.\]
	这就是说,对任意一列趋于$0$的正数$\delta_k$,存在$\varepsilon(j,k)>0$,使得当$|\alpha|\le s$时,有
	\[\Vert\DD^\alpha(\eta_j f)\ast\chi_{\varepsilon(j,k)}-\DD^\alpha(\eta_j f)\Vert_{L^2}<\frac{\delta_k}{2^j}.\]
	命$f_{\delta_k}=\sum_{j=1}^{\infty}(\eta_j f)\ast\chi_{\varepsilon(j,k)}$,显然$f_{\delta_k}\in C^\infty(G)\cap L^2(G)$.对每个$\alpha,|\alpha|\le s$,利用命题\ref{prop6.1.5},即得
	\begin{align*}
		\Vert\DD^\alpha f_{\delta_k}-\DD^\alpha f\Vert_{L^2}
		&=\left\Vert\DD^\alpha\left(\sum_{j=1}^{\infty}(\eta_j f)\ast\chi_{\varepsilon(j,k)}\right)-\DD^\alpha\left(\sum_{j=1}^{\infty}\eta_j f\right)\right\Vert_{L^2}\\
		&=\left\Vert\sum_{j=1}^{\infty}(\DD^\alpha(\eta_j f)\ast\chi_{\varepsilon(j,k)}-\DD^\alpha(\eta_j f))\right\Vert_{L^2}\\
		&\le\sum_{j=1}^{\infty}\Vert\DD^\alpha(\eta_j f)\ast\chi_{\varepsilon(j,k)}-\DD^\alpha(\eta_j f)\Vert_{L^2}\\
		&\le\sum_{j=1}^{\infty}\frac{\delta_k}{2^j}=\delta_k\to0.
	\end{align*}
因而有$\Vert f_{\delta_k}-f\Vert_s\to0,k\to\infty$.
\end{proof}
定理\ref{thm6.1.6}有时也被用来作为Sobolev空间$W^s(G)$的定义.这个定理使我们弄清了$W^s(G)$的构造,它是由$L^2(G)$中那些具有弱导数$\DD^\alpha f(|\alpha|\le s)$的函数$f$所构成.当然,具有弱导数的函数未必有普通的导数.但下面的Sobolev引理断言,具有高阶弱导数的函数必具有一定阶数的普通导数.
\begin{theorem}[(\textbf{Sobolev引理})]\label{thm6.1.7}\index{D!定理!Sobolev引理}
	设$G$是$\MR^N$中的开集,如果$s>\frac N2,k$是任意的非负整数,那么
	\[W^{s+k}(G)\subset C^k(G).\]
\end{theorem}
为此要先证下面的
\begin{lemma}[(\textbf{Sobolev不等式})]\label{lem6.1.8}\index{S!Sobolev不等式}
	设$G$是$\MR^N$中的开集,$K$是$G$中的紧集.如果$s>\frac N2$,则必存在常数$C>0$,使得对每一个$f\in A_s(G)$均有
	\[\max_{x\in K}|f(x)|\le C\Vert f\Vert_s.\]
\end{lemma}
\begin{proof}
	因为$K$是$G$中的紧集,故必存在$\rho>0$,使得对每一个$a\in K$都有$B(a,\rho)\subset G$.取定义在$[0,\infty)$中的函数$\eta$,使得$\eta\in C^\infty([0,\rho]),\supp\eta\subset[0,\rho)$,且当$r\in\left[0,\frac{\rho}{2}\right]$时,$\eta(r)\equiv1$.对固定的$a\in K$,通过$\eta$可定义$\MR^N$上的函数$\eta(|x-a|)$,仍用$\eta$简记之.任取$f\in A_s(G)$,并用极坐标,$f$可写为
	\[f(x_1,\cdots,x_N)=f(r\cos\theta_1,r\sin\theta_1\cos\theta_2,\cdots,r\sin\theta_1\cdots\sin\theta_{N-2}\sin\theta_{N-1}),\]
	在$\eta$的定义中取$a=0$,则$\eta(|x|)=\eta(r)$.由分部积分法可得
	\[\int_{0}^{\rho}r^{s-1}\frac{\partial^s(\eta f)}{\partial r^s}\dif r=(-1)^s (s-1)! f(0),\]
	或者
	\[f(0)=(-1)^s\frac1{(s-1)!}\int_{0}^{\rho}r^{s-1}\frac{\partial^s(\eta f)}{\partial r^s}\dif r.\]
	让上式两端在单位球面上积分,记单位球面的面积为$\sigma_0$,并注意到体积元素和面积元素的关系为$\dif m=r^{N-1}\dif r\dif\sigma$,就有
	\begin{align*}
		f(0)\sigma_0
		&=(-1)^s\frac1{(s-1)!}\int_{\partial B}\int_{0}^{\rho}r^{s-1}\frac{\partial^s(\eta f)}{\partial r^s}\dif r\dif\sigma\\
		&=(-1)^s\frac1{(s-1)!}\int_{\partial B}\int_{0}^{\rho}r^{s-N}\frac{\partial^s(\eta f)}{\partial r^s}r^{N-1}\dif r\dif\sigma\\
		&=(-1)^s\frac1{(s-1)!}\int_{B(0,\rho)}r^{s-N}\frac{\partial^s(\eta f)}{\partial r^s}\dif m.
	\end{align*}
用Schwarz不等式得
\begin{equation}\label{eq6.1.4}
	|f(0)\sigma_0|\le\frac1{(s-1)!}\left(\int_{B(0,\rho)}r^{2(s-N)}\dif m\right)^{\frac12}\left(\int_{B(0,\rho)}\left|\frac{\partial^s(\eta f)}{\partial r^s}\right|^2\dif m\right)^{\frac12}.
\end{equation}
由于$s>\frac N2$,所以$2(s-N)>-N$,右端第一个积分收敛.由于$B(0,\rho)\subset G$,右端第二个积分不超过
\begin{align*}
	\left(\int_G \left|\frac{\partial^s(\eta f)}{\partial r^s}\right|^2\dif m\right)^{\frac12}
	&\le\left(\sum_{|\alpha|\le s}\Vert\DD^\alpha(\eta f)\Vert_{L^2}^2\right)^{\frac12}\\
	&\le C\left(\sum_{|\alpha|\le s}\Vert\DD^\alpha f\Vert_{L^2}^2\right)^{\frac12}=C\Vert f\Vert_s.
\end{align*}
于是由\eqref{eq6.1.4}即得
\[|f(0)|\le C\Vert f\Vert_s,\]
这里$C$代表一个常数,它在不同的地方出现可以取不同的值.对于$K$中任意点$a$,取$\eta$为$\eta(|x-a|)$,所有的讨论都在$B(a,\rho)$上进行,即可得同样的结论.
\end{proof}
\begin{proof}[\textbf{定理\ref{thm6.1.7}的证明}]
	设$f\in W^{s+k}(G)$,我们要证明$f$在$G$上几乎处处等于$C^k(G)$中的一个函数.根据$W^{s+k}(G)$的定义,存在一列$\{f_j\}\subset A_{s+k}(G)$,使得当$j\to\infty$时,$\Vert f_j-f\Vert_{L^2}\to0$.因而当$i\to\infty,j\to\infty$时,有$\Vert f_i-f_j\Vert_{s+k}\to0$.由于$f_i-f_j\in A_{s+k}(G)$,所以对任意$\alpha,|\alpha|\le k,\DD^\alpha (f_i-f_j)\in A_s(G)$.于是,由引理\ref{lem6.1.8},对任意紧集$K\subset G$,有
	\begin{align*}
		\max_{x\in K}|\DD^\alpha f_i(x)-\DD^\alpha f_j(x)|
		&\le C\Vert\DD^\alpha f_i-\DD^\alpha f_j\Vert_s\\
		&\le C\Vert f_i-f_j\Vert_{s+k}\to0.
	\end{align*}
这就证明了$\{\DD^\alpha f_j\}$在$K$上一致收敛.特别取$\alpha=(0,\cdots,0)$,就有$\{f_j\}$在$K$上一致收敛.设其极限函数为$g$,由于$\{\DD^\alpha f_j\}$也在$K$上一致收敛,因而有$\DD^\alpha f_j\to\DD^\alpha g$,所以$g\in C^k(K)$.这时当然有$\Vert f_j-g\Vert_{L^2(K)}\to0(j\to\infty)$.另外从$\Vert f_j-f\Vert_{L^2}\to0$,当然也有$\Vert f_j-f\Vert_{L^2(K)}\to0$,所以$f$在$K$上几乎处处等于$g$,因为$K$是$G$中任意的紧集,所以$f$在$G$上几乎处处等于$g$.
\end{proof}
Sobolev引理是研究偏微分方程解的正则性的重要工具,它在偏微分方程的近代研究中十分重要.
\section{把$\bar{\partial}$问题归结为$L^2$估计\label{sec6.2}}
\subsection{加权的平方可积空间}
在\ref{sec4.5}中我们曾经提到过,$\MC^n$中$(p,q)$形式的一般形状为
\[f=\sum_{|I|=p}\sum_{|J|=p}f_{IJ}(z)\dz_I\wedge\dzz_J,\]
这里$I=(i_1,\cdots,i_p),J=(j_1,\cdots,j_q),\dz_I=\dz_{i_1}\wedge\cdots\wedge\dz_{i_p},\dzz_J=\dzz_{j_1}\wedge\cdots\wedge\dzz_{j_q},1\le i_1<\cdots<i_p\le n,1\le j_1<\cdots<j_q\le n$.$\bar{\partial}$算子把$(p,q)$形式映为$(p,q+1)$形式.所谓$\bar{\partial}$问题是指给定一个$(p,q+1)$形式$f$,要找一个$(p,q)$形式$u$,使得$\bar{\partial}u=f$成立.由于$\bar{\partial}^2=0$,所以上述问题有解的必要条件是$\bar{\partial}f=0$.

当$p=q=0$时,$\bar{\partial}$问题可叙述为:给定一个$(0,1)$形式$f=\sum_{j=1}^{n}f_j(z)\dzz_j$满足$\bar{\partial}f=0$,要找一个函数$u$,使得$\bar{\partial}u=f$.在\ref{sec4.7}中,我们利用平面上非齐次的Cauchy积分公式,已经讨论过平面上的$\bar{\partial}$问题,以及高维的具有紧支集的$\bar{\partial}$问题.

下面我们将从Hilbert空间的角度来考察一般的$\bar{\partial}$问题,最后把求解问题归结为证明一个不等式.
\begin{definition}\label{def6.2.1}
	设$G$是$\MC^n$中的开集,$f\colon G\to\MC$是可测函数,$\varphi\colon G\to\MR$是连续函数,定义\index[symbolindex]{\textbf{函数和映射}!$L^2(G$,$\varphi)$}
	\[L^2(G,\varphi)=\left\{f\colon\int_G |f|^2\ee^{-\varphi}\dif m<\infty\right\},\]
	这里$\dif m$是$G$中的Lebesgue测度,$\varphi$称为\textbf{权函数}\index{Q!权函数},$L^2(G,\varphi)$称为\textbf{加权的平方可积空间}\index{J!加权的平方可积空间}.
\end{definition}
设$f,g\in L^2(G,\varphi)$,定义它们的内积为\index[symbolindex]{\textbf{其它符号}!$\langle f$,$g\rangle_\varphi$}\index[symbolindex]{\textbf{其它符号}!$\Vert f\Vert_\varphi^2$}
\[\langle f,g\rangle_\varphi=\int_G f\bar{g}\ee^{-\varphi}\dif m,\]
\[\Vert f\Vert_\varphi^2=\langle f,f\rangle_\varphi=\int_G |f|^2\ee^{-\varphi}\dif m,\]
则$L^2(G,\varphi)$是一个Hilbert空间.同样定义\index[symbolindex]{\textbf{微分形式}!$L_{\text{$(p$,$q)$}}^2(G$,$\varphi)$}
\[L_{(p,q)}^2(G,\varphi)=\left\{f\colon\text{$f$是$(p,q)$形式,它的系数$f_{IJ}\in L^2(G,\varphi)$}\right\}.\]

设$f,g\in L_{(p,q)}^2(G,\varphi)$,定义它们的内积为
\[\langle f,g\rangle_\varphi=\sum_{I,J}\int_G f_{IJ}\bar{g}_{IJ}\ee^{-\varphi}\dif m,\]
\[\Vert f\Vert_\varphi^2=\langle f,f\rangle_\varphi=\sum_{I,J}\int_G |f_{IJ}|^2\ee^{-\varphi}\dif m,\]
记$|f|^2=\sum_{I,J}|f_{IJ}|^2$,则$\Vert f\Vert_\varphi^2=\int_G |f|^2\ee^{-\varphi}\dif m$.这时$L_{(p,q)}^2(G,\varphi)$也成一个Hilbert空间.

设$f=\sum_{I,J}f_{IJ}\dz_I\wedge\dzz_J\in L_{(p,q)}^2(G,\varphi)$,则
\[\bar{\partial}f=\sum_{I,J}\sum_{k=1}^{n}\pp{f_{IJ}}{\bar{z}_k}\dzz_k\wedge\dz_I\wedge\dzz_J.\]
由于$f_{IJ}$不是可微函数,这里的$\pp{f_{IJ}}{\bar{z}_k}$是$f_{IJ}$的弱导数,这时仍有$\bar{\partial}f\in L_{(p,q+1)}^2(G,\varphi)$,但$\bar{\partial}f$未必属于$L_{(p,q+1)}^2(G,\varphi_1)$,$\varphi_1$是另一个权函数.

我们把扩大定义后的$\bar{\partial}$算子记为$T$或$S,T$的定义域$D(T)$是指\index[symbolindex]{\textbf{微分形式}!$D(T)$}
\[D(T)=\left\{f\in L_{(p,q)}^2(G,\varphi_1)\colon Tf\in L_{(p,q+1)}^2(G,\varphi_2)\right\}.\]
$T$和$S$的作用如下:
\[L_{(p,q)}^2(G,\varphi_1) \overset{T}{\longrightarrow} L_{(p,q+1)}^2(G,\varphi_2)\overset{S}{\longrightarrow} L_{(p,q+2)}^2(G,\varphi_3).\]
\subsection{$\bar{\partial}$算子的基本性质}
为了把$\bar{\partial}$问题抽象成一般的Hilbert空间中的问题,注意算子$T=\bar{\partial}$的下列性质:

(1)
$T$是线性算子;

(2)
$D(T)$在$L_{(p,q)}^2(G,\varphi_1)$中稠;

用$D_{(p,q)}(G)$记$G$中的$(p,q)$形式$f$\index[symbolindex]{\textbf{微分形式}!$D_{\text{$(p$,$q)$}}(G)$},其系数$f_{IJ}\in C_0^\infty(G)$,即在$G$中有紧支集的无穷可微函数.显然,$D_{(p,q)}(G)\subset D(T)$.对于任意$f\in L_{(p,q)}^2(G,\varphi_1)$可取具有紧支集的连续函数$u_{IJ}$,使得$\Vert f_{IJ}-u_{IJ}\Vert_{L^2}<\varepsilon$.因为$u_{IJ}$有紧支集,故由定理\ref{thm6.1.1}\hyperlink{6.1.1}{(2)},存在$(u_{IJ})_\varepsilon\in C_0^\infty(G)$,使得$\Vert (u_{IJ})_\varepsilon-u_{IJ}\Vert_{L^2}\to0$,当$\varepsilon\to0$.于是得$\Vert f_{IJ}-(u_{IJ})_\varepsilon\Vert_{L^2}\to0$,这说明$D_{(p,q)}(G)$在$L_{(p,q)}^2(G,\varphi_1)$中稠,因而$D(T)$在$L_{(p,q)}^2(G,\varphi_1)$中稠.

(3)
$T$是闭算子.

设$u_k\in D(T),u_k\to u,Tu_k\to v$,要证明
\[u\in D(T),\quad\text{且$v=Tu$}.\]
事实上,因为$u_k\to u$,所以$u\in L_{(p,q)}^2(G,\varphi_1)$.由于$T$在弱导数的意义下是连续的,所以$Tu_k\to Tu$,因而$Tu\in L_{(p,q+1)}^2(G,\varphi_2)$,所以$u\in D(T)$,且$v=Tu$.
\subsection{$\bar{\partial}$问题的Hilbert空间模型}
知道了$T$的这些性质后,便可把$\bar{\partial}$问题抽象为一般的Hilbert空间模型.

设$H_1,H_2,H_3$是三个Hilbert空间,$T,S$是两个线性算子,它们有关系
\[H\overset{T}{\longrightarrow}H_2\overset{S}{\longrightarrow}H_3.\]
$T,S$还满足下述条件:

(1)
$D(T),D(S)$分别在$H_1,H_2$中稠\index[symbolindex]{\textbf{微分形式}!$D(S)$};

(2)
$T,S$是线性的闭算子;

(3)
$TS=0$(这是因为$\bar{\partial}^2=0$).

如果$Sf=0$,问$Tu=f$在$H_1$中是否有解?

容易看出,$Tu=f$在$H_1$中有解的充分必要条件是
\begin{equation}\label{eq6.2.1}
	\langle Tu,g\rangle_{H_2}=\langle f,g\rangle_{H_2}
\end{equation}
对$H_2$中某一稠密子集中的任意$g$成立.

在什么条件下才能使\eqref{eq6.2.1}成立?让我们回忆一下$T$的伴随算子$T^\ast$的定义.\index[symbolindex]{\textbf{其它符号}!$T^\ast$}\index{B!伴随算子}

因为$T$的定义域$D(T)$在全空间$H_1$是稠的,即$T\colon H_1\to H_2$是稠定算子\index{C!稠定算子}.今定义\index[symbolindex]{\textbf{微分形式}!$D(T^\ast)$}
\[D(T^\ast)=\left\{y\colon y\in H_2,\exists y^\ast \in H_1,\forall x\in D(T),\text{使$\langle Tx,y\rangle=\langle x,y^\ast\rangle$成立}\right\}.\]
不难证明,对于给定的$y\in D(T^\ast)$,对应的$y^\ast$是唯一的.这就确定了算子$T^\ast\colon y\to y^\ast$,称$T^\ast$为$T$的伴随算子.因而对任意的$x\in D(T),y\in D(T^\ast)$,有
\[\langle Tx,y\rangle=\langle x,T^\ast y\rangle.\]
我们知道$T^\ast$有下列性质:

(1)
稠定线性算子$T$的伴随算子$T^\ast$是线性算子;

(2)
稠定线性算子$T$的伴随算子$T^\ast$是闭算子;

(3)
如果$T$是闭算子,那么$(T^\ast)^\ast=T$.

容易知道,$T^\ast$对$D_{(p,q+1)}(G)$中的元素都有定义,即$D_{(p,q+1)}(G)\subset D(T^\ast)$,因而

(4)
$D(T^\ast)$在$H_2$中稠.

今设$E$是$H_2$中的稠子集,\eqref{eq6.2.1}对每个$g\in E$成立.如果$E\subset D(T^\ast)$,则因$\langle Tu,g\rangle=\langle u,T^\ast g\rangle$,\eqref{eq6.2.1}可改写为
\begin{equation}\label{eq6.2.2}
	\langle u,T^\ast g\rangle=\langle f,g\rangle.
\end{equation}
记
\[H_1'=\left\{T^\ast g\colon g\in E\subset D(T^\ast)\right\},\]
则$H_1'\subset H_1$.定义$h\colon H_1'\to \MC$如下:
\[h(T^\ast g)=\langle g,f\rangle_{H_2}.\]
$h$是定义在$H_1'$上的一个线性泛函.如果能把$h$扩展成定义在整个空间$H_1$上的一个有界线性泛函,那么根据Riesz表示定理,存在$u\in H_1$,使得$h(T^\ast g)=\langle T^\ast g,u\rangle$,即$\langle T^\ast g,u\rangle=\langle g,f\rangle$,这就是\eqref{eq6.2.2}.因而$u$就是$Tu=f$的解.

在什么条件下,$h$能扩展成全空间$H_1$上的有界线性泛函?
\begin{prop}\label{prop6.2.2}
	如果存在只依赖于$f$的常数$C_f$,使得
	\begin{equation}\label{eq6.2.3}
		|\langle g,f\rangle|\le C_f\Vert T^\ast g\Vert,g\in E\subset D(T^\ast),
	\end{equation}
则$h(T^\ast g)=\langle g,f\rangle$可以扩展成$H_1$上的有界线性泛函.
\end{prop}
\begin{proof}
	先证在条件\eqref{eq6.2.3}下,泛函$h(T^\ast g)=\langle g,f\rangle$的定义有意义.因为如果$T^\ast g_1=T^\ast g_2$,那么
	\[|\langle g_1-g_2,f\rangle|\le C_f\Vert T^\ast(g_1-g_2)\Vert=0,\]
	因而$\langle g_1,f\rangle=\langle g_2,f\rangle$.现在先把$h$扩充到$\bar{H}_1'$上.设$x\in\bar{H}_1'$,则存在$T^\ast g_l\in H_1'$,使得$T^\ast g_l\to x,l\to\infty$.于是由\eqref{eq6.2.3}得
	\[|\langle g_l-g_m,f\rangle|\le C_f\Vert T^\ast g_l-T^\ast g_m\Vert\to0,l,m\to\infty.\]
	这说明$\{\langle g_l,f\rangle\}$是$\MC$中的Cauchy序列.今定义
	\[h(x)=\lim_{l\to\infty}\langle g_l,f\rangle.\]
	因为$\bar{H}_1'$是$H_1$的一个闭子空间,命$P$是$H_1$到$\bar{H}_1'$上的投影算子,即对任意$x\in H_1$,存在唯一的$x_0\in\bar{H}_1'$,使得$Px=x_0$.今定义$h(x)=h(x_0)$.这样$h$就在整个空间$H_1$中都有了定义.下面证明$h$是$H_1$上的有界线性泛函.任取$x,y\in H_1$,它们唯一地对应着$x_0,y_0\in\bar{H}_1'$.如果$x_0,y_0\in\bar{H}_1'$,那么
	\begin{align*}
		h(\alpha x+\beta y)
		&=h(\alpha x_0+\beta y_0)=h(\alpha T^\ast g_1+\beta T^\ast g_2)\\
		&=h(T^\ast(\alpha g_1+\beta g_2))=\langle\alpha g_1+\beta g_2,f\rangle\\
		&=\alpha\langle g_1,f\rangle+\beta\langle g_2,f\rangle\\
		&=\alpha h(T^\ast g_1)+\beta h(T^\ast g_2)\\
		&=\alpha h(x_0)+\beta h(y_0)=\alpha h(x)+\beta h(y).
	\end{align*}
属于闭包的情况可通过极限证明之,所以$h$是线性的.

有界性的证明也很容易.对于任意$x\in H_1$,
\begin{align*}
	|h(x)|
	&=|h(x_0)|=|h(T^\ast g)|\\
	&=|\langle g,f\rangle|\le C_f\Vert T^\ast g\Vert=C_f\Vert x_0\Vert\\
	&\le C_f\Vert x\Vert.
\end{align*}
$x_0$属于闭包的情况可通过极限来证明.
\end{proof}
现在问,在什么条件下,\eqref{eq6.2.3}能成立?注意,上面的讨论并没有对$f$加以限制.实际上,只要对满足$Sf=0$的$f$讨论就行了.记\index{S!算子的零空间}\index{S!算子的像空间}
\[N(S)=\left\{x\in H_2\colon Sx=0\right\},\quad R(T)=\left\{Tx\colon x\in D(T)\right\}.\]
称$N(S)$为算子$S$的零空间,$R(T)$为算子$T$的像空间.$N(S),R(T)$和$N(T^\ast)$间有下述包含关系.\index[symbolindex]{\textbf{微分形式}!$N(S)$,$N(S)^\perp$}\index[symbolindex]{\textbf{微分形式}!$R(T)$,$R(T)^\perp$}\index[symbolindex]{\textbf{微分形式}!$N(T^\ast)$}
\begin{lemma}\label{lem6.2.3}
	$N(S)^\perp \subset R(T)^\perp \subset N(T^\ast)$.
\end{lemma}
\begin{proof}
	任取$x\in N(S)^\perp,y\in R(T)$,则必存在$h\in D(T)$,使得$y=Th$.由于$ST=0$,所以$Sy=STh=0$,即$y\in N(S)$,因而$x\perp y$,这说明$x\in R(T)^\perp$.这就证明了$N(S)^\perp \subset R(T)^\perp$.再取$x\in R(T)^\perp$,则对任意$y\in R(T)$,有$\langle x,y\rangle=0$.而$y=Th,h\in D(T)$,因而$\langle x,Th\rangle=0$,即$\langle T^\ast x,h\rangle=0$.由于$y$是任意的,因而$h$也是任意的,所以$T^\ast x=0$,即$x\in N(T^\ast)$,因而$R(T)^\perp \subset N(T^\ast)$.
\end{proof}
现在可以证明
\begin{prop}\label{prop6.2.4}
	设$D(T^\ast)\cap D(S)$在$H_2$中稠.如果存在常数$C$,使得对每个$g\in D(T^\ast)\cap D(S)$,有
	\begin{equation}\label{eq6.2.4}
		\Vert g\Vert^2\le C\left(\Vert T^\ast g\Vert^2+\Vert Sg\Vert^2\right),
	\end{equation}
那么
\begin{equation}\label{eq6.2.5}
	|\langle g,f\rangle|\le\sqrt{C}\Vert f\Vert\,\Vert T^\ast g\Vert
\end{equation}
对每个$f\in N(S)$和$g\in D(T^\ast)\cap D(S)$成立.
\end{prop}
\begin{proof}
	任取$g\in D(T^\ast)\cap D(S)$.把$g$对闭子空间$N(S)$作正交分解:$g=g_1+g_2$,这里$g_1\in N(S),g_2\in N(S)^\perp$.我们证明$g_1,g_2$都在$D(T^\ast)\cap D(S)$中.事实上,因为$g_1\in N(S)$,所以$Sg_1=0$,因而$g_1\in D(S)$.其次,由引理\ref{lem6.2.3},从$g_2\in N(S)^\perp$得知$g_2\in N(T^\ast)$,即$T^\ast g_2=0$,因而$g_2\in D(T^\ast)$.由于$g\in D(T^\ast)\cap D(S)$,所以$g_1=g-g_2\in D(T^\ast)$,因而$g_1\in D(T^\ast)\cap D(S)$.于是$g_2=g-g_1\in D(T^\ast)\cap D(S)$.这样
	\begin{align*}
		|\langle g,f\rangle|
		&=|\langle g_1+g_2,f\rangle|=|\langle g_1,f\rangle+\langle g_2,f\rangle|\\
		&=|\langle g_1,f\rangle|\le\Vert f\Vert\,\Vert g_1\Vert\\
		&\le\sqrt{C}\Vert f\Vert\left(\Vert T^\ast g_1\Vert^2+\Vert Sg_1\Vert^2\right)^{\frac12}\\
		&=\sqrt{C}\Vert f\Vert\,\Vert T^\ast g_1\Vert\\
		&=\sqrt{C}\Vert f\Vert\,\Vert T^\ast(g-g_2)\Vert=\sqrt{C}\Vert f\Vert\,\Vert T^\ast g\Vert.\qedhere
	\end{align*}
\end{proof}
这样一来,全部问题归结为,在什么条件下,不等式\eqref{eq6.2.4}成立.上面的结论可总结为下面的
\begin{theorem}\label{thm6.2.5}
	设$D(T^\ast)\cap D(S)$在$H_2$中稠.如果存在常数$C$,使得对每个$g\in D(T^\ast)\cap D(S)$,有
	\begin{equation}
		\Vert g\Vert^2\le C\left(\Vert T^\ast g\Vert^2+\Vert Sg\Vert^2\right),\tag{\ref{eq6.2.4}}
	\end{equation}
	那么$Tu=f$对$f\in N(S)$有解,且解$u$满足
	
	(1)\hypertarget{6.2.5}{}
	$\Vert u\Vert\le\sqrt{C}\Vert f\Vert$;
	
	(2)\hypertarget{6.2.5}{}
	存在$v\in D(T^\ast)$,使得$u=T^\ast v$;
	
	(3)\hypertarget{6.2.5}{}
	$u\in N(T)^\perp$.\\
	\eqref{eq6.2.4}称为基本不等式\index{J!基本不等式}.
\end{theorem}
\begin{proof}
	$Tu=f$对$f\in N(S)$有解这一事实已在上面证明.现在证明解$u$满足上面三条性质.因为线性泛函$h$的界是$\sqrt{C}\Vert f\Vert$,故由Riesz表示定理$\Vert u\Vert\le\sqrt{C}\Vert f\Vert$,这就是\hyperlink{6.2.5}{(1)}.为了证明\hyperlink{6.2.5}{(2)},我们先证明存在$\{g_k\}\subset D(T^\ast)$,使得
	\begin{equation}\label{eq6.2.6}
		u=\lim_{k\to\infty} T^\ast g_k.
	\end{equation}
因为按照泛函$h$的扩展过程,对一般的$x\in H_1$,对$\bar{H}_1'$作正交分解$x=x_0+x_1$,其中$x_0\in \bar{H}_1',x_1\in\left(\bar{H}_1'\right)^\perp$.然后定义$h(x)=h(x_0)$.现在让$x\in\left(\bar{H}_1'\right)^\perp$,这时$x_0=0$.于是
\[h(x)=h(x_0)=h(0)=0=\langle x,u\rangle,\]
这说明$u\in\bar{H}_1'$,因而有$u=\lim_{k\to\infty} T^\ast g_k$.其次我们再证明
\begin{equation}\label{eq6.2.7}
	T^\ast(D(T^\ast))=T^\ast(N(S)\cap D(T^\ast)).
\end{equation}
右端包含在左端内是显然的.今取$g\in D(T^\ast)$,把$g$对$N(S)$作正交分解:$g=g_1+g_2,g_1\in N(S),g_2\in N(S)^\perp$.由引理\ref{lem6.2.3}$N(S)^\perp\subset N(T^\ast)$,所以$T^\ast g=T^\ast g_1$,故\eqref{eq6.2.7}成立.由于\eqref{eq6.2.7},不妨设\eqref{eq6.2.6}中的$g_k\in N(S)\cap D(T^\ast)$.在命题\ref{prop6.2.4}的不等式\eqref{eq6.2.5}中命$f=g=g_k$,可得$\Vert g_k\Vert\le\sqrt{C}\Vert T^\ast g_k\Vert$.因为$\{T^\ast g_k\}$收敛,所以
\[\Vert g_k-g_l\Vert\le\sqrt{C}\Vert T^\ast g_k-T^\ast g_l\Vert\to0,\quad k,l\to\infty.\]
即$\{g_k\}$是一个Cauchy序列.设$g_k\to v$,由于$T^\ast$是闭算子,所以$v\in D(T^\ast)$,且$u=T^\ast v$,这就证明了\hyperlink{6.2.5}{(2)}.\hyperlink{6.2.5}{(3)}的证明很简单,任取$x\in N(T)$,由\hyperlink{6.2.5}{(2)}即得
\[\langle x,u\rangle=\langle x,T^\ast v\rangle=\langle Tx,v\rangle=0.\]
所以$u\in N(T)^\perp$.
\end{proof}
这个定理说明,当基本不等式\eqref{eq6.2.4}满足时,$\bar{\partial}u=f$对$f\in N(S)$有解,而且解$u$具有定理\ref{thm6.2.5}中的性质.注意,条件\eqref{eq6.2.4}是$\bar{\partial}$方程有解的充分条件,并不必要,即若\eqref{eq6.2.4}不成立,$\bar{\partial}$方程也可能有解,但这时解$u$就不一定具有定理\ref{thm6.2.5}中的性质了.以后我们将看到,具有定理\ref{thm6.2.5}\hyperlink{6.2.5}{(2)}的解有良好的性质.为了区别于其它的解,我们给出下面的
\begin{definition}\label{def6.2.6}
	设$u$是方程$\bar{\partial}u=f$的一个解,如果存在$v\in D(T^\ast)$,使得$u=T^\ast v$,就称$u$为方程的一个\textbf{标准解}或\textbf{Kohn解}.
\end{definition}
定理\ref{thm6.2.5}断言,当基本不等式\eqref{eq6.2.4}成立时,$\bar{\partial}u=f$对$f\in N(S)$一定有标准解.
\section{$\bar{\partial}$问题解的存在性定理\label{sec6.3}}
现在回到我们的问题,这时
\[H_1=L_{(p,q)}^2(G,\varphi_1),H_2=L_{(p,q+1)}^2(G,\varphi_2),H_3=L_{(p,q+2)}^2(G,\varphi_3),\]
这里$\varphi_1,\varphi_2,\varphi_3$是三个权函数,它们都是实值连续函数.

我们现在要证明:如果$G$是$\MC^n$中的拟凸域,那么对任意$g\in D(T^\ast)\cap D(S)$,有
\begin{equation}\label{eq6.3.1}
	\Vert g\Vert_{\varphi_2}^2\le C\left(\Vert T^\ast g\Vert_{\varphi_1}^2+\Vert Sg\Vert_{\varphi_3}^2\right).
\end{equation}
证明的思路是这样的:

(1)\hypertarget{6.3}{}
先在一定的条件下,证明对任意$g\in D(T^\ast)\cap D(S)$必有$h_k\in D_{(p,q+1)}(G)$,使得在相应的范数下,有
\[h_k\to g,T^\ast h_k\to T^\ast g,Sh_k\to Sg.\]
因而只要对$g\in D_{(p,q+1)}(G)$证明不等式\eqref{eq6.3.1}就行了.

(2)\hypertarget{6.3}{}
适当选取$\varphi_1,\varphi_2,\varphi_3$,使\hyperlink{6.3}{(1)}中需要的条件得到满足.然后再在一定的条件下,对$g\in D_{(p,q+1)}(G)$证明基本不等式\eqref{eq6.3.1}成立.

(3)
对$\MC^n$中的拟凸域,证明\hyperlink{6.3}{(2)}中需要的条件是成立的,因而基本不等式\eqref{eq6.3.1}对任意$g\in D(T^\ast)\cap D(S)$成立.
\subsection{用$D_{(p,q+1)}(G)$中的元素逼近$D(T^\ast)\cap D(S)$中的元素}
先证明下面的
\begin{lemma}\label{lem6.3.1}
	设$G$是$\MC^n$中的域,则必存在满足下面三个条件的函数列$\{\eta_j\}$:
	
	(1)\hypertarget{6.3.1}{}
	$\eta_j\in C_0^\infty(G)$;
	
	(2)\hypertarget{6.3.1}{}
	对任意$z\in G,0\le\eta_j(z)\le1$;
	
	(3)\hypertarget{6.3.1}{}
	对于$G$的任一紧集$K$,存在正整数$j_0$,当$j>j_0$时,$\eta_j(z)\equiv1$在$K$上成立.
\end{lemma}
\begin{proof}
	取$G$的一个正规穷竭$\{K_j\}$(见引理\ref{lem4.6.4}),命$g_j$是$K_j$上的特征函数.设$\delta_j=d(K_j,\partial G)$,取$\varepsilon_j=\frac12\delta_j$,作
	\[\eta_j(x)=(g_j\ast\chi_{\varepsilon_j})(x),\]
	则由定理\ref{thm6.1.1}知道$\eta_j$在$G$中有紧支集,且无穷次可微,因而$\eta_j\in C_0^\infty(G)$.显然$0\le \eta_j\le1$.对于任意紧集$K\subset G$,有充分大的$j$,使得$K\subset K'\subset K_j$,其中$K'=\{y\colon |x-y|\le\varepsilon_j,x\in K\}$.于是,当$x\in K$时,便有
	\begin{align*}
		\eta_j(x)
		&=\int_{\MR^{2n}}g_j(y)\chi_{\varepsilon_j}(x-y)\dif m(y)=\int_{K'}g_j(y)\chi_{\varepsilon_j}(x-y)\dif m(y)\\
		&=\int_{K'}\chi_{\varepsilon_j}(x-y)\dif m(y)=\int_{\MR^{2n}}\chi_{\varepsilon_j}(x-y)\dif m(y)=1.
	\end{align*}
因此$\{\eta_j\}$就是所求的序列.
\end{proof}
\begin{lemma}\label{lem6.3.2}
	设$G$是$\MC^n$中的域,$\eta\in C_0^\infty(G)$,且满足
	\begin{equation}\label{eq6.3.2}
		\ee^{-\varphi_{j+1}}\sum_{l=1}^{n}\left|\pp{\eta}{\bar{z}_l}\right|^2\le\ee^{-\varphi_j},\quad j=1,2,
	\end{equation}
那么

(1)\hypertarget{6.3.2}{}
对于$f=\sum_{I,J}f_{IJ}\dz_I\wedge\dzz_J$,有
\begin{equation}\label{eq6.3.3}
	\left|\bar{\partial}(\eta f)-\eta\bar{\partial}f\right|^2\ee^{-\varphi_{j+1}}\le|f|^2\ee^{-\varphi_j},j=1,2;
\end{equation}

(2)\hypertarget{6.3.2}{}
对如果$f\in D(T^\ast)$,那么$\eta f\in D(T^\ast)$.
\end{lemma}
\begin{proof}
	\hyperlink{6.3.2}{(1)}
	因为$\bar{\partial}\eta=\sum_{l=1}^{n}\pp{\eta}{\bar{z}_l}\dzz_l,\bar{\partial}(\eta f)=\bar{\partial}\eta\wedge f+\eta\bar{\partial}f$,所以
	\begin{align*}
		\left|\bar{\partial}(\eta f)-\eta\bar{\partial}f\right|
		&=\left|\bar{\partial}\eta\wedge f\right|^2=\left|\sum_{I,J}\sum_{l=1}^n \pp{\eta}{\bar{z}_l}f_{IJ}\dzz_l\wedge\dz_I\wedge\dzz_J\right|^2\\
		&=\sum_{I,J}\sum_{l=1}^{n}\left|\pp{\eta}{\bar{z}_l}\right|^2\left|f_{IJ}\right|^2.
	\end{align*}
于是由\eqref{eq6.3.2}即得
\[\left|\bar{\partial}(\eta f)-\eta\bar{\partial}f\right|^2\ee^{-\varphi_{j+1}}=\sum_{I,J}\left|f_{IJ}\right|^2\sum_{l=1}^{n}\left|\pp{\eta}{\bar{z}_l}\right|^2\ee^{-\varphi_{j+1}}\le\left|f\right|^2\ee^{-\varphi_j}.\]
	
	\hyperlink{6.3.2}{(2)}
	要证明$\eta f\in D(T^\ast)$,就是要证明对于任意$u\in D(T)$,存在$v\in L_{(p,q)}^2(G,\varphi_1)$,使得$\langle Tu,\eta f\rangle_{\varphi_2}=\langle u,v\rangle_{\varphi_1}$.为此只要证明$\langle Tu,\eta f\rangle_{\varphi_2}$是$u$的有界线性泛函就行.事实上,
	\begin{align*}
		\langle Tu,\eta f\rangle_{\varphi_2}
		&=\langle\bar{\eta}Tu,f\rangle_{\varphi_2}=\langle T(\bar{\eta}u),f\rangle_{\varphi_2}+\langle\bar{\eta}Tu-T(\bar{\eta}u),f\rangle_{\varphi_2}\\
		&=\langle u,\eta T^\ast f\rangle_{\varphi_1}+\langle\bar{\eta}Tu-T(\bar{\eta}u),f\rangle_{\varphi_2},
	\end{align*}
和式的第一项由Schwarz不等式得
\[|\langle u,\eta T^\ast f\rangle_{\varphi_1}|\le\Vert \eta T^\ast f\Vert_{\varphi_1}\Vert u\Vert_{\varphi_1}.\]
第二项由Schwarz不等式以及\hyperlink{6.3.2}{(1)}可得
\[\left|\langle\bar{\eta}Tu-T(\bar{\eta}u),f\rangle_{\varphi_2}\right|\le\Vert f\Vert_{\varphi_2}\Vert\bar{\eta}Tu-T(\bar{\eta}u)\Vert_{\varphi_2}\le\Vert f\Vert_{\varphi_2}\Vert u\Vert_{\varphi_1}.\]
由此即得$\langle Tu,\eta f\rangle_{\varphi_2}$是$L_{(p,q)}^2(G,\varphi_1)$上的有界线性泛函.由Riesz表示定理,存在$v\in L_{(p,q)}^2(G,\varphi_1)$,使得$\langle Tu,\eta f\rangle_{\varphi_2}=\langle u,v\rangle_{\varphi_1}$,因而$\eta f\in D(T^\ast)$,且$v=T^\ast(\eta f)$.
\end{proof}
在证明过程中,需要用到伴随算子$T^\ast$的表达式.
\begin{lemma}\label{lem6.3.3}
	设$f=\sum_{I,J}f_{IJ}\dz_i\wedge\dzz_J\in D(T^\ast)$,这里$|I|=p,|J|=q+1$,那么
	\begin{equation}\label{eq6.3.4}
		T^\ast f=(-1)^{p-1}\sum_{I,K}\sum_{j=1}^{n}\ee^{\varphi_1}\pp{}{z_j}\left(\ee^{-\varphi_2}f_{I,jK}\right)\dz_I\wedge\dzz_K
	\end{equation}
或
\begin{equation}\label{eq6.3.5}
	\ee^{\varphi_2-\varphi_1}T^\ast f=(-1)^{p-1}\sum_{I,K}\sum_{j=1}^{n}\left(\pp{f_{I,jK}}{z_j}-f_{I,jK}\pp{\varphi_2}{z_j}\right)\dz_I\wedge\dzz_K,
\end{equation}
这里$|K|=q,\varphi_2\in C^1(G)$.
\end{lemma}
\begin{proof}
	取$u=\sum_{I,K}u_{IK}\dz_I\wedge\dzz_K\in D_{(p,q)}(G)$,那么
	\begin{align}\label{eq6.3.6}
		Tu
		&=\bar{\partial}u=\sum_{|I|=p}\sum_{|K|=q}\sum_{j=1}^{n}\pp{u_{IK}}{\bar{z}_j}\dzz_j\wedge\dz_I\wedge\dzz_K\notag\\
		&=(-1)^p\sum_{|I|=p}\sum_{|K|=q}\sum_{j=1}^{n}\pp{u_{IK}}{\bar{z}_j}\dz_I\wedge\dzz_j\wedge\dzz_K,\notag\\
		\langle f,Tu\rangle_{\varphi_2}
		&=(-1)^p\sum_{I,K}\sum_{j=1}^n \int_G f_{I,jK}\pp{\bar{u}_{IK}}{z_j}\ee^{-\varphi_2}\dif m\notag\\
		&=(-1)^{p-1}\sum_{I,K}\sum_{j=1}^n \int_G\bar{u}_{IK}\pp{}{z_j}\left(\ee^{-\varphi_2}f_{I,jK}\right)\dif m\notag\\
		&=(-1)^{p-1}\sum_{I,K}\sum_{j=1}^n \int_G \ee^{\varphi_1}\pp{}{z_j}\left(\ee^{-\varphi_2}f_{I,jK}\right)\ee^{-\varphi_1}\bar{u}_{IK}\dif m.
	\end{align}
上面第二个等式用了分部积分公式,因为$u_{IK}$有紧支集.另一方面,
\[\langle T^\ast f,u\rangle_{\varphi_1}=\sum_{I,K}\int_G (T^\ast f)_{IK}\bar{u}_{IK}\ee^{-\varphi_1}\dif m.\]
由于$\langle f,Tu\rangle_{\varphi_2}=\langle T^\ast f,u\rangle_{\varphi_1}$,把上式和\eqref{eq6.3.6}比较即得
\[(T^\ast f)_{IK}=(-1)^{p-1}\sum_{j=1}^{n}\ee^{\varphi_1}\pp{}{z_j}\left(\ee^{-\varphi_2}f_{I,jK}\right).\]
因而
\[T^\ast f=(-1)^{p-1}\sum_{I,K}\sum_{j=1}^{n}\ee^{\varphi_1}\pp{}{z_j}\left(\ee^{-\varphi_2}f_{I,jK}\right)\dz_I\wedge\dzz_K.\]

上式也可写为
\[T^\ast f=(-1)^{p-1}\sum_{I,K}\sum_{j=1}^{n}\ee^{\varphi_1}\left(\ee^{-\varphi_2}\pp{}{z_j}f_{I,jK}-\ee^{-\varphi_2}f_{I,jK}\pp{\varphi_2}{z_j}\right)\dz_I\wedge\dzz_K.\]
由此即得\eqref{eq6.3.5}.
\end{proof}
现在可以证明
\begin{prop}\label{prop6.3.4}
	设$G$是$\MC^n$中的域,$\{\eta_k\}$是引理\ref{lem6.3.1}中的函数列.如果$\varphi_2\in C^1(G)$,且
	\begin{equation}\label{eq6.3.7}
		\ee^{-\varphi_{j+1}}\sum_{l=1}^{n}\left|\pp{\eta_k}{\bar{z}_l}\right|^2\le\ee^{-\varphi_j},\quad j=1,2;k=1,2,\cdots,
	\end{equation}
那么对于任意$f\in D(T^\ast)\cap D(S)$,存在$h_k\in D_{(p,q+1)}(G)$,使得当$k\to\infty$时,有
\[\Vert h_k-f\Vert_{\varphi_2}\to0,\quad\Vert T^\ast h_k-T^\ast f\Vert_{\varphi_1}\to0,\quad\Vert Sh_k-Sf\Vert_{\varphi_3}\to0.\]
\end{prop}
\textbf{证明的思路}\quad 证明分为两部分:

(一)\hypertarget{6.3.4}{}
先证对于任意$f\in D(T^\ast)\cap D(S)$,必有具有紧支集的$g_k\in D(T^\ast)\cap D(S)$,使得当$k\to\infty$时,有
\begin{equation}\label{eq6.3.8}
	\Vert g_k-f\Vert_{\varphi_2}\to0,\quad\Vert T^\ast g_k-T^\ast f\Vert_{\varphi_1}\to0,\quad\Vert Sg_k-Sf\Vert_{\varphi_3}\to0.
\end{equation}

(二)\hypertarget{6.3.4}{}
对于具有紧支集的$f\in D(T^\ast)\cap D(S)$,证明存在$h_k\in D_{(p,q+1)}(G)$,使得当$k\to\infty$时,有
\begin{equation}\label{eq6.3.9}
	\Vert h_k-f\Vert_{\varphi_2}\to0,\quad\Vert T^\ast h_k-T^\ast f\Vert_{\varphi_1}\to0,\quad\Vert Sh_k-Sf\Vert_{\varphi_3}\to0.
\end{equation}
\begin{proof}
	\hyperlink{6.3.4}{(一)}
	对于任意的$f\in D(T^\ast)\cap D(S)$,命$g_k=\eta_k f$,则$g_k$就满足\eqref{eq6.3.8}.因为$\eta_k$有紧支集,所以$g_k$有紧支集.现在先来证明\eqref{eq6.3.8}的第三个极限式.由引理\ref{lem6.3.2}\hyperlink{6.3.2}{(1)}可得
	\[\left|\bar{\partial}(\eta_k f)-\eta_k\bar{\partial}f\right|^2\ee^{-\varphi_3}<|f|^2\ee^{-\varphi_2},\]
	由于$f\in L_{(p,q+1)}^2(G,\varphi_2)$,上式右端可积,根据Lebesgue控制收敛定理有
	\begin{align}\label{eq6.3.10}
		\lim_{k\to\infty}\Vert S(\eta_k f)-\eta_k Sf\Vert_{\varphi_3}^2
		&=\int_G\lim_{k\to\infty}|S(\eta_k f)-\eta_k Sf|^2\ee^{-\varphi_3}\dif m\notag\\
		&=\lim_{l\to\infty}\int_{K_l}\lim_{k\to\infty}|S(\eta_k f)-\eta_k Sf|^2\ee^{-\varphi_3}\dif m=0,
	\end{align}
这里$\{K_l\}$是$G$的一个正规穷竭,当$l$取定时,能取到充分大的$k_0$,当$k>k_0$时,$\eta_k$在$K_l$上恒等于$1$,因而被积函数为$0$.因为$f\in D(S)$,按照定义$Sf\in L_{(p,q+2)}^2(G,\varphi_3)$.由于$|\eta_k Sf-Sf|\le 2|Sf|$,故可用控制收敛定理,得
\begin{align}\label{eq6.3.11}
	\lim_{k\to\infty}\Vert \eta_k Sf-Sf\Vert_{\varphi_3}
	&=\int_G\lim_{k\to\infty}|\eta_k Sf-Sf|^2\ee^{-\varphi_3}\dif m\notag\\
	&=\lim_{l\to\infty}\int_{K_l}\lim_{k\to\infty}|\eta_k Sf-Sf|^2\ee^{-\varphi_3}\dif m=0.
\end{align}
于是,从\eqref{eq6.3.10},\eqref{eq6.3.11}即得
\[\Vert Sg_k-Sf\Vert_{\varphi_3}\le\Vert S(\eta_k f)-\eta_k Sf\Vert_{\varphi_3}+\Vert \eta_k Sf-Sf\Vert_{\varphi_3}\to0.\]
这就是\eqref{eq6.3.8}的第三个极限式.\eqref{eq6.3.8}的第一个极限式用控制收敛定理直接可得.最后正\eqref{eq6.3.8}的第二式.由引理\ref{lem6.3.2}\hyperlink{6.3.2}{(2)},$\eta_k f\in D(T^\ast)$.于是
\begin{align*}
	&|\langle T^\ast(\eta_k f)-\eta_k T^\ast f,u\rangle_{\varphi_1}|\\
	=&|\langle T^\ast(\eta_k f),u\rangle_{\varphi_1}-\langle \eta_k T^\ast f,u\rangle_{\varphi_1}|=|\langle \eta_k f,Tu\rangle_{\varphi_1}-\langle T^\ast f,\eta_k u\rangle_{\varphi_1}|\\
	=&|\langle f,\eta_k Tu\rangle_{\varphi_1}-\langle f,T(\eta_k u)\rangle_{\varphi_1}|=|\langle f,\eta_k Tu-T(\eta_k u)\rangle_{\varphi_2}|.
\end{align*}

再用引理\ref{lem6.3.2}\hyperlink{6.3.2}{(1)},即得
\begin{align*}
	\left|\int_G (T^\ast(\eta_k f)-\eta_k T^\ast f)\bar{u}\ee^{-\varphi_1}\dif m\right|
	&\le\int_G |f|\,|\eta_k Tu-T(\eta_k u)|\ee^{-\varphi_2}\dif m\\
	&\le\int_G |f|\,|u|\ee^{-\frac{\varphi_1}{2}+\frac{\varphi_2}{2}}\cdot\ee^{-\varphi_2}\dif m\\
	&=\int_G |f|\,|u|\ee^{-\frac{\varphi_1}{2}-\frac{\varphi_2}{2}}\dif m.
\end{align*}
取$u=T^\ast(\eta_k f)-\eta_k T^\ast f$,并注意到它的支集在$G\setminus K_k$中,上式可写为
\begin{align*}
	&\int_G|T^\ast(\eta_k f)-\eta_k T^\ast f|^2\ee^{-\varphi_1}\dif m\\
	\le&\int_G |f|\,|T^\ast(\eta_k f)-\eta_k T^\ast f|\ee^{-\frac{\varphi_1}{2}-\frac{\varphi_2}{2}}\dif m\\
	=&\int_{G\setminus K_k}|f|\ee^{-\frac{\varphi_2}{2}}|T^\ast(\eta_k f)-\eta_k T^\ast f|\ee^{-\frac{\varphi_1}{2}}\dif m\\
	\le&\left(\int_{G\setminus K_k}|f|^2\ee^{-\varphi_2}\right)^{\frac12}\left(\int_{G\setminus K_k}|T^\ast(\eta_k f)-\eta_k T^\ast f|^2\ee^{-\varphi_1}\dif m\right)^{\frac12}.
\end{align*}
于是得
\[\Vert T^\ast(\eta_k f)-\eta_k T^\ast f\Vert_{\varphi_1}\le\left(\int_{G\setminus K_k}|f|^2\ee^{-\varphi_2}\right)^{\frac12}\to0,\quad k\to\infty.\]
用类似于上面的方法,即可得证
\[\Vert T^\ast g_k-T^\ast f\Vert_{\varphi_1}\to0,\quad k\to\infty.\]
	
	\hyperlink{6.3.4}{(二)}
	设$f=\sum_{I,J}f_{IJ}\dz_I\wedge\dzz_J\in D(T^\ast)\cap D(S)$,它在$G$中有紧支集,即$f_{IJ}$在$G$中有公共的紧支集.规定
	\[(f\ast\chi_\varepsilon)(x)=\sum_{I,J}(f_{IJ}\ast\chi_\varepsilon)(x)\dz_I\wedge\dzz_J.\]
	即$f_\varepsilon(x)=\sum_{I,J}(f_{IJ})_\varepsilon \dz_I\wedge\dzz_J$.设$f_{IJ}$的公共紧支集为$K$,则由定理\ref{thm6.1.1},$(f_{IJ})_\varepsilon$的紧支集为
	\[K_1=\{x\colon d(x,K)\le\varepsilon\}.\]
	取$\varepsilon$充分小,可使$K_1\subset G$.因而$(f_{IJ})_\varepsilon\in C_0^\infty(G)$,而且$\Vert (f_{IJ})_\varepsilon-f_{IJ}\Vert_{L^2}\to0$.因而当$\varepsilon\to0$时,$\Vert f_\varepsilon-f\Vert_{\varphi_2}\to0$,这里$f_\varepsilon\in D_{(p,q+1)}(G)$.由命题\ref{prop6.1.5}知道$Sf_\varepsilon=(Sf)_\varepsilon$,因而当$\varepsilon\to0$时,
	\[\Vert Sf_\varepsilon-Sf\Vert_{\varphi_3}=\Vert(Sf)_\varepsilon-Sf\Vert_{\varphi_3}\to0.\]
	最后来证明当$\varepsilon\to0$时,$\Vert T^\ast f_\varepsilon-T^\ast f\Vert_{\varphi_1}\to0$.根据引理\ref{lem6.3.3}中$T^\ast f$的表达式\eqref{eq6.3.5},若记\index[symbolindex]{\textbf{其它符号}!$\theta f$}
	\[\theta f=(-1)^{p-1}\sum_{I,K}\sum_{j=1}^{n}\pp{f_{I,jK}}{z_j}\dz_I\wedge\dzz_K,\]
	\[a f=(-1)^p\sum_{I,K}\sum_{j=1}^n f_{I,jK}\pp{\varphi_2}{z_j}\dz_I\wedge\dzz_K,\]
	则
	\[\ee^{\varphi_2-\varphi_1}T^\ast f=\theta f+a f.\]
	由命题\ref{prop6.1.5}可得
	\begin{align*}
		(\theta+a)f_\varepsilon
		&=\theta f_\varepsilon+a f_\varepsilon=(\theta f)_\varepsilon+a f_\varepsilon\\
		&=((\theta+a)f)_\varepsilon-(af)_\varepsilon+af_\varepsilon.
	\end{align*}
根据定理\ref{thm6.1.1}知道,上式在$L^2$中收敛于
\[(\theta+a)f-af+af=(\theta+a)f.\]
此即$\Vert T^\ast f_\varepsilon-T^\ast f\Vert_{\varphi_1}\to0(\varepsilon\to0)$.取$h_k=(f)_{\frac1k}$,即得\hyperlink{6.3.4}{(二)}的证明.

现在对任意的$f\in D(T^\ast)\cap D(S)$,按照\hyperlink{6.3.4}{(一)}有一列具有紧支集的$\{g_k\}\subset D(T^\ast)\cap D(S)$,使得\eqref{eq6.3.8}成立.对于每个$g_k$,按照\hyperlink{6.3.4}{(二)}有$\{g_{kl}\}\subset D_{(p,q+1)}(G)$,使得\eqref{eq6.3.9}成立.对每个$g_k$,必能在$\{g_{kl}\}$中选出一个,记为$h_k$,使得$\Vert g_k-h_k\Vert_{\varphi_2}<\frac1k$,同时使另外两式也成立.对任意$\varepsilon>0$,取$k>k_0$,使得
\[\Vert f-g_k\Vert_{\varphi_2}<\frac{\varepsilon}{2},\Vert g_k-h_k\Vert_{\varphi_2}<\frac1k<\frac{\varepsilon}{2},\]
于是$\Vert f-h_k\Vert_{\varphi_2}<\varepsilon$,另外两式也同时成立,$\{h_k\}$即为所求.
\end{proof}
\subsection{基本不等式成立的条件}
因为基本不等式\eqref{eq6.3.1}只涉及$\Vert f\Vert_{\varphi_2},\Vert T^\ast f\Vert_{\varphi_1},\Vert Sf\Vert_{\varphi_3}$这三个量,有了命题\ref{prop6.3.4}之后,只要对$f\in D_{(p,q+1)}(G)$来证明不等式\eqref{eq6.3.1}就行了.

现在来看,能否找到$\varphi_1,\varphi_2,\varphi_3$,使得命题\ref{prop6.3.4}中的条件\eqref{eq6.3.7}成立.
对于给定的$x\in G$,总能有紧集$K\subset G$,使得$x\in K$,故当$k>k_0$时,$\eta_k(x)\equiv1$在$K$上成立,于是$\pp{\eta_k(x)}{\bar{z}_l}=0$.因此
\[\psi_1(x)=\sup\left\{\sum_{l=1}^{n}\left|\pp{\eta_1(x)}{\bar{z}_l}\right|^2,\sum_{l=1}^{n}\left|\pp{\eta_2(x)}{\bar{z}_l}\right|^2,\cdots\right\}\]
在$G$上有定义且取正值.命$\psi(x)=\log\psi_1(x)$,于是
\[\sum_{l=1}^{n}\left|\pp{\eta_k(x)}{\bar{z}_l}\right|^2\le\psi_1(x)=\ee^{\psi(x)}.\]
对于任意的$\varphi\in C^2(G)$,命
\[\varphi_1=\varphi-2\psi,\quad \varphi_2=\varphi-\psi,\quad \varphi_3=\varphi,\]
我们有
\[\ee^{-\varphi_2}\sum_{l=1}^{n}\left|\pp{\eta_k}{\bar{z}_l}\right|^2\le\ee^{-\varphi_2}\ee^\psi=\ee^{-\varphi+2\psi}=\ee^{-\varphi_1},\]
\[\ee^{-\varphi_3}\sum_{l=1}^{n}\left|\pp{\eta_k}{\bar{z}_l}\right|^2\le\ee^{-\varphi_3}\ee^\psi=\ee^{-(\varphi-\psi)}=\ee^{-\varphi_2},\]
即条件\eqref{eq6.3.7}成立.

现在可以证明
\begin{prop}\label{prop6.3.5}
	设$\varphi,\psi$及$\varphi_1,\varphi_2,\varphi_3$如上所述.如果对$w\in\MC^n$,有
	\begin{equation}\label{eq6.3.12}
		\sum_{j,k=1}^{n}\pppp{\varphi}{z_j}{\bar{z}_k}w_j\bar{w}_k\ge2\left(|\bar{\partial}\psi|^2+\ee^{\psi}\right)\sum_{j=1}^{n}|w_j|^2,
	\end{equation}
那么对任意$f\in D_{(p,q+1)}(G)$,便有
\[\Vert f\Vert_{\varphi_2}^2\le\Vert T^\ast f\Vert_{\varphi_1}^2+\Vert Sf\Vert_{\varphi_3}^2.\]
\end{prop}
在证明命题\ref{prop6.3.5}之前,先证明
\begin{lemma}\label{lem6.3.6}
	设$f=\sum_{|I|=p}\sum_{|J|=q+1}f_{IJ}\dz_I\wedge\dzz_J$,那么
	\[\left|\bar{\partial}f\right|^2=\sum_{I,J}\sum_{j=1}^{n}\left|\pp{f_{IJ}}{\bar{z}_j}\right|^2-\sum_{I,K}\sum_{j,k=1}^n \pp{f_{I,jK}}{\bar{z}_k}\cdot\bar{\pp{f_{I,kK}}{\bar{z}_j}},\]
	这里$|K|=q$.
\end{lemma}
\begin{proof}
	因为
	\begin{align*}
		\bar{\partial}f
		&=\sum_{|I|=p}\sum_{|J|=q+1}\sum_{j=1
		}^{n}\pp{f_{IJ}}{\bar{z}_j}\dzz_j\wedge\dz_I\wedge\dzz_J\\
		&=(-1)^p\sum_{|I|=p}\sum_{|J|=q+1}\sum_{j=1
		}^{n}\pp{f_{IJ}}{\bar{z}_j}\dz_I\wedge\dzz_j\wedge\dzz_J,
	\end{align*}
	这里$I,J$分别是$p$个数字和$q+1$个数字的增排列,但$jJ$就不一定是增排列了.因此不能写$\left|\bar{\partial}f\right|^2=\sum_{I,J}\sum_{j=1}^{n}\left|\pp{f_{IJ}}{\bar{z}_j}\right|^2$,还必须从此式减去那些因$jJ$不是增排列而产生的项.为此,把$\left|\bar{\partial}f\right|^2$写成一般形式
	\begin{equation}\label{eq6.3.13}
		\left|\bar{\partial}f\right|^2=\sum_{I,J,L}\sum_{j,l=1}^{n}\pp{f_{IJ}}{\bar{z}_j}\bar{\pp{f_{IL}}{\bar{z}_l}}\varepsilon_{lL}^{jJ}.
	\end{equation}
	显然,如果$j\in J$或$l\in L$,则$\varepsilon_{lL}^{jJ}=0$;如果$\{j\}\cup J\neq \{l\}\cup L$,则也有$\varepsilon_{lL}^{jJ}=0$.所以在下面的讨论中,排除这三种情况.
	
	如果$jJ$和$lL$作为$q+2$个数字的排列,它们的奇偶性不同,这意味着$jJ$和$lL$都排成增排列时,$\dzz_j\wedge\dzz_J$和$\dzz_l\wedge\dzz_L$的符号不同,因而$\varepsilon_{lL}^{jJ}=-1$.反之,则$\varepsilon_{lL}^{jJ}=1$.实际上,$\varepsilon_{lL}^{jJ}$就是置换$\binom{
	jJ}{lL}$的符号.现在仔细研究\eqref{eq6.3.13}中的项.首先考虑$j=l$的项,这时必有$J=L,j\notin J$和$l\notin L$,于是$\varepsilon_{lL}^{jJ}=1$,故\eqref{eq6.3.13}中相应的项为
	\[\sum_{I,J}\sum_{j\notin J}\left|\pp{f_{IJ}}{z_j}\right|^2.\]
	再考虑$j\neq l$的项,因为$\varepsilon_{lL}^{jJ}\neq0$,故必有$l\in J,j\in L$,但由于$\{j\}\cup J=\{l\}\cup L$,故从$J$中去掉$l$和从$L$中去掉$j$所剩下的$q$个数字的集合$K$和$K'$完全相同.由于
	\[\varepsilon_{lL}^{jJ}=\varepsilon_{jlK}^{jJ}\varepsilon_{ljK}^{jlK}\varepsilon_{lL}^{ljK}=-\varepsilon_{lK}^{J}\varepsilon_{L}^{jK},\]
	所以对应于$j\neq l$的项是
	\[-\sum_{I,J,L}\sum_{j\neq l}\pp{f_{IJ}}{\bar{z}_j}\bar{\pp{f_{IL}}{\bar{z}_l}}\varepsilon_{lK}^J \varepsilon_{L}^{jK}=-\sum_{I,K}\sum_{j\neq l}\pp{f_{I,lK}}{\bar{z}_j}\bar{\pp{f_{I,jK}}{\bar{z}_l}}.\]
	于是
	\begin{align*}
		\left|\bar{\partial}f\right|^2
		&=\sum_{I,J}\sum_{j\neq J}\left|\pp{f_{IJ}}{\bar{z}_j}\right|^2-\sum_{I,K}\sum_{j\neq l}\pp{f_{I,lK}}{\bar{z}_j}\bar{\pp{f_{I,jK}}{\bar{z}_l}}\\
		&=\sum_{I,J}\sum_{j=1}^n\left|\pp{f_{IJ}}{\bar{z}_j}\right|^2-\sum_{I,K}\sum_{j,l=1}^n\pp{f_{I,lK}}{\bar{z}_j}\bar{\pp{f_{I,jK}}{\bar{z}_l}}\qedhere
	\end{align*}
\end{proof}
\begin{lemma}\label{lem6.3.7}
		设$\varphi\in C^2(G)$是一个取实值的函数,记\index[symbolindex]{\textbf{其它符号}!$\delta_j w$}
		\[\delta_j w=\ee^\varphi \pp{}{z_j}\left(w\ee^{-\varphi}\right)=\pp{w}{z_j}-w\pp{\varphi}{z_j},\]
		那么
		
		(1)\hypertarget{6.3.7}{}
		若$u,v\in C_0^\infty(G)$,则
		\[\int_G u\bar{\pp{v}{\bar{z}_j}}\ee^{-\varphi}\dif m=-\int_G (\delta_j u)\bar{v}\ee^{-\varphi}\dif m.\]
		
		
		(2)\hypertarget{6.3.7}{}
		若$w_j\in C_0^\infty(G),j=1,\cdots,n$,则
		\[\sum_{j,k=1}^{n}\int_G\left\{(\delta_j w_j)\bar{(\delta_k w_k)}-\pp{w_j}{\bar{z}_k}\bar{\pp{w_k}{\bar{z}_j}}\right\}\ee^{-\varphi}\dif m=\sum_{j,k=1}^{n}\int_G w_j\bar{w}_k\pppp{\varphi}{z_j}{\bar{z}_k}\ee^{-\varphi}\dif m.\]
\end{lemma}
\begin{proof}	
	\hyperlink{6.3.7}{(1)}
	由于$u,v$在$G$中有紧支集,由分部积分法,
	\[\int_G u\bar{\pp{v}{\bar{z}_j}}\ee^{-\varphi}\dif m=\int_G u\pp{\bar{v}}{z_j}\ee^{-\varphi}\dif m=-\int_G \bar{v}\pp{}{z_j}\left(u\ee^{-\varphi}\right)\dif m=-\int_G(\delta_j u)\bar{v}\ee^{-\varphi}\dif m.\]
	
	\hyperlink{6.3.7}{(2)}
	由直接验算可得
	\[\left(\delta_k\pp{}{\bar{z}_j}-\pp{}{\bar{z}_j}\delta_k\right)w_k=\pppp{\varphi}{z_k}{\bar{z}_j}w_k,\]
	即
	\[-\pp{}{\bar{z}_j}\delta_kw_k=\pppp{\varphi}{z_k}{\bar{z}_j}w_k-\delta_k\pp{}{\bar{z}_j}w_k.\]
	现在\hyperlink{6.3.7}{(1)}中取$u=w_j,v=\delta_k w_k$,那么
	\begin{align*}
		\int_G(\delta_jw_j)\bar{(\delta_kw_k)}\ee^{-\varphi}
		&=-\int_G w_j\bar{\pp{}{\bar{z}_j}(\delta_kw_k)}\ee^{-\varphi}\dif m\\
		&=\int_G w_j\bar{w}_k\pppp{\varphi}{\bar{z}_k}{z_j}\ee^{-\varphi}\dif m-\int_G w_j\bar{\delta_k\pp{w_k}{\bar{z}_j}}\ee^{-\varphi}\dif m,
	\end{align*}
	因而
	\begin{equation}\label{eq6.3.14}
		\sum_{j,k=1}^{n}\int_G(\delta_jw_j)\bar{(\delta_kw_k)}\ee^{-\varphi}\dif m
		=\sum_{j,k=1}^{n}\int_G\pppp{\varphi}{\bar{z}_k}{z_j}w_j\bar{w}_k\ee^{-\varphi}\dif m-\sum_{j,k=1}^{n}\int_Gw_j\bar{\delta_k\pp{w_k}{\bar{z}_j}}\ee^{-\varphi}\dif m,
	\end{equation}
	这里我们已经把上式右端第二项求和指标$j,k$作了调换.另一方面,再在\hyperlink{6.3.7}{(1)}中取$u=\pp{w_j}{\bar{z}_k},v=w_k$,有
	\[\int_G \pp{w_j}{\bar{z}_k}\bar{\pp{w_k}{\bar{z}_j}}\ee^{-\varphi}\dif m=-\int_G \delta_j\pp{w_j}{\bar{z}_k}\bar{w}_k\ee^{-\varphi}\dif m,\]
	取共轭并对$j,k$求和得
	\begin{align*}
		-\sum_{j,k=1}^n\int_G w_k\bar{\delta_j\pp{w_j}{\bar{z}_k}}\ee^{-\varphi}\dif m
		&=\sum_{j,k=1}^n\int_G \bar{\pp{w_j}{\bar{z}_k}}\pp{w_k}{\bar{z}_j}\ee^{-\varphi}\dif m\\
		&=\sum_{j,k=1}^n \int_G \pp{w_j}{\bar{z}_k}\bar{\pp{w_k}{\bar{z}_j}}\ee^{-\varphi}\dif m.
	\end{align*}
	从\eqref{eq6.3.14}减去上式,即得
	\[\sum_{j,k=1}^{n}\int_G\left\{(\delta_j w_j)\bar{(\delta_k w_k)}-\pp{w_j}{\bar{z}_k}\bar{\pp{w_k}{\bar{z}_j}}\right\}\ee^{-\varphi}\dif m=\sum_{j,k=1}^{n}\int_G w_j\bar{w}_k\pppp{\varphi}{\bar{z}_k}{z_j}\ee^{-\varphi}\dif m.\qedhere\]
\end{proof}
\begin{proof}[\textbf{命题\ref{prop6.3.5}的证明}]
	
	由引理\ref{lem6.3.3}的\eqref{eq6.3.5},$T^\ast f$可以表示为
	\[\ee^{\varphi_2-\varphi_1}T^\ast f=(-1)^{p-1}\sum_{I,K}\sum_{j=1}^{n}\left(\pp{}{z_j}f_{I,jK}-f_{I,jK}\pp{\varphi_2}{z_j}\right)\dz_I\wedge\dzz_K.\]
	注意到$\psi=\varphi_2-\varphi_1,\varphi_2=\varphi-\psi$以及$\delta_j w=\pp{w}{z_j}-w\pp{\varphi}{z_j}$,上式可写为
	\[\ee^\psi T^\ast f=(-1)^{p-1}\sum_{I,K}\sum_{j=1}^{n}\delta_j f_{I,jK}\dz_I\wedge\dzz_K+(-1)^{p-1}\sum_{I,K}\sum_{j=1}^{n}f_{I,jK}\pp{\psi}{z_j}\dz_I\wedge\dzz_K,\]
	把右端两式分别简记为$X$和$Y$,上式可简写为
	\[\ee^{-\psi}X=T^\ast f-\ee^{-\psi}Y.\]
	因而得
	\begin{align}\label{eq6.3.15}
		\Vert \ee^{-\psi}X\Vert_{\varphi_1}^2
		&=\left(\Vert T^\ast f\Vert_{\varphi_1}^2+\Vert \ee^{-\psi} Y\Vert_{\varphi_1}\right)^2\notag\\
		&\le2\Vert T^\ast f\Vert_{\varphi_1}^2+2\Vert\ee^{-\psi}Y\Vert_{\varphi_1}^2.
	\end{align}
	从$X,Y$的表达式得知
	\begin{align*}
		\Vert\ee^{-\psi}X\Vert_{\varphi_1}^2
		&=\int_G \sum_{I,K}\left|\sum_{j=1}^{n}\delta_j f_{I,jK}\right|^2\ee^{-2\psi}\ee^{-\varphi_1}\dif m\\
		&=\int_G\sum_{I,K}\sum_{j,k=1}^{n}(\delta_j f_{I,jK})\bar{(\delta_k f_{I,kK})}\ee^{-\varphi}\dif m,\\
		\Vert \ee^{-\psi}Y\Vert_{\varphi_1}^2
		&=\int_G\sum_{I,K}\left|\sum_{j=1}^{n}f_{I,jK}\pp{\psi}{z_j}\right|^2\ee^{-\varphi}\dif m\\
		&\le\int_G \sum_{I,K}\left\{\sum_{j=1}^{n}|f_{I,jK}|^2\sum_{j=1}^{n}\left|\pp{\psi}{z_j}\right|^2\right\}\ee^{-\varphi}\dif m\\
		&\le\int_G |f|^2\,|\partial\psi|^2\ee^{-\varphi}\dif m.
	\end{align*}
	代入\eqref{eq6.3.15}得
	\[\int_G\sum_{I,K}\sum_{j,k=1}^{n}(\delta_j f_{I,jK})\bar{(\delta_k f_{I,kK})}\ee^{-\varphi}\dif m\le 2\Vert T^\ast f\Vert_{\varphi_1}^2+2\int_G |f|^2\,|\partial\psi|^2\ee^{-\varphi}\dif m.\]
	把引理\ref{lem6.3.6}中$\left|\bar{\partial}f\right|^2$表达式的两端在$G$上积分并分别加在上面不等式的两端得
	\begin{align*}
		&\int_G\sum_{I,K}\sum_{j,k=1}^{n}\left\{(\delta_j f_{I,jK})\bar{(\delta_k f_{I,kK})}-\pp{f_{I,jK}}{\bar{z}_k}\bar{\pp{f_{I,kK}}{\bar{z}_j}}\right\}\ee^{-\varphi}\dif m+\int_G\sum_{I,J}\sum_{j=1}^{n}\left|\pp{f_{IJ}}{\bar{z}_j}\right|^2\ee^{-\varphi}\dif m\\
		\le& 2\Vert T^\ast f\Vert_{\varphi_1}^2+\Vert Sf\Vert_{\varphi_3}^2+2\int_G |f|^2\,|\partial\psi|^2 \ee^{-\varphi}\dif m.
	\end{align*}
	在引理\ref{lem6.3.7}\hyperlink{6.3.7}{(2)}中取$w_j=f_{I,jK}$,上式便可写为
	\begin{align}\label{eq6.3.16}
		&\sum_{I,K}\int_G\sum_{j,k=1}^{n}f_{I,jK}\bar{f}_{I,kK}\pppp{\varphi}{z_j}{\bar{z}_k}\ee^{-\varphi}\dif m+\sum_{I,J}\sum_{j=1}^{n}\int_G\left|\pp{f_{IJ}}{\bar{z}_j}\right|^2\ee^{-\varphi}\dif m\notag\\
		&\le 2\Vert T^\ast f\Vert_{\varphi_1}^2+\Vert Sf\Vert_{\varphi_3}^2+2\int_G |f|^2\,|\partial\psi|^2\ee^{-\varphi}\dif m.
	\end{align}
	根据命题\ref{prop6.3.5}的条件\eqref{eq6.3.12},
	\[\sum_{j,k=1}^{n}\pppp{\varphi}{z_j}{\bar{z}_k}f_{I,jK}\bar{f}_{I,kK}\ge2\left(\left|\bar{\partial}\psi\right|^2+\ee^{\psi}\right)\sum_{j=1}^{n}|f_{I,jK}|^2,\]
	因而有
	\begin{align*}
		&2\Vert T^\ast f\Vert_{\varphi_1}^2+\Vert Sf\Vert_{\varphi_3}^2+2\int_G |f|^2\,|\bar{\partial}\psi|^2\ee^{-\varphi}\dif m\\
		\ge&\sum_{I,K}\int_G 2\left(\left|\bar{\partial}\psi\right|^2+\ee^{\psi}\right)\sum_{j=1}^{n}|f_{I,jK}|^2\ee^{-\varphi}\dif m+\sum_{I,J}\sum_{j=1}^{n}\int_G\left|\pp{f_{IJ}}{\bar{z}_j}\right|\ee^{-\varphi}\dif m\\
		=&2\int_G |f|^2\,\left|\bar{\partial}\psi\right|^2\ee^{-\varphi}\dif m+2\Vert f\Vert_{\varphi_2}^2+\sum_{I,J}\sum_{j=1}^{n}\int_G\left|\pp{f_{IJ}}{\bar{z}_j}\right|^2\ee^{-\varphi}\dif m\\
		\ge&2\int_G |f|^2\,\left|\bar{\partial}\psi\right|^2\ee^{-\varphi}\dif m+2\Vert f\Vert_{\varphi_2}^2.
	\end{align*}
	由此即得
	\[\Vert f\Vert_{\varphi_2}^2\le\Vert T^\ast f\Vert_{\varphi_1}^2+\Vert Sf\Vert_{\varphi_3}^2.\]
	这就是要证明的不等式.
\end{proof}
\subsection{$\bar{\partial}$问题在拟凸域上有解}
有了这些准备之后,便可证明我们的主要定理.
\begin{theorem}\label{thm6.3.8}
	设$G$是$\MC^n$中的拟凸域.如果$f\in L_{(p,q+1)}^2(G,\loc)$且满足$\bar{\partial}f=0$,那么方程$\bar{\partial}u=f$有一个解$u\in L_{(p,q)}^2(G,\loc)$.
\end{theorem}	
	证明之前先解释记号$L_{(p,q)}^2(G,\loc)$.定义\index[symbolindex]{\textbf{微分形式}!$L_{\text{$(p$,$q)$}}^2(G$,$\loc)$}
	\[L^2(G,\loc)=\left\{f\colon\text{对每个$K\subset\subset G,\int_K |f|^2 \dif m<\infty$}\right\}.\]
	$L^2(G,\loc)$中的元素称为局部平方可积函数\index{J!局部平方可积函数}.显然$L^2(G,\varphi)\subset L^2(G,\loc)$.因为对任何$f\in L^2(G,\varphi)$,取$K\subset\subset G$,则在$K$上有$|\varphi|\le M$,于是
	\begin{align*}
		\int_K |f|^2 \dif m
		&\le \ee^M\int_K |f|^2\ee^{-\varphi}\dif m\\
		&\le \ee^M\int_G|f|^2\ee^{-\varphi}\dif m<\infty,
	\end{align*}
	即$f\in L^2(G,\loc)$.弄清了$L^2(G,\loc)$的意义以后,$L_{(p,q)}^2(G,\loc)$的意义自然就明白了.
	\begin{proof}[\textbf{定理\ref{thm6.3.8}的证明}]
		根据定理\ref{thm5.5.10},在拟凸域$G$上存在强多重次调和穷竭函数$\eta\in C^\infty(G)$.因而对任意实数$t$,有
		\[K_t=\{z\in G\colon \eta(z)\le t\}\subset\subset G.\]
		因为$\eta$是强多重次调和函数,故对任意$z\in G$及$\xi\in\MC^n$,有
		\[\sum_{j,k=1}^{n}\pppp{\eta(z)}{z_j}{\bar{z}_k}\xi_j\bar{\xi}_k>0.\]
		对于取定的$z\in G$,该二次型在单位球面上的最小值设为$m(z)>0$.于是,有
		\[\sum_{j,k=1}^{n}\pppp{\eta(z)}{z_j}{\bar{z}_k}\xi_j\bar{\xi}_k\ge m(z)\sum_{j=1}^{n}|\xi_j|^2.\]
		如命题\ref{prop6.3.5}那样取定$\psi$,命
		\[h(t)=\sup_{K_t}2\frac{\left|\bar{\partial}\psi\right|^2+\ee^{\psi(z)}}{m(z)},\]
		因为$K_t$是紧的,故$h(t)$有限且是$t$的单调增加的连续函数.今取$g\colon \MR\to\MR,g\in C^\infty(\MR),g$是递增的凸函数,$g'\ge0,g''>0$且$g'\ge h$.现在命$\varphi(z)=g(\eta(z))$,则有
		\begin{align*}
			\sum_{j,k=1}^{n}\pppp{\varphi}{z_j}{\bar{z}_k}w_j\bar{w}_k
			&=g''(\eta(z))\left|\sum_{j=1}^{n}\pp{\eta}{z_j}w_j\right|^2+g'(\eta(z))\sum_{j,k=1}^{n}\pppp{\varphi}{z_j}{\bar{z}_k}w_j\bar{w}_k\\
			&\ge g'(\eta(z))\sum_{j,k=1}^{n}\pppp{\varphi}{z_j}{\bar{z}_k}w_j\bar{w}_k\\
			&\ge g'(\eta(z))m(z)\sum_{j=1}^{n}|w_j|^2.
		\end{align*}
		对于取定的$z$,设$\eta(z)=t$,于是
		\[g'(\eta(z))=g'(t)\ge h(t)\ge2\frac{\left|\bar{\partial}\psi\right|^2+\ee^{\psi(z)}}{m(z)},\]
		代入上式即得
		\[\sum_{j,k=1}^{n}\pppp{\varphi}{z_j}{\bar{z}_k}w_j\bar{w}_k\ge2\left(\left|\bar{\partial}\psi\right|^2+\ee^\psi\right)\sum_{j=1}^{n}|w_j|^2.\]
		此式说明这样选取的$\varphi$满足命题\ref{prop6.3.5}的条件\eqref{eq6.3.12}.
		
		另一方面,因为$f\in L_{(p,q+1)}^2(G,\loc)$,对于任何$j$,积分$\int_{K_{j+1}\setminus K_j}|f|^2\dif m=b_j$是有限数,适当选取$\delta_j$,使得
		\[\sum_{j=1}^{\infty}b_j\delta_j<\infty.\]
		现在要求上面选取的$g$还满足
		\[g(t)\ge\log\frac1{\delta_j}+\sup\left\{\psi(z)\colon z\in K_{j+1}\setminus K_j\right\},j<t\le j+1.\]
		这样,当$z\in K_{j+1}\setminus K_j$时,$j<\eta(z)\le j+1$,因而
		\[\varphi(z)=g(\eta(z))\ge\log\frac1{\delta_j}+\psi(z),\]
		即
		\[\ee^{-(\varphi-\psi)}\le\delta_j\quad(z\in K_{j+1}\setminus K_j).\]
		于是
		\begin{align*}
			\Vert f\Vert_{\varphi_2}^2
			&=\int_G |f|^2\ee^{-(\varphi-\psi)}\dif m\\
			&=\sum_{j=1}^{\infty}\int_{K_{j+1}\setminus K_j}|f|^2\ee^{-(\varphi-\psi)}\dif m\le\sum_{j=1}^{\infty} b_j\delta_j<\infty.
		\end{align*}
		则$f\in L_{(p,q+1)}^2(G,\varphi_2)$,因而$\bar{\partial}u=f$有解
		\[u\in L_{(p,q)}^2(G,\varphi_1)\subset L_{(p,q)}^2(G,\loc).\qedhere\]
	\end{proof}
\section{$\bar{\partial}$问题解的正则性\label{sec6.4}}
上面我们已经证明了,如果$G$是$\MC^n$中的拟凸域,当$f\in L_{(p,q+1)}^2(G,\loc),\bar{\partial}f=0$时,$\bar{\partial}u=f$有一个解$u$,而且$u\in L_{(p,q)}^2(G,\loc)$.我们自然希望,当$f$有足够的可微性时,解$u$也有相应的可微性,但情况并不一定如此.

例如设$G=B_2$,即$\MC^2$中的单位球.命
\[f=2\left(\bar{z}_1+\bar{z}_2\right)\dzz_1\wedge\dzz_2.\]
它是一个具有$C^\infty$系数的$(0,2)$形式,命
\[h_1(z)=\begin{cases}
	0,&\Re z_2\le0,\\
	1,&\Re z_2>0,
\end{cases}\]
\[h_2(z)=\begin{cases}
	0,&\Re z_1\le0,\\
	1,&\Re z_1>0,
\end{cases}\]
\[u(z)=\left(h_1\pp{h_2}{\bar{z}_1}-\bar{z}_2^2\right)\dzz_1+\left(h_2\pp{h_1}{\bar{z}_2}+\bar{z}_1^2\right)\dzz_2.\]
通过直接计算易得$\bar{\partial}u=f$,即$u$是$\bar{\partial}u=f$的一个解,但这个解却是不连续的.另一方面,容易看出
\[v=-\bar{z}_2^2\dzz_1+\bar{z}_1^2\dzz_2\]
也满足$\bar{\partial}v=f$,而这个解的系数都是属于$C^\infty$的.

这个例子说明,当$f$有足够的可微性时,方程$\bar{\partial}u=f$的许多解中,既有性质良好的解,也有性质不好的解.我们的任务是要设法找出那些性质良好的解.
\subsection{$\MC^n$中的Sobolev空间\index{S!Sobolev空间}}
这一节的主要工具是Sobolev空间的理论.在\ref{sec6.1}中,我们曾在$\MR^N$的开集上引入Sobolev空间的概念.把$\MC^n$看成$\MR^{2n}$,\ref{sec6.1}中建立的Sobolev空间的理论在$\MC^n$中仍然有效.但是那里的微分算子$\DD^\alpha=\frac{\partial^{|\alpha|}}{\partial x_1^{\alpha_1}\cdots\partial x_N^{\alpha_N}}$现在应改成\index[symbolindex]{\textbf{导数}!$\DD_z^\alpha$,$\DD_{\bar{z}}^\beta$}
\[\DD_z^\alpha \DD_{\bar{z}}^\beta=\frac{\partial^{|\alpha|}}{\partial z_1^{\alpha_1}\cdots\partial z_n^{\alpha_n}}\frac{\partial^{|\beta|}}{\partial \bar{z}_1^{\alpha_1}\cdots\partial \bar{z}_n^{\alpha_n}},\]
这里$\alpha=(\alpha_1,\cdots,\alpha_n),\beta=(\beta_1,\cdots,\beta_n)$都是多重指标.这时Sobolev空间可定义为
\begin{definition}\label{def6.4.1}
	设$G$是$\MC^n$中的域,$s$是非负整数.$G$上的$s$阶Sobolev空间$W^s(G)$是所有这样的函数$f$构成的集合:对所有满足$|\alpha|+|\beta|\le s$的$\alpha,\beta$,它的弱导数$\DD_z^\alpha \DD_{\bar{z}}^\beta f\in L^2(G)$.
\end{definition}
若在这空间中引进如下的范数
\[\Vert f\Vert_s=\sum_{|\alpha|+|\beta|\le s}\Vert\DD_z^\alpha\DD_{\bar{z}}^\beta f\Vert_{L^2}^2,\]
则$W^s(G)$成为Banach空间.

若引进如下的内积
\[\langle f,g\rangle_s=\sum_{|\alpha|+|\beta|\le s}\int_G(\DD_z^\alpha\DD_{\bar{z}}^\beta f)\bar{(\DD_z^\alpha\DD_{\bar{z}}^\beta g)}\dif m,\]
则$W^s(G)$成为Hilbert空间.
\begin{definition}\label{def6.4.2}
	设$G$是$\MC^n$中的域,$s$是非负整数.$G$上的$s$阶$(p,q)$形式的Sobolev空间定义为\index[symbolindex]{\textbf{微分形式}!$W_{\text{$(p$,$q)$}}^s(G)$}
	\[W_{(p,q)}^s(G)=\left\{f=\sum_{|I|=p}\sum_{|J|=q}f_{IJ}\dz_I\wedge\dzz_J\colon f_{IJ}\in W^s(G)\right\},\]
	
	它的范数定义为
	\[\Vert f\Vert_s^2=\sum_{|I|=p}\sum_{|J|=q}\Vert f_{IJ}\Vert_s^2.\]
	
	自然定义下面的\index[symbolindex]{\textbf{函数和映射}!$W^s(G$,$\loc)$}\index[symbolindex]{\textbf{微分形式}!$W_{\text{$(p$,$q)$}}^s(G$,$\loc)$}
	\[W^s(G,\loc)=\left\{f\colon \DD_z^\alpha\DD_{\bar{z}}^\beta f\in L^2(G,\loc),|\alpha|+|\beta|\le s\right\},\]
	\[W_{(p,q)}^s(G,\loc)=\left\{f=\sum_{|I|=p}\sum_{|J|=q}f_{IJ}\dz_I\wedge\dzz_J\colon f_{IJ}\in W^s(G,\loc)\right\}.\]
\end{definition}
我们需要下面的
\begin{prop}\label{prop6.4.3}
	如果$f\in L^2(\MC^n)$具有紧支集,且$\pp{f}{\bar{z}_j}\in L^2(\MC^n),j=1,\cdots,n$,那么$f\in W^1(\MC^n)$.
\end{prop}
\begin{proof}
	按照$W^1(\MC^n)$的定义,只要证明,对所有的$j=1,\cdots,n$,有$\pp{f}{z_j}\in L^2(\MC^n)$就行了.先设$f\in C_0^\infty(\MC^n)$,由分部积分法即得
	\begin{align*}
		\int_{\MC^n}\left|\pp{f}{z_j}\right|^2\dif m
		&=\int_{\MC^n}\pp{f}{z_j}\bar{\pp{f}{z_j}}\dif m\\
		&=-\int_{\MC^n}\pppp{f}{\bar{z}_j}{z_j}\bar{f}\dif m=\int_{\MC^n}\left|\pp{f}{\bar{z}_j}\right|^2\dif m<\infty.
	\end{align*}
	对于一般的具有紧支集的$f\in L^2(\MC^n)$,用\ref{sec6.1}的函数光滑化的方法,作$f_\varepsilon=f\ast\chi_\varepsilon$.因为$f$具有紧支集,所以$f_\varepsilon\in C_0^\infty(\MC^n)$,且$\Vert f_\varepsilon-f\Vert_{L^2}\to0$(当$\varepsilon \to0$).从$\pp{f}{\bar{z}_j}\in L^2(\MC^n)$可得$\pp{f_\varepsilon}{\bar{z}_j}=\left(\pp{f}{\bar{z}_j}\right)_\varepsilon\in L^2(\MC^n)$,因而由上面所证,即知$\pp{f_\varepsilon}{z_j}\in L^2(\MC^n)$,所以$\left(\pp{f}{z_j}\right)_\varepsilon\in L^2(\MC^n)$.由于当$\varepsilon\to0$时,$\left(\pp{f}{z_j}\right)_\varepsilon$在$L^2$中收敛于$\pp{f}{z_j}$,因而$\pp{f}{z_j}\in L^2(\MC^n)$对所有$j=1,\cdots,n$成立.
\end{proof}
\begin{prop}\label{prop6.4.4}
	设$f=\sum_{I,J}f_{IJ}\dz_I\wedge\dzz_J\in L_{(p,q+1)}^2(\MC^n)$具有紧支集.\\
	如果$\bar{\partial}f\in L_{(p,q+2)}^2(\MC^n),\theta f\in L_{(p,q)}^2(\MC^n)$,这里
	\[\theta f=(-1)^{p-1}\sum_{|I|=p\atop|K|=q}\sum_{j=1}^{n}\pp{f_{I,jK}}{z_j}\dz_I\wedge\dzz_K,\]
	如命题\ref{prop6.3.4}的证明中所示,那么$f\in W_{(p,q+1)}^s(\MC^n)$.
\end{prop}
\begin{proof}
	和证明命题\ref{prop6.4.3}的方法一样.先设$f\in D_{(p,q+1)}(\MC^n)$.注意到\ref{sec6.3}的\eqref{eq6.3.16}对任意的$\varphi$和$\psi$都是成立的,现取$\varphi=\psi=0$,这时$\varphi_1=\varphi-2\psi=0,\varphi_2=\varphi-\psi=0,\varphi_3=\varphi=0$.于是\eqref{eq6.3.16}变成
	\[\sum_{I,J}\sum_{j=1}^{n}\int_{\MC^n}\left|\pp{f_{IJ}}{\bar{z}_j}\right|^2\dif m\le2\Vert T^\ast f\Vert_{L^2}^2+\Vert Sf\Vert_{L^2}^2.\]
	这时$T^\ast f$可以表示为
	\[T^\ast f=(-1)^{p-1}\sum_{|I|=p\atop|K|=q}\sum_{j=1}^{n}\pp{f_{I,jK}}{z_j}\dz_I\wedge\dzz_K=\theta f,\]
	故上式可写为
	\[\sum_{I,J}\sum_{j=1}^{n}\int_{\MC^n}\left|\pp{f_{IJ}}{\bar{z}_j}\right|^2\dif m\le2\Vert \theta f\Vert_{L^2}^2+\Vert \bar{\partial}f\Vert_{L^2}^2,\]
	由此得$\pp{f_{IJ}}{\bar{z}_j}\in L^2(\MC^n)$.由命题\ref{prop6.4.3}得$f_{IJ}\in W^1(\MC^n)$,因而$f\in W_{(p,q+1)}^1(\MC^n)$.对于一般的具有紧支集的$f\in L_{(p,q+1)}^2(\MC^n)$,和命题\ref{prop6.4.3}的证法一样,用光滑化的方法即可得证.
\end{proof}
\subsection{内部正则性定理}
现在可以证明
\begin{theorem}\label{thm6.4.5}
	设$G$是$\MC^n$中的拟凸域,$s$是非负整数.如果$f\in W_{(p,q+1)}^s(G,\loc)$,且$\bar{\partial}f=0$,那么
	
	(1)\hypertarget{6.4.5}{}
	如果$q\ge1$,那么方程$\bar{\partial}u=f$的标准解$u\in W_{(p,q)}^{s+1}(G,\loc)$;
	
	(2)\hypertarget{6.4.5}{}
	如果$q=0$,那么方程$\bar{\partial}u=f$的任意解$u\in W_{(p,0)}^{s+1}(G,\loc)$.
\end{theorem}
\begin{proof}
	先证\hyperlink{6.4.5}{(1)}. 设$q\ge1$.从定理\ref{thm6.3.8}、命题\ref{prop6.3.5}和定理\ref{thm6.2.5}知道方程$\bar{\partial}u=f$存在标准解(标准解的定义见定义\ref{def6.2.6}).现在证明这个标准解$u$还满足$\theta\left(\ee^{-\varphi_1}u\right)=0$,这里$\theta$是命题\ref{prop6.4.4}中提到的算子.事实上,因为$u$是标准解,所以存在$v\in D(T^\ast)$,使得$u=T^\ast v$.于是
	\begin{align*}
		\ee^{-\varphi_1} T^\ast v
		&=(-1)^{p-1}\sum_{I,K}\sum_{j=1}^{n}\pp{}{z_j}\left(\ee^{-\varphi_2}v_{I,jK}\right)\dz_I\wedge\dzz_K\\
		&=\theta\left(\ee^{-\varphi_2}v\right).
	\end{align*}
	从直接验算可得$\theta^2=0$,因而
	\[\theta\left(\varphi^{-\varphi_1}u\right)=\theta\left(\ee^{-\varphi_1}T^\ast v\right)=\theta^2\left(\ee^{-\varphi_2}v\right)=0.\]
	另一方面,
	\begin{align*}
		\theta\left(\ee^{-\varphi_1}u\right)
		&=(-1)^{p-1}\sum_{I,L}\sum_{j=1}^{n}\pp{}{z_j}\left(\ee^{-\varphi_1}u_{I,jL}\right)\dz_I\wedge\dzz_L\\
		&=(-1)^{p-1}\sum_{I,L}\sum_{j=1}^{n}\left(-\ee^{-\varphi_1}\pp{\varphi_1}{z_j}u_{I,jL}+\ee^{-\varphi_1}\pp{u_{I,jL}}{z_j}\right)\dz_I\wedge\dzz_L\\
		&=\ee^{-\varphi_1}\theta u-\ee^{-\varphi_1}bu,
	\end{align*}
	这里
	\[bu=(-1)^{p-1}\sum_{I,L}\sum_{j=1}^{n}u_{I,jL}\pp{\varphi_1}{z_j}\dz_I\wedge\dzz_L,|L|=q-1.\]
	由此得$\theta u=bu$,这里$b$是一个系数属于$C^\infty$的零阶微分算子.现在假定对某个$m(0\le m\le s)$证明了
	\[u\in W_{(p,q)}^m(G,\loc),\]
	至少$m=0$时这个假定是成立的.任取$\chi\in C_0^\infty(G)$,则
	\[\bar{\partial}(\chi u)=(\bar{\partial}\chi)\wedge u+\chi\bar{\partial}u=(\bar{\partial}\chi)\wedge u+\chi f\in W_{(p,q+1)}^m(G,\loc).\]
	这里我们已经用了$f\in W_{(p,q+1)}^s(G,\loc)$的假定.\\
	因为$\theta u=bu$,所以$\theta(\chi u)\in W_{(p,q-1)}^m(G,\loc)$.现在设多重指标$\alpha=(\alpha_1,\cdots,\alpha_n),\beta=(\beta_1,\cdots,\beta_n)$,满足$|\alpha|+|\beta|\le m$,按照Sobolev空间的定义有
	\[\bar{\partial}(\DD_z^\alpha\DD_{\bar{z}}^\beta(\chi u))=\DD_z^\alpha\DD_{\bar{z}}^\beta(\bar{\partial}(\chi u))\in L_{(p,q+1)}^2(G),\]
	\[\theta(\DD_z^\alpha\DD_{\bar{z}}^\beta(\chi u))=\DD_z^\alpha\DD_{\bar{z}}^\beta(\theta(\chi u))\in L_{(p,q-1)}^2(G).\]
	于是,由命题\ref{prop6.4.4}得$\DD_z^\alpha\DD_{\bar{z}}^\beta(\chi u)\in W_{(p,q)}^1(G)$,因而$\chi u\in W_{(p,q)}^{m+1}$,所以$u\in W_{(p,q)}^{m+1}(G,\loc)$.这样,我们从$u\in W_{(p,q)}^m(G,\loc)$推出了$u\in W_{(p,q)}^{m+1}(G,\loc)$.继续做下去,即得$u\in W_{(p,q)}^{s+1}(G,\loc)$.
	
	\hyperlink{6.4.5}{(2)}
	设$q=0$.设$u=\sum_I u_I\dz_I$是$\bar{\partial}u=f$的任意一个解.于是$f\in W_{(p,1)}^s(G,\loc)$意味着$\pp{u}{\bar{z}_j}=f_{I,j}\in W^s(G,\loc)$.和上面一样,假定对某个$m(0\le m\le s)$已经证明了$u\in W_{(p,0)}^m(G,\loc)$.取$\chi\in C_0^\infty(G)$,则
	\[\pp{}{\bar{z}_j}(\chi u_I)=\chi f_{I,j}+\pp{\chi}{\bar{z}_j}u_I\in W^m(G).\]
	下面的讨论和\hyperlink{6.4.5}{(1)}中完全一样,但只要用命题\ref{prop6.4.3}就行了.
\end{proof}
这一节的主要结果是下面的.
\begin{theorem}\label{thm6.4.6}
	设$G$是$\MC^n$中的拟凸域.如果$f\in C_{(p,q+1)}^\infty(G)$且$\bar{\partial}f=0$,那么存在$u\in C_{(p,q)}^\infty(G)$,使得$\bar{\partial}u=f$成立.
\end{theorem}
\begin{proof}
	定理\ref{thm6.1.7}(Sobolev引理)现在可以写成$W_{(p,q)}^{s+n+1}(G)\subset C_{(p,q)}^s(G)$,又因可微分性是一个局部性质,因而又可写成
	\begin{equation}\label{eq6.4.1}
		W_{(p,q)}^{s+n+1}(G,\loc)\subset C_{(p,q)}^s(G).
	\end{equation}
	于是,由定理\ref{thm6.4.5}和\eqref{eq6.4.1}即得本定理.
\end{proof}
定理\ref{thm6.4.6}称为内部正则性定理,在强拟凸域的情形,Kohn得到更强的结果.
\begin{theorem}\label{thm6.4.7}
	设$G$是$\MC^n$中具有$C^2$边界的有界强拟凸域.如果$f\in C_{(p,q+1)}^\infty(\bar{G})$且$\bar{\partial}f=0$,那么存在$u\in C_{(p,q)}^\infty(\bar{G})$,使得$\bar{\partial}u=f$成立.
\end{theorem}
这个定理在\ref{sec6.7.1}中要用到,它的证明就不在这里讲了.
\section{Levi问题\index{L!Levi问题}\label{sec6.5}}
作为$\bar{\partial}$问题的第一个应用,我们证明Levi猜测:拟凸域一定是全纯域.\index{L!Levi猜测}

我们从一个全纯开拓问题谈起.设$G$是$\MC^n$中的域,命$E=\{z\in G\colon z_n=0\}$,那么$E$可以看成$\MC^{n-1}$中的一个集合.设$f$是$E$上的全纯函数,问是否存在$F\in H(G)$,使得$F|_E=f$?从下面的例子可以看出,一般来说,这样的$F$是不存在的.
\begin{example}\label{exa6.5.1}
	设$G=B(0,1)\setminus \bar{B}\left(0,\frac12\right)\subset\MC^2$,即$G=\left\{z\in\MC^2\colon\frac14<|z_1|^2+|z_2|^2<1\right\}$,所以$E=\left\{(z_1,0)\colon\frac12<|z_1|<1\right\}$.命$f\colon E\to\MC$为$f(z_1,0)=\frac1{z_1-\frac12}$,当然$f\in H(E)$.如果存在$F\in H(G)$,使得$F|_E=f$,那么由推论\ref{cor1.3.6}知道,$F$是单位球$B(0,1)$上的全纯函数,因而在点$\left(\frac12,0\right)$的任一邻域中有界,这和$F|_E=f$相矛盾.这就是说$f$的全纯开拓$F$是不存在的.
\end{example}
但对拟凸域,这样的全纯开拓是存在的.
\begin{lemma}\label{lem6.5.2}
	设$G$是$\MC^n$中的拟凸域,命
	\[E=\{z\in G\colon z_n=0\},\quad \widetilde{E}=\left\{(z_1,\cdots,z_{n-1})\in \MC^{n-1}\colon(z_1,\cdots,z_{n-1},0)\in E\right\},\]
	如果记$(z_1,\cdots,z_{n-1})$的函数$f(z_1,\cdots,z_{n-1},0)$在$\widetilde{E}$上全纯,则必存在全纯函数$F\colon G\to\MC$,使得$F|_E=f$.
\end{lemma}
\begin{proof}
	记$\pi(z_1,\cdots,z_n)=(z_1,\cdots,z_{n-1})$为$\MC^n$到$\MC^{n-1}$的射影,$H=\{z\in G\colon \pi z\notin\widetilde{E}\}$.那么$E$和$H$是$G$的不相交的闭子集.作为$E$的邻域$V_1,V_2$,使得$E\subset V_1\subset V_2$,且$V_2\cap H=\varnothing$.虽然$E$不一定是$G$中的紧子集,但用证明推论\ref{cor4.6.6}的方法,仍然可以证明,存在$C^\infty$函数$\varphi\colon G\to[0,1]$,使得$\supp\varphi\subset V_2$,且在$V_1$上有$\varphi(z)\equiv1$.定义$G$上的$(0,1)$形式
	\begin{equation}\label{eq6.5.1}
		g=\begin{cases}
			\frac{(-\bar{\partial}\varphi(z))f(\pi(z),0)}{z_n},&z\in G\setminus E,\\
			0,&z\in E,
		\end{cases}
	\end{equation}
	因为$\bar{\partial}\varphi=0$在$V_1$上成立,所以$g$是属于$C^\infty$的$(0,1)$形式.由于$f$全纯,且$\bar{\partial}^2=0$,所以$\bar{\partial}g=0$.于是,由定理\ref{thm6.4.6}知,$\bar{\partial}$方程$\bar{\partial}u=g$有解$u\in C^\infty(G)$.用这个$u$构造函数
	\[F(z)=\varphi(z)f(\pi(z),0)+z_n u(z),\]
	则$\bar{\partial}F=0$,即$F\in H(G)$,且$F|_E=f$.
\end{proof}
现在证明Levi猜测.
\begin{theorem}\label{thm6.5.3}
	设$G$是$\MC^n$中的拟凸域,那么$G$一定是全纯域.
\end{theorem}
\begin{proof}
	对空间的维数用归纳法.因为$\MC$上任意域都是全纯域,所以$n=1$时定理成立.今设定理对$\MC^{n-1}$中的拟凸域成立.任取$a\in\partial G$,适当选取$b\in\MC^n$,作
	\[H=\left\{z\in\MC^n\colon\sum_{j=1}^{n}(z_j-a_j)b_j=0\right\},\]
	使得$a\in\partial(G\cap H)$.作坐标变换
	\[w_j=z_j-a_j,j=1,\cdots,n-1;\quad w_n=\sum_{j=1}^{n}(z_j-a_j)b_j,\]
	则$a$变为原点,$H$变为$\{w\in\MC^n\colon w_n=0\}$.命$E=G\cap H$,在新的坐标系下,$E=\{w\in G\colon w_n=0\}$.由于$G$是$\MC^n$中的拟凸域,容易直接证明$E$是$\MC^{n-1}$中的拟凸域,由归纳假定,它是$\MC^{n-1}$中的全纯域,因而存在$E$上的全纯函数$f$,它不能通过$a$点全纯开拓出去由引理\ref{lem6.5.2},存在$G$上的全纯函数$F$,使得$F|_E=f$,所以$F$不能通过$a$点全纯开拓出去,因而$G$是全纯域.
\end{proof}
\section{Cousin问题和除法问题\label{sec6.6}}
\subsection{Cousin问题}
单复变中有两个重要的问题:

(1)
设$G$是$\MC$中的域,$a_1,a_2,\cdots$是一列趋于$\partial G$的点.问是否存在$G$中的亚纯函数,使它在$a_j$处有给定的主要部分
\[\frac{a_{j1}}{z-a_j}+\cdots+\frac{a_{jl_j}}{(z-a_j)^{l_j}},\quad j=1,2,\cdots\]

(2)
设$G$是$\MC$中的域,$a_1,a_2,\cdots$是$G$中一列点,在$G$中没有极限点.问是否存在$G$中的全纯函数以$\{a_k\}$为其零点集?

这两个问题的答案都是肯定的,前者就是所谓的Mittag-Leffler定理\index{D!定理!Mittag-Leffler定理},后者是通过Weierstrass无穷乘积来解决的\index{W!Weierstrass无穷乘积},称为Weierstrass定理.\index{D!定理!Weierstrass定理}

在多复变中,全纯函数的零点和极点都不是孤立的,所以这两个问题在多复变中有不同的提法.

Mittag-Leffler定理在多复变中的推广即所谓的Cousin第一问题\index{C!Cousin第一问题},它的提法如下:

设$G$是$\MC^n$中的域,$\{U_j\}$是$G$的一个开覆盖,在每个$U_j$上有亚纯函数$f_j$,满足条件
\[f_j-f_k\in H(U_j\cap U_k).\]
问是否存在一个$G$上的亚纯函数$f$,使得$f-f_j$在$U_j$上是全纯函数?

Weierstrass定理在多复变中的推广即所谓的Cousin第二问题\index{C!Cousin第二问题},它的提法如下:

设$G$是$\MC^n$中的域,$\{U_j\}$是$G$的一个开覆盖,在每一个开集$U_j$上有一个全纯函数$f_j$,满足条件$\frac{f_j}{f_k}$和$\frac{f_k}{f_j}$在$U_j\cap U_k$中全纯.问是否存在一个$G$上的全纯函数$f$,使得$\frac{f_j}{f}$在$U_j$中全纯且没有零点?

这里要求$\frac{f_j}{f}$在$U_j$中没有零点且全纯,意味着$f$和$f_j$在$U_j$中有相同的零点集.

Cousin第一问题在全纯域上永远有解,Cousin第二问题则不然.
\begin{theorem}\label{thm6.6.1}
	设$G$是$\MC^n$中的全纯域,则Cousin第一问题有解.
\end{theorem}
\begin{proof}
	我们只要能在每一个$U_i$上找到全纯函数$u_i$,使得在$U_i\cap U_j$上有$f_j-f_i=u_j-u_i$,这时命
	\[f=f_i-u_i,\quad z\in U_i,\]
	则$f$是$U_i$上的亚纯函数.由于当$z\in U_i\cap U_j$时,$f_i-u_i=f_j-u_j$,所以$f$在$G$上有定义,$f$是$G$上的亚纯函数,且当$z\in U_i$时,$f-f_i=-u_i$是$U_i$上的全纯函数,因而$f$即为所求.为了找到满足上述条件$u_i$,记$f_{ij}=f_j-f_i$,则由假定$f_{ij}$在$U_i\cap U_j$上全纯.我们先找在$U_i$上是$C^\infty$的函数$g_i$,使得在$U_i\cap U_j$上有$f_{ij}=g_j-g_i$.为此,取相应于$\{U_i\}$的单位分解$\{\rho_i\}$,即一组$C^\infty$函数$\rho_i\ge0,\supp\rho_i\subset U_{i'}$且$\sum_{i=1}^{\infty}\rho_i=1$.命
	\[g_j(z)=\sum_{k=1}^{\infty}\rho_k(z)f_{kj}(z),\quad z\in U_j.\]
	容易看出,当$z\in U_j$时,如果$U_{k'}\cap U_j=\varnothing$,则$\rho_{k'}(z)=0$.因此,上面的级数只对那些$U_{k'}\cap U_j\neq\varnothing$的$k'$求和,这时$f_{k' j}$是全纯的,因而$g_j\in C^\infty(U_j)$.如果写
	\begin{align*}
		g_j(z)
		&=\sum_{k=1}^{\infty}\rho_k(z)(f_j(z)-f_k(z))\\
		&=f_j(z)-\sum_{k=1}^{\infty}\rho_k(z)f_k(z),\quad z\in U_j,
	\end{align*}
	那么当$z\in U_i\cap U_j$时,便有
	\[g_j(z)-g_i(z)=f_j(z)-f_i(z).\]
	命$\omega=\bar{\partial}g_i,z\in U_i$,由于当$z\in U_i\cap U_j$时,$\bar{\partial}g_j-\bar{\partial}g_i=\bar{\partial}f_{ij}=0$,即$\bar{\partial}g_j=\bar{\partial}g_i$,所以$\omega$在$G$上有定义,且$\omega\in C_{(0,1)}^\infty(G)$.由于$G$是全纯域,即拟凸域,所以$\bar{\partial}$方程
	\[\bar{\partial}u=\omega\]
	在$G$上有解$u\in C^\infty(G)$.今在$U_i$上定义$u_i=g_i-u$,因为$\bar{\partial} u_i=\bar{\partial} g_i-\bar{\partial}u=0$,所以$u_i\in H(U_i)$,而且
	\[u_j-u_i=g_j-g_i=f_{ij}=f_j-f_i.\]
	因而$f=f_i-u_i(z\in U_i)$即为所求之解.
\end{proof}
Cousin第二问题在全纯域上一般是没有解的,Oka曾经举出过使Cousin第二问题无解的全纯域的例子(见\cite[p.250]{krantz2001function}).但我们有下面的
\begin{theorem}\label{thm6.6.2}
	设$G$是$\MC^n$中的全纯域.如果存在$g\in C^\infty(G)$,使得$\frac{f_i}{g}$在$U_i$上是$C^\infty$的且无零点,则Cousin第二问题有解.
\end{theorem}
\begin{proof}
	命$f_{ij}=\frac{f_j}{f_i}=\frac{\frac{f_j}{g}}{\frac{f_i}{g}}$.根据假定,$\frac{f_j}{g}$与$\frac{f_i}{g}$在$U_i\cap U_j$上都是$C^\infty$的且无零点,因而
	\[\log f_{ij}=\log\frac{f_j}{g}-\log\frac{f_i}{g},\quad z\in U_i\cap U_j.\]
	两端可能相差$2\pi\ii$的整数倍.因为$\frac{f_j}{f_i}$在$U_i\cap U_j$中全纯且无零点,故$\log f_{ij}$在$U_i\cap U_j$中全纯.因而$\bar{\partial}\log f_{ij}=0$,或者$\bar{\partial}\log\frac{f_j}{g}=\bar{\partial}\log\frac{f_i}{g}$在$U_i\cap U_j$上成立.在$U_i$上定义$(0,1)$形式如下:
	\[\omega=\bar{\partial}\log\frac{f_i}{g},\quad z\in U_i.\]
	由于当$z\in U_i\cap U_j$时,$\bar{\partial}\log\frac{f_i}{g}=\bar{\partial}\log\frac{f_j}{g}$,所以$\omega$在$G$上有定义,且$\omega\in C_{(0,1)}^\infty(G)$.从$\bar{\partial}^2=0$知$\bar{\partial}\omega=0$.故由定理\ref{thm6.4.6}知$\bar{\partial}$方程$\bar{\partial}u=\omega$有解$u\in C^\infty(G)$.今在$U_i$上命$v_i=\log\frac{f_i}{g}-u$,因为
	\[\bar{\partial}v_i=\bar{\partial}\log\frac{f_i}{g}-\bar{\partial}u=0,\]
	所以$v_i\in H(U_i)$.今在$U_i$上定义
	\[f=\frac{f_i}{\ee^{v_i}},\]
	则$f\in H(U_i)$.当$z\in U_i\cap U_j$时,
	\[\frac{\ee^{v_j}}{\ee^{v_i}}=\frac{\ee^{\log\frac{f_j}{g}-u}}{\ee^{\log\frac{f_i}{g}-u}}=\frac{f_j}{f_i},\]
	即$\frac{f_i}{\ee^{v_i}}=\frac{f_j}{\ee^{v_j}}$.所以$f$是$G$上的全纯函数,且在$U_i$上,$\frac{f_i}{f}=\ee^{v_i}\neq0$.所以$f$就是Cousin第二问题的解.
\end{proof}
\subsection{除法问题\index{C!除法问题}}
作为Cousin第一问题的一个应用,我们来讨论除法问题.

所谓除法问题是指:是$G$是$\MC^n$中的域,$0\in G,f$是$G$上的全纯函数,满足$f(0)=0$.问是否存在$G$中的全纯函数$g_1,\cdots,g_n$,使得
\begin{equation}\label{eq6.6.1}
	f(z)=\sum_{j=1}^{n}z_jg_j(z)
\end{equation}
在$G$中成立?

显然,由于$f\in H(G)$且$f(0)=0$,把$f$在原点附近展开成幂级数就能得到\eqref{eq6.6.1}.但只能断言在原点附近成立,至于\eqref{eq6.6.1}是否能在$G$中成立是不容易回答的.

这里我们将证明,如果$G$是$\MC^n$中的拟凸域,那么除法问题有解.这一结论在下面讨论$\bar{\partial}$问题解的一致估计时将要用到.

为此,我们先证明两个引理.
\begin{lemma}\label{lem6.6.3}
	设$G$是$\MC^n$中的拟凸域.如果集合
	\[E=\{z\in G\colon z_1=0\}\]
	不空,那么每一个$E$上的全纯函数都可全纯开拓到$G$上.
\end{lemma}

这就是引理\ref{lem6.5.2},我们已经证明过.现在作为Cousin第一问题应用的例子,给出一个新的证明.
\begin{proof}
	任取$\varphi\in H(E)$,对每个$z\in E$,存在$\MC^n$中的邻域$U_z$,使得$\varphi$在$U_z$中全纯.命$U=\bigcup_{z\in E}U_z$,则$U$是$E$的一个邻域,$\varphi$在$U$上全纯,因为$\frac{\varphi}{z_1}$是$U$上的一个亚纯函数.命$U_1=U\cap G,f_1=\frac{\varphi}{z_1},U_2=G\setminus E,f_2=0$,则$\{U_1,U_2\}$构成$G$的一个开覆盖,$f_1$在$U_1$上亚纯,$f_2$在$U_2$上全纯,且在$U_1\cap U_2$上,$f_2-f_1=-\frac{\varphi}{z_1}$全纯.由于$G$是拟凸域,Cousin第一问题有解,即在$G$上存在亚纯函数$\widetilde{f}$,使得$\widetilde{f}-f_j\in H(U_j),j=1,2$.因为$f_2=0$,故$\widetilde{f}$在$U^2$上全纯.在$U_1$上,函数
	\begin{equation}\label{eq6.6.2}
		h(z)=\widetilde{f}-\frac{\varphi}{z_1}
	\end{equation}
	是全纯的,因而$z_1\widetilde{f}=\varphi(z)+z_1h(z)$也在$U_1$上全纯.当然$z_1\widetilde{f}$也在$U_2$上全纯,所以$z_1\widetilde{f}\in H(G)$.命$f=z_1\widetilde{f}$,当$z\in E$时,由\eqref{eq6.6.2}知,$z_1h(z)=f(z)-\varphi(z)$.但此时$z_1=0$,故$f(z)=\varphi(z)$,所以$f$是$\varphi$在$G$上的一个全纯开拓.
\end{proof}
\begin{lemma}\label{lem6.6.4}
	设$G$是$\MC^n$中的拟凸域.如果集合
	\[E=\{z\in G\colon z_1=z_2=\cdots=z_k=0\},\quad 1\le k\le n\]
	不空,那么对每个$f\in H(G)$且满足$f|_E=0$者,必存在$g_1,\cdots,g_k\in H(G)$,使得
	\[f(z)=\sum_{j=1}^{k}z_jg_j(z)\]
	在$G$上成立.
\end{lemma}
\begin{proof}
	用归纳法证明.当$k=1$时,命
	\[g_1(z)=\begin{cases}
		\pp{f}{z_1}(0),&z\in E,\\
		\frac{f(z)}{z_1},&z\in G\setminus E,
	\end{cases}\]
	则$g_1\in H(G)$,且$f=z_1g_1$在$G$上成立,即引理当$k=1$时成立.现设$k\le m-1$时引理成立.命
	\[G_m=\{z\in G\colon z_m=0\}.\]
	因为$G$是拟凸域,所以$G_m$的每个连通分支在$\MC^{n-1}$中是拟凸域.考虑$n-1$个变量的函数
	\[F=f(z_1,\cdots,z_{m-1},0,z_{m+1},\cdots,z_n),\]
	它在$G_m$中全纯,由$f|_E=0$得知$F$在
	\[\{z\in G_m\colon z_1=z_2=\cdots=z_{m-1}=0\}\]
	上取零值.由归纳假定,存在$\widetilde{g}_j\in H(G_m),j=1,\cdots,m-1$,使得
	\[F=\sum_{j=1}^{m-1}z_j\widetilde{g}_j(z_1,\cdots,z_{m-1},z_{m+1},\cdots,z_n).\]
	由引理\ref{lem6.6.3}知,每个$\widetilde{g}_j$对应着一个$g_j\in H(G)$,使得当$z\in G_m$时,$\widetilde{g}_j(z)=g_j(z)$.作函数
	\[\varphi(z)=f(z)-\sum_{j=1}^{m-1}z_jg_j(z),\]
	显然$\varphi\in H(G)$,且当$z\in G_m$时
	\[\varphi(z)=F(z)-\sum_{j=1}^{m-1}z_j\widetilde{g}_j(z)=0.\]
	再应用引理$k=1$的情形,即得$\varphi(z)=z_mg_m(z)$,这里$g_m(z)\in H(G)$.由此即得
	\[f(z)=\sum_{j=1}^{m}z_jg_j(z).\]
	证明完毕.
\end{proof}
现在很容易证明我们的主要定理.
\begin{theorem}\label{thm6.6.5}
	设$G$是$\MC^n$中的拟凸域,则除法问题有解.
\end{theorem}
\begin{proof}
	因为$f\in H(G),f(0)=0$,由于$0\in G$,所以
	\[E=\{z\in G\colon z_1=\cdots=z_n=0\}=\{0\}\]
	不空.$f(0)=0$即$f|_E=0$.在引理\ref{lem6.6.4}中取$k=n$即得定理的证明.
\end{proof}
\section{$\bar{\partial}$问题解的一致估计\index{Y!一致估计}\label{sec6.7.1}}
\subsection{Henkin核的构造}
前面我们用$L^2$估计的方法,证明在拟凸域上$\bar{\partial}$问题有解,而且得到了解的$L^2$估计(见定理\ref{thm6.2.5}\hyperlink{6.2.5}{(1)}).在\ref{sec4.7}我们利用复平面上的非齐次Cauchy积分公式(定理\ref{thm4.7.1})导出了$\MC$中有界域上$\bar{\partial}$问题解的积分表示,并由此而得到了解的一致估计(定理\ref{thm4.7.2}).这一节的目的是要把定理\ref{thm4.7.2}的结果推广到$\MC^n$中去,得到$\MC^n$中有界强拟凸域上$\bar{\partial}$问题解的积分表示,并由此而得到解的一致估计.

让我们来分析一下如何进行这项工作.\,
定理\ref{thm4.7.2}是由定理\ref{thm4.7.1}推导出来的,\,
定理\\
\ref{thm4.7.1}在$\MC^n$中的推广就是Bochner-Martinelli积分公式(定理\ref{thm4.8.2}),那么从Bochner-Martinelli积分公式出发,能否得到$\MC^n$中有界强拟凸域上$\bar{\partial}$问题解的积分表示呢?答案是否定的.关键的原因是Bochner-Martinelli核$K_{B\text{-}M}(z,\zeta)$对$z$不是全纯的,这是它与Cauchy核最大的不同之处.

这样就产生了一个新的问题,能否构造一个新的核$K(z,\zeta)$,它对变量$z$是全纯的,但同时类似于定理\ref{thm4.8.2}\eqref{eq4.8.3}的Bochner-Martinelli积分公式又能成立?

我们还是从分析Bochner-Martinelli核的构造入手.为此需要下面两个引理.
\begin{lemma}\label{lem6.7.1}
	设$s$和$g$是$\MC^n$中的开集$G$到$\MC^n\setminus\{0\}$的映射.如果函数$\varphi\colon G\to\MC$使得$g=\varphi s$成立,那么$\eta(g)=\varphi^n\eta(s)$,这里
	\[\eta(g)=\sum_{j=1}^{n}(-1)^{j-1}g_j\dif g_1\wedge\cdots\wedge[j]\wedge\cdots\wedge\dif g_n,\]
	\[\eta(s)=\sum_{j=1}^{n}(-1)^{j-1}s_j\dif s_1\wedge\cdots\wedge[j]\wedge\cdots\wedge\dif s_n.\]
\end{lemma}
\begin{proof}
	由假定$\varphi$在$G$中没有零点.今设$s_1\neq0$,则$g_1\neq0$.由于$\dif\left(\frac{g_k}{g_1}\right)=\frac1{g_1}\dif g_k-\frac{g_k}{g_1}\dif g_1,k=2,\cdots,n$,所以
	\begin{align*}
		\dif\left(\frac{g_2}{g_1}\right)\wedge\cdots\wedge\dif\left(\frac{g_n}{g_1}\right)
		&=\frac1{g_1^n}\sum_{j=1}^{n}(-1)^{j-1}g_j\dif g_1\wedge\cdots\wedge[j]\wedge\cdots\wedge\dif g_n\\
		&=\frac1{g_1^n}\eta(g).
	\end{align*}
	同样道理,
	\[\dif\left(\frac{s_2}{s_1}\right)\wedge\cdots\wedge\dif\left(\frac{s_n}{s_1}\right)=\frac1{s_1^n}\eta(s).\]
	因为$g=\varphi s$,所以$\frac{s_k}{s_1}=\frac{g_k}{g_1},k=1,2,\cdots,n$.由此即得
	\[\eta(g)=\varphi^n \eta(s).\]
	如果$s_1=0$,可取其它$s_j\neq0$代替之.
\end{proof}
\begin{lemma}\label{lem6.7.2}
	设$\theta_j=\sum_{k=1}^{n}a_{jk}\dz_k,j=1,\cdots,n$.如果$\theta_1,\cdots,\theta_n$线性相关,那么$\theta_1\wedge\cdots\wedge\theta_n=0$.
\end{lemma}
\begin{proof}
	存在不全为$0$的$c_1,\cdots,c_n$,使得$\sum_{j=1}^{n}c_j\theta_j=0$,但是
	\[\sum_{j=1}^{n}c_j\theta_j=\sum_{j=1}^{n}c_j\sum_{k=1}^{n}a_{jk}\dz_k=\sum_{k=1}^{n}\left(\sum_{j=1}^{n}c_ja_{jk}\right)\dz_k,\]
	所以$\sum_{j=1}^{n}c_ja_{jk}=0,k=1,\cdots,n$.因而$\det(a_{jk})=0$.于是
	\begin{align*}
		\theta_1\wedge\cdots\wedge\theta_n
		&=\sum_{j_1=1}^{n}a_{1j_1}\dz_{j_1}\wedge\cdots\wedge\sum_{j_n=1}^{n}a_{nj_n}\dz_{j_n}\\
		&=\sum_{j_1=1}^{n}\cdots\sum_{j_n=1}^{n}a_{1j_1}\cdots a_{nj_n}\dz_{j_1}\wedge\cdots\wedge\dz_{j_n}\\
		&=\sum_{(j_1\cdots j_n)}a_{1j_1}\cdots a_{nj_n}\dz_{j_1}\wedge\cdots\wedge\dz_{j_n}\\
		&=\sum_{(j_1\cdots j_n)}(-1)^{\tau(j_1\cdots j_n)}a_{1j_1}\cdots a_{nj_n}\dz_{1}\wedge\cdots\wedge\dz_{n}\\
		&=\det(a_{jk})\dz_{1}\wedge\cdots\wedge\dz_{n}=0,
	\end{align*}
	这里$\sum_{(j_1\cdots j_n)}$表示对所有的排列$(j_1\cdots j_n)$求和.
\end{proof}
现在可以来分析一下Bochner-Martinelli(下面简写$B\text{-}M$)核.从\ref{sec4.8.1}知道,$B\text{-}M$核为
\[K_{B\text{-}M}(z,\zeta)=\frac{\eta\left(\bar{\zeta}-\bar{z}\right)\wedge\omega(\zeta)}{|\zeta-z|^{2n}},\]
这里$\omega(\zeta)=\dif\zeta_1\wedge\cdots\wedge\dif\zeta_n,z\in G,\zeta\in\partial G$,所以$\zeta-z\neq0$.若取$s=\bar{\zeta}-\bar{z},g=\frac{\bar{\zeta}-\bar{z}}{|\zeta-z|^2},\varphi=\frac1{|\zeta-z|^2}$,则$g=\varphi s$,由引理\ref{lem6.7.1}得
\[\eta(g)=\frac1{|\zeta-z|^{2n}}\eta(s)=\frac{\eta\left(\bar{\zeta}-\bar{z}\right)}{|\zeta-z|^{2n}}.\]
因而$B\text{-}M$核又可改写为
\begin{align*}
	K_{B\text{-}M}(z,\zeta)
	=&\eta(g)\wedge\omega(\zeta)\\
	=&\sum_{j=1}^{n}(-1)^{j-1}g_j\dif g_1\wedge\cdots\dif g_{j-1}\wedge\dif g_{j+1}\wedge\cdots\wedge\dif g_n\wedge\omega(\zeta)\\
	=&\sum_{j=1}^{n}(-1)^{j-1}g_j(\partial g_1+\bar{\partial}g_1)\wedge\cdots(\partial g_{j-1}+\bar{\partial}g_{j-1})\wedge\\
	&(\partial g_{j+1}+\bar{\partial}g_{j+1})\wedge\cdots\wedge(\partial g_n+\bar{\partial}g_n)\wedge\omega(\zeta).
\end{align*}
因为$\omega(\zeta)=\dif\zeta_1\wedge\cdots\wedge\dif \zeta_n$是$(n,0)$形式,所以$\partial g_k\wedge\omega(\zeta)=0$,因而
\begin{equation}\label{eq6.7.1}
	K_{B\text{-}M}(z,\zeta)=\sum_{j=1}^{n}(-1)^{j-1}g_j\bar{\partial}g_1\wedge\cdots\wedge\bar{\partial}g_{j-1}\wedge\bar{\partial}g_{j+1}\wedge\cdots\wedge\bar{\partial}g_n\wedge\omega(\zeta),
\end{equation}
这里$g_j(z,\zeta)=\frac{\bar{\zeta}-\bar{z}}{|\zeta-z|^2},j=1,\cdots,n$.

现在可以从\eqref{eq6.7.1}出发来证明$B\text{-}M$核对$\zeta$而言是$\bar{\partial}$闭形式,即$\bar{\partial}K_{B\text{-}M}(z,\zeta)=0$,这是证明$B\text{-}M$积分公式关键的一步.事实上,从$g_j$的表达式立刻可得
\begin{equation}\label{eq6.7.2}
	\sum_{j=1}^{n}(\zeta_j-z_j)g_j(z,\zeta)=1,
\end{equation}
因而$\sum_{j=1}^{n}(\zeta_j-z_j)\bar{\partial}g_j(z,\zeta)=0$.这说明$n$个$(0,1)$形式$\bar{\partial}g_1,\cdots,\bar{\partial}g_n$是线性相关的.于是,从引理\ref{lem6.7.2}即得$\bar{\partial}g_1\wedge\cdots\wedge\bar{\partial}g_n=0$.这样从\eqref{eq6.7.1}立刻可得
\[\bar{\partial}K_{B-M}(z,\zeta)=\sum_{j=1}^{n}(-1)^{j-1}\bar{\partial}g_j\wedge\bar{\partial}g_1\wedge\cdots\wedge\bar{\partial}g_{j-1}\wedge\bar{\partial}g_{j+1}\wedge\cdots\wedge\bar{\partial}g_n\wedge\omega(\zeta)=0.\]
由此可见,从$B\text{-}M$核的表达式\eqref{eq6.7.1}出发证明它是$\bar{\partial}$闭形式的关键是$g_j$所满足的等式\eqref{eq6.7.2},而这些$g_j(z,\zeta)$对$z$恰恰不是全纯函数.

这样自然产生一种想法,能否构造一种新的$1$的分解(即等式\eqref{eq6.7.2})使$g_j(z,\zeta)$是$z$的全纯函数.然后用这种新的$g_j(z,\zeta)$按\eqref{eq6.7.1}那样产生一个新的核$K(z,\zeta)$,这个新的核自然是$\bar{\partial}$闭的.这就是下面构造Henkin核的基本出发点.

构造这种新的$g_j(z,\zeta)$时,需要对域附加条件才行.我们先证下面的
\begin{lemma}\label{lem6.7.3}
	设$G$是$\MC^n$中具有$C^2$边界的有界强拟凸域,则对任意给定的$\zeta\in\partial G$,存在$\widetilde{G}\supset G$,使得$\zeta$是$\widetilde{G}$的内点,而且在$\widetilde{G}$上存在全纯函数$\Phi(z,\zeta)$满足$\Phi(\zeta,\zeta)=0$\index[symbolindex]{\textbf{函数和映射}!$\Phi(z$,$\zeta)$},且当$z\in G$时,$\Phi(z,\zeta)\neq0$.
\end{lemma}
\begin{proof}
	根据定理\ref{thm5.3.13},存在$G$的定义函数$\rho$和正数$\lambda_0$,使得对任意非零的$\xi\in\MC^n$和$\zeta\in\partial G$有
	\begin{equation}\label{eq6.7.3}
		\sum_{j,k=1}^{n}\pppp{\rho(\zeta)}{z_j}{\bar{z}_k}\xi_j\bar{\xi}_k\ge\lambda_0|\xi|^2.
	\end{equation}
	今取定$\zeta\in\partial G$,让$\rho(z)$在$\zeta$附近作Taylor展开
	\begin{align*}
		\rho(z)
		=&\sum_{j=1}^{n}\pp{\rho(\zeta)}{z_j}(z_j-\zeta_j)+\sum_{j=1}^{n}\pp{\rho(\zeta)}{\bar{z}_j}(\bar{z}_j-\bar{\zeta}_j)+\\
		&\frac12\sum_{j,k=1}^{n}\pppp{\rho(\zeta)}{z_j}{z_k}(z_j-\zeta_j)(z_k-\zeta_k)+\\
		&\frac12\sum_{j,k=1}^{n}\pppp{\rho(\zeta)}{\bar{z}_j}{\bar{z}_k}(\bar{z}_j-\bar{\zeta}_j)(\bar{z}_k-\bar{\zeta}_k)+\\
		&\sum_{j,k=1}^{n}\pppp{\rho(\zeta)}{z_j}{\bar{z}_k}(z_j-\zeta_j)(\bar{z}_k-\bar{\zeta}_k)+o(|z-\zeta|^2).
	\end{align*}
	由\eqref{eq6.7.3}可得
	\[\sum_{j,k=1}^{n}\pppp{\rho(\zeta)}{z_j}{\bar{z}_k}(z_j-\zeta_j)(\bar{z}_k-\bar{\zeta}_k)\ge\lambda_0|z-\zeta|^2.\]
	若记\index[symbolindex]{\textbf{函数和映射}!$F(z$,$\zeta)$}
	\[F(z,\zeta)=\sum_{j=1}^{n}\pp{\rho(\zeta)}{z_j}(\zeta_j-z_j)-\frac12\sum_{j,k=1}^{n}\pppp{\rho(\zeta)}{z_j}{z_k}(z_j-\zeta_j)(z_k-\zeta_k),\]
	则上式可写为
	\[\rho(z)\ge-2\Re F(z,\zeta)+\lambda_0|z-\zeta|^2+o(|z-\zeta|^2).\]
	当$z\in G$时,$\rho(z)<0$,因而当$z$充分接近$\zeta$时有
	\[2\Re F(z,\zeta)>\lambda_0 |z-\zeta|^2.\]
	这说明存在$\zeta$的邻域$V_1(\zeta)$,当$z\in G\cap V_1(\zeta)$时,$F(z,\zeta)\neq0$.今取$\zeta$的邻域$V_2(\zeta)\subset V_1(\zeta)$,由推论\ref{cor4.6.6},存在非负的$C^\infty$函数$\psi$,使得当$z\in V_2(\zeta)$时,$\psi(z)\equiv1$;当$z\notin V_1(\zeta)$时,$\psi\equiv0$.命
	\[\widetilde{\rho}(z)=\rho(z)-\psi(z),\quad\widetilde{G}=\left\{z\in\MC^n\colon\widetilde{\rho}(z)<0\right\},\]
	则显然有$G\subset\widetilde{G}\subset G\cup V_1(\zeta)$,这是因为$\psi$是非负的,$G\subset\widetilde{G}$是明显的.今若取$z\in\widetilde{G}$,则$\widetilde{\rho}(z)<0$,即$\rho(z)<\psi(z)$.如果$z\notin V_1(\zeta)$,则$\psi(z)=0$,因而$\rho(z)<0$,即$z\in G$.所以$\widetilde{G}\subset G\cup V_1(\zeta)$.现在不妨设
	\[\widetilde{G}=G\cup V_3(\zeta),\quad\text{其中$V_3(\zeta)\subset V_1(\zeta)$.}\]
	根据上面的讨论,当$z\in G\cap V_3(\zeta)$时,$F(z,\zeta)\neq0$.今取$g\in C^\infty(\widetilde{G})$,使得$g$在$G$上处处不为$0$,而在$V_3(\zeta)$上,$g=F(z,\zeta)$.在$G$和$V_3(\zeta)$上分别考虑全纯函数$1$和$F(z,\zeta)$,在$G\cap V_3(\zeta)$上$\frac1{F(z,\zeta)}$和$F(z,\zeta)$都没有零点,且全纯;且在$\widetilde{G}$上存在$C^\infty$函数$g$,使得$\frac1 g$和$\frac{F(z,\zeta)}{g}=1$分别在$G$和$V_3(\zeta)$上是$C^\infty$的,且没有零点.于是,根据定理\ref{thm6.6.2},Cousin第二问题有解,即存在$\widetilde{G}$上的全纯函数$\Phi(z,\zeta)$,使得$\frac1{\Phi(z,\zeta)}$和$\frac{F(z,\zeta)}{\Phi(z,\zeta)}$分别在$G$和$V_3(\zeta)$全纯且无零点.$\frac{1}{\Phi(z,\zeta)}$在$G$上全纯,说明$\Phi(z,\zeta)$在$G$上处处不为$0$.在$V_3(\zeta)$上记$\frac{F(z,\zeta)}{\Phi(z,\zeta)}=H(z,\zeta)$,则$H(z,\zeta)$全纯且不为$0$.写$F(z,\zeta)=\Phi(z,\zeta)H(z,\zeta)$,则因$F(\zeta,\zeta)=0$,所以$\Phi(\zeta,\zeta)=0$.因而$\Phi(z,\zeta)$就是要找的函数.
\end{proof}
现在很容易证明下面的
\begin{theorem}\label{thm6.7.4}
	设$G$是$\MC^n$中具有$C^2$边界的有界强拟凸域,则存在全纯的单位分解\index{D!单位分解!全纯单位分解}
	\[1=\sum_{j=1}^{n}(\zeta_j-z_j)h_j(z,\zeta),\]
	这里$z\in G,\zeta\in\partial G,h_j(z,\zeta)(j=1,\cdots,n)$是$z\in G$的全纯函数.
\end{theorem}
\begin{proof}
	根据引理\ref{lem6.7.3},对于给定的$\zeta\in\partial G$,存在$\widetilde{G}\supset G$和$\widetilde{G}$上的全纯函数$\Phi(z,\zeta)$,它在$G$上处处不为$0,\Phi(\zeta,\zeta)=0$.作变换$w=\zeta-z$,设$\widetilde{G},G$分别被变为$\widetilde{\Omega}$和$\Omega$,则$0\in\widetilde{\Omega}$.若记
	\[\Phi(z,\zeta)=\Phi(\zeta-w,\zeta)=f(w),\]
	则$f$是$\widetilde{\Omega}$中的全纯函数,且$f(0)=\Phi(\zeta,\zeta)=0$.于是,根据定理\ref{thm6.6.5},除法问题有解,即存在$\widetilde{\Omega}$上的全纯函数$f_1,\cdots,f_n$,使得
	\[f(w)=\sum_{j=1}^{n}w_jf_j(w).\]
	若记$f_j(w)=f_j(\zeta-z)=P_j(z,\zeta)$,则上式可写为
	\[\Phi(z,\zeta)=\sum_{j=1}^{n}(\zeta_j-z_j)P_j(z,\zeta).\]
	因$\Phi(z,\zeta)$在$G$中处处不为$0$,所以$h_j(z,\zeta)=\frac{P_j(z,\zeta)}{\Phi_j(z,\zeta)}$是$G$中的全纯函数,而且有
	\[1=\sum_{j=1}^{n}(\zeta_j-z_j)h_j(z,\zeta).\qedhere\]
\end{proof}
有了这个全纯的单位分解定理,我们就可以构造所需要的核了.

现在有两种单位分解
\begin{align*}
	1
	&=\sum_{j=1}^{n}(\zeta_j-z_j)\frac{\bar{\zeta}_j-\bar{z}_j}{|\zeta-z|^2}\\
	1
	&=\sum_{j=1}^{n}(\zeta_j-z_j)h_j(z,\zeta),
\end{align*}
引进参数$\lambda$,把第一式乘$\lambda$,第二式乘$1-\lambda$,然后相加,又得一新的单位分解
\[1=\sum_{j=1}^{n}(\zeta_j-z_j)\left\{\lambda\frac{\bar{\zeta}_j-\bar{z}_j}{|\zeta-z|^2}+(1-\lambda)h_j(z,\zeta)\right\}.\]
命
\[g_j(z,\zeta)=\lambda\frac{\bar{\zeta}_j-\bar{z}_j}{|\zeta-z|^2}+(1-\lambda)h_j(z,\zeta),\quad 0\le\lambda\le1.\]

所谓Henkin核是指\index[symbolindex]{\textbf{微分形式}!$K(\lambda$,$z$,$\zeta)$}\index{H!Henkin核}
\[K(\lambda,z,\zeta)=\sum_{j=1}^{n}(-1)^{j-1}g_j\dif g_1\wedge\cdots\wedge\dif g_{j-1}\wedge\dif g_{j+1}\wedge\cdots\wedge\dif g_n\wedge\dif\zeta_1\wedge\cdots\wedge\dif \zeta_n.\]
容易看出,当$\lambda=1$时,$K(1,z,\zeta)$就是$B\text{-}M$核.

下面还要做三件事:

(1)
建立Henkin核的积分公式,即相当于$B\text{-}M$核的$B\text{-}M$积分公式.

(2)
证明Henkin核的积分公式能给出$\bar{\partial}$问题的解.

(3)
对解作出一致估计.

这里(1),(2)两件事都还容易做,最困难的是(3).
\subsection{$\bar{\partial}$问题解的Henkin积分表示}
下面就是Henkin核的积分公式.

	\begin{theorem}\label{thm6.7.5}
		设$G$是$\MC^n$中具有$C^2$边界的有界强拟凸域,$f\in C^1(\bar{G})$,则
		\begin{align*}
			f(z)=
			&\frac1{nc_n}\int_{\partial G}\bar{\partial}f\wedge\int_{0}^{1}K(\lambda,z,\zeta)\dif\lambda+\\
			&\frac1{nc_n}\int_G\bar{\partial}f\wedge K(1,z,\zeta)-\\
			&\frac1{nc_n}\int_{\partial G}f(\zeta)K(0,z,\zeta)
		\end{align*}
		在$G$上成立,这里
		\[c_n=(-1)^{\frac{n(n-1)}{2}}\frac{(2\pi\ii)^n}{n!}.\]
		
		这个公式称为Leray-Stokes公式.\index{L!Leray-Stokes公式}	
	\end{theorem}
\begin{figure}[htp]
	\centering
	\tikzset{global scale/.style={
			scale=#1,
			every node/.append style={scale=#1}
		}
	}
	\begin{tikzpicture}[>=Stealth,thick,global scale=0.8]
		\draw(-2,0)node[left]{$\lambda=1$}--(-2,6)node[left]{$\lambda=0$};
		\draw(2,0)--(2,6);
		\draw (0,6) ellipse [x radius=2, y radius=.8];
		\draw[dashed](2,0) arc [start angle =0, end angle =180, x radius=2, y radius=.8];
		\draw(2,0) arc [start angle =0, end angle=-180, x radius=2, y radius=.8];
		\draw (0,0) ellipse [x radius=1, y radius=.4];
		\node at (0,0) {$z$};
		\node at (1,-1) [below]{$B(z,\varepsilon)$};
		\draw (0.2,-0.2)--(1,-1);
		\draw[->] (-3.5,6.7)--(-3.5,0.8)node[below]{$R$};
	\end{tikzpicture}
	\captionof{figure}{\label{fig6.1}}
\end{figure}
\begin{proof}
	从$K(\lambda,z,\zeta)$的构造知道,$\dif K(\lambda,z,\zeta)=0$,所以有$\dif (f(\zeta)K(\lambda,z,\zeta))=\bar{\partial}f(\zeta)\wedge K(\lambda,z,\zeta)$.现在让$\dif (f(\zeta)K(\lambda,z,\zeta))$在下列曲面上积分,积分区域为一曲面,它无上底($\lambda=0,\zeta\in G$),但有下底($\lambda=1,\zeta\in G\setminus B(z,\varepsilon)$)和侧面($(\zeta,\lambda)\in\partial G\times[0,1]$),它的边界是$\{\lambda=0,\zeta\in\partial G\}\cup\{\lambda=1,\zeta\in\partial B(z,\varepsilon)\}$.根据Stokes公式有
	\begin{align}\label{eq6.7.4}
		&\int_{\lambda\in[0,1]\atop\zeta\in\partial G}\bar{\partial}f(\zeta)\wedge K(\lambda,z,\zeta)+\int_{\lambda=1\atop\zeta\in G\setminus B(z,\varepsilon)}\bar{\partial}f(\zeta)\wedge K(\lambda,z,\zeta)\notag\\
		=&\int_{\lambda=0\atop\zeta\in\partial G}f(\zeta)K(\lambda,z,\zeta)+\int_{\lambda=1\atop\zeta\in\partial B(z,\varepsilon)}f(\zeta)K(\lambda,z,\zeta).
	\end{align}
	左端第一项可以写成
	\[\int_{\partial G}\bar{\partial}f(\zeta)\wedge\int_{0}^{1}K(\lambda,z,\zeta)\dif\lambda,\]
	左端第二项当$\varepsilon\to0$时有
	\begin{align*}
		\int_{\lambda=1\atop\zeta\in G\setminus B(z,\varepsilon)}\bar{\partial}f(\zeta)\wedge K(\lambda,z,\zeta)
		&=\int_{G\setminus B(z,\varepsilon)}\bar{\partial}f(\zeta)\wedge K(1,z,\zeta)\\
		&\to\int_G \bar{\partial}f(\zeta)\wedge K(1,z,\zeta).
	\end{align*}
	根据定理\ref{thm4.8.2}的证明,当$\varepsilon\to0$时,右端第二项有极限
	\[\int_{\partial B(z,\varepsilon)}f(\zeta)K(1,z,\zeta)=\int_{\partial B(z,\varepsilon)}f(\zeta)K_{B\text{-}M}(z,\zeta)\to nc_nf(z).\]
	故在\eqref{eq6.7.4}中命$\varepsilon\to0$,即得所要证的公式.
\end{proof}
\begin{corollary}\label{cor6.7.6}
	设$G$是$\MC^n$中具有$C^2$边界的有界强拟凸域.如果$f\in C^1(\bar{G})\cap H(G)$,那么
	\[f(z)=-\frac1{nc_n}\int_{\partial G}f(\zeta)K(0,z,\zeta).\]
\end{corollary}
\begin{proof}
	用Leray-Stokes公式并注意$\bar{\partial}f=0$即得.
\end{proof}
应用Leray-Stokes公式容易得到$\bar{\partial}$问题解的Henkin积分表示.\index{H!Henkin积分表示}
\begin{theorem}\label{thm6.7.7}
	设$G$是$\MC^n$中具有$C^2$边界的有界强拟凸域.$f\in C_{(0,1)}^\infty(\bar{G})$且满足$\bar{\partial}f=0$,则由公式
	\begin{equation}\label{eq6.7.5}
		u(z)=\frac1{nc_n}\int_{\partial G}f\wedge\int_{0}^{1}K(\lambda,z,\zeta)\dif\lambda+\frac1{nc_n}\int_G f\wedge K(1,z,\zeta)
	\end{equation}
	所定义的$u$满足方程$\bar{\partial}u=f$,且$u\in C^\infty(G)$.
\end{theorem}
\begin{proof}
	由解的正则性定理\ref{thm6.4.7},在所设条件下,$\bar{\partial}u=f$存在解$u_1\in C^\infty(\bar{G})$.对$u_1$用Leray-Stokes公式:
	\begin{align}\label{eq6.7.6}
		u_1(z)
		=&\frac1{nc_n}\int_{\partial G}\bar{\partial}u_1 \wedge\int_{0}^{1}K(\lambda,z,\zeta)\dif\lambda+
		\frac1{nc_n}\int_G\bar{\partial}u_1 \wedge K(1,z,\zeta)-\notag\\
		&\frac1{nc_n}\int_{\partial G}u_1(\zeta)K(0,z,\zeta)\notag\\
		=&\frac1{nc_n}\int_{\partial G}f \wedge\int_{0}^{1}K(\lambda,z,\zeta)\dif\lambda+
		\frac1{nc_n}\int_G f\wedge K(1,z,\zeta)-\notag\\
		&\frac1{nc_n}\int_{\partial G}u_1(\zeta)K(0,z,\zeta)\notag\\
		=&u(z)-\frac1{nc_n}\int_{\partial G}u_1(\zeta)K(0,z,\zeta).
	\end{align}
	由于$K(0,z,\zeta)$是$z$的全纯函数,所以
	\[\bar{\partial}\left(\int_{\partial G}u_1(\zeta)K(0,z,\zeta)\right)=0.\]
	于是,从\eqref{eq6.7.6}即得$u\in C^\infty(G)$,而且
	\[\bar{\partial}u=\bar{\partial}u_1=f.\qedhere\]
\end{proof}
\subsection{$\bar{\partial}$问题解的一致估计}
最后来完成最困难的一件事,即证明由\eqref{eq6.7.5}给出的解具有一致估计.为此,先给出Henkin核的另一个表达式.
\begin{lemma}\label{lem6.7.8}
	设$\theta_j=\sum_{k_j=1}^{n}a_{jk_j}\dif\zeta_{k_j},j=1,\cdots,n-2$,那么
	\[\theta_1\wedge\cdots\wedge\theta_{n-2}
	=\sum_{1\le q_1<\cdots<q_{n-2}\le n}\det\begin{pmatrix}
		a_{1q_1} & \cdots & a_{1q_{n-2}}\\
		\vdots & & \vdots\\
		a_{n-2,q_1} & \cdots & a_{n-2,q_{n-2}}
	\end{pmatrix}\dif\zeta_{q_1}\wedge\cdots\wedge\dif\zeta_{q_{n-2}}.\]
\end{lemma}
\begin{proof}
	\[\theta_1\wedge\cdots\wedge\theta_{n-2}=\sum_{k_1=1}^{n}\cdots\sum_{k_{n-2}=1}^{n}a_{1k_1}\cdots a_{n-2,k_{n-2}}\dif\zeta_{k_1}\wedge\cdots\wedge\dif\zeta_{k_{n-2}}.\]
	因为每个$k_j(j=1,\cdots,n-2)$都从$1$变到$n$,故上式给出了所有可能的线性组合.现固定$(q_1\cdots q_{n-2})$,使得$q_1<q_2<\cdots<q_{n-2}$,我们来计算$\dif\zeta_{q_1}\wedge\cdots\wedge\dif\zeta_{q_{n-2}}$的系数.设$(k_1\cdots k_{n-2})$是$(q_1\cdots q_{n-2})$的一个排列,于是$\dif\zeta_{q_1}\wedge\cdots\wedge\dif\zeta_{q_{n-2}}$的系数为
	\[\sum_{(k_1\cdots k_{n-2})}(-1)^{\tau(k_1\cdots k_{n-2})}a_{1k_1}\cdots a_{n-2,k_{n-2}}=\det\begin{pmatrix}
		a_{1q_1} & \cdots & a_{1q_{n-2}}\\
		\vdots & & \vdots\\
		a_{n-2,q_1} & \cdots & a_{n-2,q_{n-2}}
	\end{pmatrix},\]
	由此即得所要证的等式.
\end{proof}
\begin{lemma}\label{lem6.7.9}
	Henkin核可表示为
	\[K(\lambda,z,\zeta)=\sum_{1\le q_1<\cdots<q_{n-2}\le n}\det A_{q_1\cdots q_{n-2}}\dif\lambda\wedge\dif\bar{\zeta}_{q_1}\wedge\cdots\wedge\dif\bar{\zeta}_{q_{n-2}}\wedge\dif\zeta_1\wedge\cdots\wedge\dif\zeta_n,\]
	这里$A_{q_1\cdots q_{n-2}}$是下面的$n$阶方阵
	\[A_{q_1\cdots q_{n-2}}
	=\begin{pmatrix}
		g_1 & g_2 & \cdots & g_n\\
		\pp{g_1}{\lambda} & \pp{g_2}{\lambda} & \cdots & \pp{g_n}{\lambda}\\
		\pp{g_1}{\bar{\zeta}_{q_1}} & \pp{g_2}{\bar{\zeta}_{q_1}} & \cdots &\pp{g_n}{\bar{\zeta}_{q_1}}\\
		\vdots & \vdots & & \vdots\\
		\pp{g_1}{\bar{\zeta}_{q_{n-2}}} & \pp{g_2}{\bar{\zeta}_{q_{n-2}}} &\cdots &\pp{g_n}{\bar{\zeta}_{q_{n-2}}}
	\end{pmatrix},\]
	这里$g_j(z,\zeta)=\lambda\frac{\bar{\zeta}_j-\bar{z}_j}{|\zeta-z|^2}+(1-\lambda)h_j(z,\zeta),j=1,\cdots,n$.
\end{lemma}
\begin{proof}
	根据Henkin核的定义
	\[K(\lambda,z,\zeta)=\sum_{j=1}^{n}(-1)^{j-1}g_j\dif g_1\wedge\cdots\wedge[\dif g_j]\wedge\cdots\wedge\dif g_n\wedge\omega(\zeta),\]
	这里$[\dif g_j]$表示乘积中不出现这一项.因为
	\[\dif g_j=\partial g_j+\bar{\partial}g_j+\pp{g_j}{\lambda}\dif\lambda\]
	且$\partial g_j\wedge\omega(\zeta)=0$,所以有
	\begin{align*}
		K(\lambda,z,\zeta)
		=&\sum_{j=1}^{n}(-1)^{j-1}g_j\left(\bar{\partial}g_1+\pp{g_1}{\lambda}\dif\lambda\right)\wedge\cdots\wedge\\
		&\left[\bar{\partial}g_j+\pp{g_j}{\lambda}\dif\lambda\right]\wedge\cdots\wedge\left(\bar{\partial}g_n+\pp{g_n}{\lambda}\dif\lambda\right)\wedge\omega(\zeta)\\
		=&\sum_{j=1}^{n}(-1)^{j-1}g_j\sum_{l\neq j}\bar{\partial}g_1\wedge\cdots\wedge\bar{\partial}g_{l-1}\wedge\pp{g_l}{\lambda}\dif\lambda\wedge\\
		&\bar{\partial}g_{l+1}\wedge\cdots\wedge[\bar{\partial}g_j]\wedge\cdots\wedge\bar{\partial}g_n\wedge\omega(\zeta)\\
		=&\sum_{j=1}^{n}\sum_{l<j}(-1)^{j+l}g_j\pp{g_l}{\lambda}\dif\lambda\wedge\bar{\partial}g_1\wedge\cdots\wedge[\bar{\partial}g_l]\wedge\cdots\wedge\\
		&[\bar{\partial}g_j]\wedge\cdots\wedge\bar{\partial}g_n\wedge\omega(\zeta)+\\
		&\sum_{j=1}^{n}\sum_{l>j}(-1)^{j+l-1}g_j\pp{g_l}{\lambda}\dif\lambda\wedge\bar{\partial}g_1\wedge\cdots\wedge[\bar{\partial}g_j]\wedge\cdots\wedge\\
		&[\bar{\partial}g_l]\wedge\cdots\wedge\bar{\partial}g_n\wedge\omega(\zeta).
	\end{align*}
	在第一个和中把$j$和$l$的指标对换,并注意到等式$\sum_{l=1}^{n}\sum_{j<l}a_{jl}=\sum_{j=1}^{n}\sum_{j<l}a_{jl}$,可得
	\begin{align*}
		K(\lambda,z,\zeta)=
		&\sum_{j=1}^{n}\sum_{j<l}(-1)^{j+l}g_l\pp{g_j}{\lambda}\dif\lambda\wedge\bar{\partial}g_1\wedge\cdots\wedge\\
		&[\bar{\partial}g_j]\wedge\cdots\wedge[\bar{\partial}g_l]\wedge\cdots\wedge\bar{\partial}g_n\wedge\omega(\zeta)+\\
		&\sum_{j=1}^{n}\sum_{j<l}(-1)^{j+l-1}g_j\pp{g_l}{\lambda}\dif\lambda\wedge\bar{\partial}g_1\wedge\cdots\wedge\\
		&[\bar{\partial}g_j]\wedge\cdots\wedge[\bar{\partial}g_l]\wedge\cdots\wedge\bar{\partial}g_n\wedge\omega(\zeta)\\
		=&\sum_{j=1}^{n}\sum_{j<l}(-1)^{j+l-1}\det\begin{pmatrix}
			g_j & g_l\\
			\pp{g_j}{\lambda} & \pp{g_l}{\lambda}
		\end{pmatrix}\dif\lambda\wedge\bar{\partial}g_1\wedge\cdots\wedge\\
		&[\bar{\partial}g_j]\wedge\cdots\wedge[\bar{\partial}g_l]\wedge\cdots\wedge\bar{\partial}g_n\wedge\omega(\zeta).
	\end{align*}
	由引理\ref{lem6.7.8},
	\begin{equation}\label{eq6.7.7}
		\bar{\partial}g_1\wedge\cdots\wedge
		[\bar{\partial}g_j]\wedge\cdots\wedge[\bar{\partial}g_l]\wedge\cdots\wedge\bar{\partial}g_n=\sum_{1\le q_1<\cdots<q_{n-2}\le n}\det(\ast)\dif\zeta_{q_1}\wedge\cdots\wedge\dif\zeta_{q_{n-2}},
	\end{equation}
	其中
	\[(\ast)=\begin{pmatrix}
		\pp{g_1}{\bar{\zeta}_{q_1}} & \cdots &\pp{g_{j-1}}{\bar{\zeta}_{q_1}} & \pp{g_{j+1}}{\bar{\zeta}_{q_1}} & \cdots & \pp{g_{l-1}}{\bar{\zeta}_{q_1}} & \pp{g_{l+1}}{\bar{\zeta}_{q_1}} & \cdots &\pp{g_n}{\bar{\zeta}_{q_1}}\\
		\vdots & & \vdots &\vdots & & \vdots & \vdots & & \vdots\\
		\pp{g_1}{\bar{\zeta}_{q_{n-2}}} & \cdots &\pp{g_{j-1}}{\bar{\zeta}_{q_{n-2}}} & \pp{g_{j+1}}{\bar{\zeta}_{q_{n-2}}} & \cdots & \pp{g_{l-1}}{\bar{\zeta}_{q_{n-2}}} & \pp{g_{l+1}}{\bar{\zeta}_{q_{n-2}}} & \cdots &\pp{g_n}{\bar{\zeta}_{q_{n-2}}}
	\end{pmatrix}.\]
	把\eqref{eq6.7.7}式代入$K(\lambda,z,\zeta)$的表示式,得
	\begin{align}\label{eq6.7.8}
K(\lambda,z,\zeta)=
&\sum_{1\le q_1<\cdots<q_{n-2}\le n}\left\{\sum_{j=1}^{n}\sum_{j<l}(-1)^{j+l-1}\det\begin{pmatrix}
	g_j & g_l\\
	\pp{g_j}{\lambda} & \pp{g_l}{\lambda}
\end{pmatrix}\det(\ast)\right\}\cdot\notag\\
&\dif\lambda\wedge\dif\zeta_{q_1}\wedge\cdots\wedge\dif\zeta_{q_{n-2}}\wedge\omega(\zeta).		
	\end{align}
	根据行列式的Laplace展开式,\eqref{eq6.7.8}式花括弧中的量就是$n$阶方阵$A_{q_1\cdots q_{n-2}}$的行列式,这就是要证明的.
		\end{proof}
现在可以证明一致估计的定理了.
\begin{theorem}\label{thm6.7.10}
	设$G$是$\MC^n$中具有$C^2$边界的有界强拟凸域,$f\in C_{(0,1)}^\infty(\bar{G})$且满足$\bar{\partial}f=0$,则由\eqref{eq6.7.5}确定的$\bar{\partial}$问题的解$u$具有一致估计
	\[\Vert u\Vert_\infty\le C\Vert f\Vert_\infty,\]
	其中$C$是不依赖于$f$的某个常数.
\end{theorem}
\begin{proof}
	先看\eqref{eq6.7.5}的第二项.设$f=\sum_{l=1}^{n}f_l\dif\bar{\zeta}_l$,
	\begin{align*}
		&|f\wedge K(1,z,\zeta)|\\
		=&|f\wedge K_{B-M}(z,\zeta)|=\left|\sum_{l=1}^{n}f_l\dif\bar{\zeta}_l\wedge\frac{\eta(\bar{\zeta}-\bar{z})\wedge\omega(\zeta)}{|\zeta-z|^{2n}}\right|\\
		=&\left|\sum_{l=1}^{n}f_l\dif\bar{\zeta}_l\wedge\sum_{j=1}^{n}(-1)^{j-1}\frac{\bar{\zeta}_j-\bar{z}_j}{|\zeta-z|^{2n}}\dif\bar{\zeta}_1\wedge\cdots\wedge[j]\wedge\cdots\wedge\dif\bar{\zeta}_n\wedge\omega(\zeta)\right|\\
		\le&\sum_{j=1}^{n}\frac{|\bar{\zeta}_j-\bar{z}_j|}{|\zeta-z|^{2n}}\Vert f\Vert_\infty|\omega(\bar{\zeta})\wedge\omega(\zeta)|.
	\end{align*}
	若记$\zeta_j=x_j+\ii y_j$,那么
	\[\dif\bar{\zeta}_j\wedge\dif\zeta_j=(\dx_j-\ii\dy_j)\wedge(\dx_j+\ii\dy_j)=2\ii \dx_j\wedge\dy_j,\]
	因而\index[symbolindex]{\textbf{微分形式}!$\omega(\bar{\zeta})\wedge\omega(\zeta)$}
	\begin{align*}
		\omega(\bar{\zeta})\wedge\omega(\zeta)
		&=\dif\bar{\zeta}_1\wedge\cdots\wedge\dif\bar{\zeta}_n\wedge\dif\zeta_1\wedge\cdots\wedge\dif\zeta_n\\
		&=(-1)^{\frac{n(n-1)}{2}}(2\ii)^n \dx_1\wedge\dy_1\wedge\cdots\wedge\dx_n\wedge\dy_n\\
		&=(-1)^{\frac{n(n-1)}{2}}(2\ii)^n\dif m_{2n}.
	\end{align*}
	由此即得
	\[|f\wedge K(1,z,\zeta)|\le 2^n \Vert f\Vert_\infty\sum_{j=1}^{n}\frac{|\zeta_j-z_j|}{|\zeta-z|^{2n}}\dif m_{2n}.\]
	因为$G$是有界域,故可选取充分大的$R$,使得$G\subset B(z,R)$.于是
	\begin{align}\label{eq6.7.9}
		\left|\int_G f\wedge K(1,z,\zeta)\right|
		&\le \int_{B(z,R)}|f\wedge K(1,z,\zeta)|\notag\\
		&\le 2^n\Vert f\Vert_\infty \int_{B(z,R)}\sum_{j=1}^{n}\frac{|\zeta_j-z_j|}{|\zeta-z|^{2n}}\dif m_{2n}.
	\end{align}
	现在来估计积分$\int_{B(z,R)}\frac{|\zeta_1-z_1|}{|\zeta-z|^{2n}}\dif m_{2n}$.若记$z=(x_1,\cdots,x_{2n}),\zeta=(y_1,\cdots,y_{2n})$,则$z_1=x_1+\ii x_2,\zeta_1=y_1+\ii y_2$,采用球坐标
	\[y_1-x_1=r\cos\varphi_1,\quad y_2-x_2=r\sin\varphi_1\cos\varphi_2,\cdots,\]
	\[y_{2n}-x_{2n}=r\sin\varphi_1\sin\varphi_2\cdots\sin\varphi_{2n-1},\]
	这时体积元素
	\[\dif m_{2n}=r^{2n-1}\sin^{2n-2}\varphi_1\sin^{2n-3}\varphi_2\cdots\sin\varphi_{2n-2}\dif r\dif\varphi_1\cdots\dif\varphi_{2n-1}.\]
	于是
	\[\int_{B(z,R)}\frac{|\zeta_1-z_1|}{|\zeta-z|^{2n}}\dif m_{2n}\le\int_{B(z,R)}\frac{|y_1-x_1|+|y_2-x_2|}{|\zeta-z|^{2n}}\dif m_{2n},\]
	而
	\begin{align*}
    \int_{B(z,R)}\frac{|y_1-x_1|}{|\zeta-z|^{2n}}
    &\le\int_{0}^{R}\frac{r}{r^{2n}}r^{2n-1}\dif r\int_{0}^{\pi}\int_{0}^{\pi}\cdots\int_{0}^{2\pi}\sin^{2n-2}\varphi_1\cdots\sin\varphi_{2n-2}\dif\varphi_1\cdots\dif\varphi_{2n-1}\\
    &=O(1).		
	\end{align*}
	易知其它类似的积分也都是有界量.于是,从\eqref{eq6.7.9}即得
	\begin{equation}\label{eq6.7.10}
		\left|\int_G f\wedge K(1,z,\zeta)\right|\le C_1\Vert f\Vert_\infty,
	\end{equation}
	剩下来需要证明的是
	\begin{equation}\label{eq6.7.11}
		\left|\int_{\partial G}f\wedge\int_{0}^{1}K(\lambda,z,\zeta)\dif\lambda\right|\le C_2\Vert f\Vert_\infty.
	\end{equation}
	这比证明不等式\eqref{eq6.7.10}要困难得多.首先注意,要证明\eqref{eq6.7.11},只要证明当$\lambda\in[0,1]$时,有
	\[\left|\int_{\partial G}f\wedge K(\lambda,z,\zeta)\right|\le C_2\Vert f\Vert_\infty\]
	就行了.根据引理\ref{lem6.7.9}关于Henkin核的表达式,可写
	\[f\wedge K(\lambda,z,\zeta)=\sum_{1\le q_1<\cdots<q_{n-2}\le n}f_j\det A_{q_1\cdots q_{n-2}}\dif\bar{\zeta}_j\dif\lambda\wedge\dif\bar{\zeta}_{q_1}\wedge\cdots\wedge\dif\bar{\zeta}_{q_{n-2}}\wedge\omega(\zeta).\]
	这样全部问题归结为证明
	\begin{equation}\label{eq6.7.12}
		\left|\int_{\partial G}\det A_{q_1\cdots q_{n-2}}\dif\bar{\zeta}_j\dif\lambda\wedge\dif\bar{\zeta}_{q_1}\wedge\cdots\wedge\dif\bar{\zeta}_{q_{n-2}}\wedge\omega(\zeta)\right|=O(1)
	\end{equation}
	对$z\in G$成立就行.
	
	为此,我们要计算$\det A_{q_1\cdots q_{n-2}}$.注意
	\[g_j(z)=\lambda\frac{\bar{\zeta}_j-\bar{z}_j}{|\zeta-z|^2}+(1-\lambda)\frac{P_j(z,\zeta)}{\Phi(z,\zeta)}\]
	(关于$\Phi(z,\zeta)$和$P_j(z,\zeta)$的定义分别参见引理\ref{lem6.7.3}和定理\ref{thm6.7.4}的证明),所以$\pp{g_j}{\lambda}=\frac{\bar{\zeta}_j-\bar{z}_j}{|\zeta-z|^2}-\frac{P_j}{\Phi}$.简记$a_j=\frac{\bar{\zeta}_j-\bar{z}_j}{|\zeta-z|^2},b_j=\frac{P_j}{\Phi}$,则$g_j$和$\pp{g_j}{\lambda}$可分别写成
	\[g_j=\lambda a_j+b_j-\lambda b_j,\pp{g_j}{\lambda}=a_j-b_j,\]
	如果记$a=(a_1,\cdots,a_n),b=(b_1,\cdots,b_n)$,则$\det A_{q_1\cdots q_{n-2}}$可拆分称第一、第二行分别如下的$6$个行列式的和
	\[\left(\begin{array}{c}
		\lambda a\\
		a
	\end{array}\right)+\left(\begin{array}{c}
	b\\
	a
	\end{array}\right)-\lambda\left(\begin{array}{c}
	b\\
	a
	\end{array}\right)-\lambda\left(\begin{array}{c}
	a\\
	b
	\end{array}\right)-\left(\begin{array}{c}
	b\\
	b
	\end{array}\right)+\lambda\left(\begin{array}{c}
	b\\
	b
	\end{array}\right),\]
	其中第一、五、六个为$0$,第三、四个之和为$0$,因而得
	\[\det A_{q_1\cdots q_{n-2}}=-\begin{vmatrix}
		\frac{\bar{\zeta}_1-\bar{z}_1}{|\zeta-z|^2} & \cdots &\frac{\bar{\zeta}_n-\bar{z}_n}{|\zeta-z|^2}\\
		\frac{P_1}{\Phi} & \cdots & \frac{P_n}{\Phi}\\
		\pp{g_1}{\bar{\zeta}_{q_1}} & \cdots & \pp{g_n}{\bar{\zeta}_{q_1}}\\
		\vdots & & \vdots\\
		\pp{g_1}{\bar{\zeta}_{q_{n-2}}} & \cdots & \pp{g_n}{\bar{\zeta}_{q_{n-2}}}
	\end{vmatrix}.\]
	直接算出$\pp{g_j}{\bar{\zeta}_k}$之后再用刚才打散行列式的办法得
	\begin{equation}\label{eq6.7.13}
		\det A_{q_1\cdots q_{n-2}}=-\begin{vmatrix}
			\frac{\bar{\zeta}_1-\bar{z}_1}{|\zeta-z|^2} & \cdots &\frac{\bar{\zeta}_n-\bar{z}_n}{|\zeta-z|^2}\\
			\frac{P_1}{\Phi} & \cdots & \frac{P_n}{\Phi}\\
			\frac{\lambda\delta_{1q_1}}{|\zeta-z|^2}+\frac{1-\lambda}{\Phi}\pp{P_1}{\bar{\zeta}_{q_1}} & \cdots & \frac{\lambda\delta_{nq_1}}{|\zeta-z|^2}+\frac{1-\lambda}{\Phi}\pp{P_n}{\bar{\zeta}_{q_1}}\\
			\vdots & & \vdots\\
			\frac{\lambda\delta_{1q_{n-2}}}{|\zeta-z|^2}+\frac{1-\lambda}{\Phi}\pp{P_1}{\bar{\zeta}_{q_{n-2}}} & \cdots & \frac{\lambda\delta_{nq_{n-2}}}{|\zeta-z|^2}+\frac{1-\lambda}{\Phi}\pp{P_n}{\bar{\zeta}_{q_{n-2}}}
		\end{vmatrix}.
	\end{equation}
	由此可见,$\det A_{q_1\cdots q_{n-2}}$的奇点就发生在$z=\zeta\in\partial G$处,因而只需考虑当$z\to\zeta\in\partial G$时\eqref{eq6.7.12}成立就行了.固定$\zeta\in\partial G$,已知在$\zeta$的某个邻域中,$\Phi(z,\zeta)$和$F(z,\zeta)$有相同的零点,这里
	\begin{equation}\label{eq6.7.14}
		F(z,\zeta)
		=\sum_{j=1}^{n}\pp{\rho(\zeta)}{z_j}(\zeta_j-z_j)-\frac12\sum_{j,k=1}^{n}\pppp{\rho(\zeta)}{z_j}{z_k}(z_j-\zeta_j)(z_k-\zeta_k).
	\end{equation}
	因此在估计\eqref{eq6.7.12}中的积分时,不妨将$\Phi(z,\zeta)$换成$F(z,\zeta)$.另一方面,从$\det A_{q_1\cdots q_{n-2}}$的表达式\eqref{eq6.7.13}可以看出,在它的项中,在$\zeta$处奇性最高的是
	\[\frac1{|\zeta-z|^{2(n-1)-1}|F|}=\frac1{|\zeta-z|^{2n-3}|F|}.\]
	因此,只要证明
	\begin{equation}\label{eq6.7.15}
		\left|\int_{\partial G}\frac1{|\zeta-z|^{2n-3}|F|}\dif\bar{\zeta}_j\wedge\dif\bar{\zeta}_{q_1}\wedge\cdots\wedge\dif\bar{\zeta}_{q_{n-2}}\wedge\omega(\zeta)\right|=O(1),
	\end{equation}
	\eqref{eq6.7.12}也就成立了.
	
	现在设$z$在$\zeta$的邻域中,这时$\dif\rho\neq0$,不妨设$\pp{\rho}{z_1}\neq0$.在这个邻域中作坐标变换
	\[(x_1,\cdots,x_{2n})\to(u_1,\cdots,u_{2n}),\]
	这里$\zeta_j=x_{2j-1}+\ii x_{2j},w_j=u_{2j-1}+\ii u_{2j}$,命
	\[u_1=\rho(\zeta),\quad u_2=\Im F(z,\zeta),\quad u_3=x_3,\cdots,\quad u_{2n}=x_{2n}.\]
	直接计算可得这个变换的Jacobi行列式为$\left|\pp{\rho}{\zeta_1}\right|^2\neq0$.由于$\partial G=\{\zeta\colon\rho(\zeta)=0\}$.所以在这个邻域中$\partial G$变为$u_1$.在引理\ref{lem6.7.3}的证明中,我们得到过不等式
	\[\rho(z)\ge-2\Re F(z,\zeta)+\lambda_0|z-\zeta|^2+o(|z-\zeta|^2),\]
	即$2\Re F(z,\zeta)\ge|\rho(z)|+\lambda_0|z-\zeta|^2\ge\lambda_0|z-\zeta|^2$,所以
	\begin{align*}
		|F|
		&=\left\{(\Re F)^2+(\Im F)^2\right\}\ge\frac1{\sqrt{2}}(|\Re F|+|\Im F|)\\
		&\ge\frac1{\sqrt{2}}\left(\frac{\lambda_0}{2}|z-\zeta|^2+|\Im F|\right)\ge C(|z-\zeta|^2+|\Im F|).
	\end{align*}
	于是,从局部来看,\eqref{eq6.7.15}左端的积分不超过
	\[\int_{r<1}\frac{\dif u_2\cdots\dif u_{2n}}{r^{2n-3}(r^2+|u_2|)}\]
	的常数倍.用球坐标
	\[u_2=r\cos\theta_1,\quad u_3=r\sin\theta_1\cos\theta_2,\quad\cdots\quad,u_{2n}=r\sin\theta_1\cdots\sin\theta_{2n-2},\]
	则$\dif u_2\cdots\dif u_{2n}=r^{2n-2}\sin^{2n-3}\theta_1\cdots\sin\theta_{2n-3}\dif r\dif\theta_1\cdots\dif\theta_{2n-2}$.于是上述积分不超过
	\begin{align*}
		\int_{0}^{1}\int_{0}^{2\pi}\frac{\dif r\dif\theta_1}{r+|\cos\theta_1|}
		&=4\int_{0}^{1}\int_{0}^{\frac{\pi}{2}}\frac{\dif r\dif\theta_1}{r+\sin\theta_1}\le4\int_{0}^{1}\int_{0}^{\frac{\pi}{2}}\frac{\dif r\dif\theta_1}{r+\frac{2}{\pi}\theta_1}\\
		&=4\int_{0}^{\frac{\pi}{2}}\left(\log\left(1+\frac{2}{\pi}\theta_1\right)-\log\frac{2}{\pi}\theta_1\right)\dif\theta_1<\infty.
	\end{align*}
	这就证明了\eqref{eq6.7.15}成立,因而\eqref{eq6.7.12}也成立.
\end{proof}
\section*{注记}\addcontentsline{toc}{section}{注记}
1937年Oka首先在全纯凸域上证明$\bar{\partial}$问题的解是存在的,不过他是用Cousin第一问题的解存在的方式叙述的.关于拟凸域上$\bar{\partial}$问题解的存在性,第一个直接证明是1965年由L. H\"ormander\cite{hormander19652}给出的.所谓直接证明是指,它既不依赖于全纯凸域上Cousin第一问题的解,也不依赖于Levi问题的解.这里介绍的就是H\"ormander给出的方法.作为$\bar{\partial}$问题解的应用,我们给出了Cousin第一问题和第二问题的解.从历史上来看,Cousin第一问题早在1895年就由P. Cousin在乘积域上解决了,但是在以后的40年中,这个重要问题没有任何进展.直到1936年,Oka首先在多项式凸域上、稍后于1937年在一般的全纯凸域上解决了这个问题.实际上,H. Cartan已于1935年在$\MC^2$的多项式凸域上,解决了Cousin第一问题,但Oka不知道此情况.那么在非全纯凸域上,Cousin第一问题是否可能有解呢?H. Cartan\cite[536$\sim$538]{cartan1979collected}在1938年首先在$\MC^3$中构造了一个非全纯凸域,在其上Cousin第一问题有解.

Levi问题1942年首先由Oka在$\MC^2$中解决,后来Oka于1953年,H. Bremermann\cite{bremermann1954aquivalenz}于1954年,F. Norguet\cite{norguet1954domaines}于1954年独立地解决了任意维数空间$\MC^n$中的Levi问题.1958年H. Grauert\cite{grauert1958levi}用凝聚解析层的理论解决了复流形上的Levi问题.

强拟凸域上$\bar{\partial}$问题解的积分表示,在1970年前后由H. Greuert和I. Lieb\cite{grauert1970ramirezsche},G. M. Henkin\cite{henkin1969integral,henkin1970integral}和E. Ramirez\cite{ramirez1970divisions}得到.当$f$是$(0,1)$形式时,强拟凸域上解的$L^\infty$估计是由H. Grauert和I. Lieb\cite{grauert1970ramirezsche}和G. M. Henkin\cite{henkin1970integral}解决的.当$f$是$(0,q)$形式($q>1$)时,强拟凸域上解的$L^\infty$估计是由I. Lieb\cite{lieb1970cauchy}解决的.
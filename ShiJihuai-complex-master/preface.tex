

\chapter*{序\qquad 言}\addcontentsline{toc}{chapter}{序\qquad 言}

1906年Hartogs发现,在$n$个复变数的空间$\mathbb{C}^n(n>1)$中存在这样的域,在这种域中,每一个全纯函数都可解析延拓到更大的域中去.球环$\{(z_1,\cdots,z_n)\colon r^2<|z_1|^2+\cdots+|z_n|^2<R^2\}$就是这样的域.在复平面$\mathbb{C}$上,这样的域是不存在的.多复变函数论在许多方面与单复变函数论有着本质的区别(上面提到的Hartogs现象只是其中之一),这些区别使得在研究单复变函数时行之有效的工具和方法在研究多复变函数时不再适用.于是,在多复变函数的研究工作中,不断出现新的工具和方法.许多现代数学(例如,偏微分方程、泛函分析、微分几何、代数几何、李群等)的概念和方法常被用来解决多复变中的问题,而多复变的发展又推动着这些学科的发展.

鉴于多复变函数论在近代数学中的重要地位,一些高等学校希望把它作为数学系高年级本科生的选修课,或低年级研究生的基础课.编写本书的目的就是想为这个课程提供一本入门教材.

本书的第\ref{chap1}章介绍多复变量全纯函数的基础性质,阐明Hartogs现象是如何发生的.特别是,通过引入次调和函数来证明重要的Hartogs定理.第\ref{chap1}章的内容是全书的基础.第\ref{chap2}章介绍全纯映射的基本性质.除了介绍H.Cartan定理和球的全纯自同构等基本内容外,还特别介绍了多圆盘和球上的星形映射和凸映射,以及与之有关的增长定理和偏差定理.这里有不少是我国学者获得的结果.第\ref{chap3}章介绍经典的Bergman核函数.特别介绍了我国已故著名数学家华罗庚教授关于计算四类典型域的核函数的方法.多复变量Cauchy积分公式与单复变量Cauchy积分公式有着本质的区别.一般地,全纯函数在部分边界上的值就能确定它在域内的值,而且不同的域有不同的Cauchy积分公式.第\ref{chap4}章专门讨论这个问题.这里也特别介绍了华罗庚教授在这方面的杰出工作.第\ref{chap4}章最后介绍了Bochner-Martinelli积分公式,它的作用之一是为第\ref{chap6}章构造全纯的Henkin核埋下伏笔.第\ref{chap5}章介绍全纯凸域的Cartan-Thullen理论,Levi拟凸域和拟凸域,以及强拟凸域的基本知识,并讨论了它们和全纯域的关系,从而引出了重要的Levi问题.第\ref{chap6}章介绍$\bar{\partial}$问题及其应用,证明$\bar{\partial}$问题在拟凸域上是有解的,从而证明了Levi猜想.最后给出了强拟凸域上$\bar{\partial}$问题解的一致估计.

本书是作为多复变函数论的一本入门教材来编写的,凡具有数学分析、线性代数、复变函数、实变函数以及少许泛函分析知识的读者都能读懂本书(凡涉及上述内容以外的知识,诸如微分形式和Stokes公式、弱导数和Sobolev空间等,书中都作了专门的介绍).有了本书的知识作为基础,再深入到多复变的各个领域都会方便很多.希望有志于进入多复变研究领域的青年能从本书中得到一些帮助.

作者曾多次为中国科学技术大学基础数学的研究生讲授多复变函数论.本书是在这些讲稿的基础上编写的,以后又在既有高年级本科生又有研究生参加的班上讲授过.在教学过程中,学生们提出的问题和建议使作者改进了某些内容的讲法,在此作者向他们表示感谢.

参加1994年理科数学与力学教学指导委员会分析与函数论教材建设组会议的专家,特别是本书的主审人复旦大学张锦豪教授,仔细审阅了本书的初稿,提出了很多宝贵的意见,对本书的修改提供了有益的帮助.对此,作者对他们表示衷心的感谢.

由于作者水平有限,书中的缺点错误在所难免,希望得到读者的批评指正.

\vspace*{1cm}
\hfill {\kaishu 史济怀}\hspace*{1.2cm}

\hfill 1995年3月\hspace*{0.8cm}

\hfill 于中国科学技术大学

\newpage
\chapter*{前\qquad 言}\addcontentsline{toc}{chapter}{前\qquad 言}
多复变函数论基础是史济怀先生所著的一本介绍多复变的优秀书籍,也是我学习多复变时用的教材.本次是借用\href{https://github.com/yuxtech}{向禹老师}的模板重排了本书,希望为有兴趣学习多复变的学生提供一点帮助.此外,如果读者发现文中有错误,请通过\href{t-ma@edu.hse.ru}{t-ma@edu.hse.ru}与我联系.

除了正文中引用的文献外,还有\cite{龚昇1982多复变数的奇异积分}、\cite{萧荫堂2013多复变函数论}、\cite{钟家庆1983多复分析}、\cite{钟同德1986多复变函数的积分表示与多维奇异积分方程}以及\cite{grauert2012several}可供参考.

\hfill {\kaishu 马泽灵}\hspace*{1.2cm}

\hfill 2023年7月\hspace*{0.8cm}

\clearpage
